\usepackage[]{setspace}
\usepackage[]{textcomp}
\usepackage[output-decimal-marker={,}]{siunitx}
\usepackage{longtable}
\usepackage{pgfplots}
\usepackage{colortbl}

%\usepackage[left=25mm,right=35mm,top=30mm,bottom=40mm]{geometry}

% Definition eines Kommandos für mathematische Formeln
\newcommand{\formel}[1]{\begin{center}$#1$\end{center}}
\newcommand{\formelleft}[1]{$#1$}

% Definition eines Kommandos für mathematische Variablen
\newcommand{\variable}[1]{$#1$}

% Definition für Zitate mit Übersetzung
\newcommand{\citeEnglish}[3]{\seFootcite{siehe}{Seite #1, Übersetzung: #2}{#3}}

\newcommand{\anmerkung}[1]{\footnote{Anmerkung: #1}}

\newcommand{\figref}[1]{\footnote{siehe \vref{#1}.}}

% Fußnote passend einrücken 
\setlength{\footnotemargin}{1.3em}

% kein Umbruch bei Fußnote
% siehe http://texwelt.de/wissen/fragen/2421/wie-kann-ich-verhindern-dass-funoten-auf-zwei-seiten-verteilt-ausgegeben-werden
\interfootnotelinepenalty=10000

%\KOMAoption{headings}{small}
\seNoChapterSkip[11.5mm]