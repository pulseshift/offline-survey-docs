%  se-wa-input-styles-v098.tex
%
%  Joerg Baumgart 01.08.2011
%
%  Zusammenfassung und Konfiguration wichtiger Styles f\"ur die 
%  Erzeugung von Seminar-, Projekt- und Bachelorarbeiten
%
%  2012-03-12: auf Version 0.94 umgestellt
%
%
% 2012-12-13: auf Version 0.95 umgestellt
%                     Sprachoptionen englisch/deutsch zusammengef\"uhrt
%                     bchart.sty hinzugenommen
%                 
%
% 2013-01-27: auf Version 0.96 umgestellt
%                     algorithm2e-Paket integriert
%
%
% 2013-07-08: auf Version 0.97 umgstellt 
%                     utf8, Fehlerkorrekturen bei pa1
%
% 2014-02-02: auf Version 0.971 umgestellt
%                     Workaround für Fehler im KOMAScript 3.2
%                     KOMAoption listof un documentclass übernommen
%
%
%
%
% 2014-07-22: etex-Paket hinzugenommen
%
%
% 2016-04-01: Sperrvermerk und Ehrenwörtliche Erklärung aus der PO 2015
%
%
%
% 2016-12-18: auf Version 0.98 umgestellt
%                     bchart.sty und algorithm2e haben veraltete Kommandos verwendet (\sf und \bf), 
%                     die unter MacTeX 2016 nicht mehr unterstützt werden 
%

\documentclass[12pt,BCOR=10mm,headinclude=on,footinclude=off,bibliography=totoc,listof=ignorechapter]{scrreprt}
\usepackage{etex}
\usepackage[T1]{fontenc}
\usepackage[utf8]{inputenc}
\usepackage{ifthen}
% 2012-12-13
\ifthenelse{\equal{\seWaSprache}{deutsch}}{% Deutsche Einstellungen
\usepackage[ngerman]{babel}% 
}{% Englische Einstellungen
\usepackage[english]{babel}% 
}

\usepackage{lmodern}

\usepackage{tikz} % Graphikpaket, das zu pdfLaTeX kompatibel ist
\usepackage{xkeyval} % Definition von Kommandos mit mehreren optionalen Argumenten
\usepackage{listings} % Formatierung von Programmlistings
\usepackage{graphicx} % Einbinden von Graphiken
\usepackage{color}
\usepackage{\seWaPathSty/slashbox} % Diagonalen in Tabellenfeldern
\usepackage{framed} % Erzeugung schwarzer Linien am linken Rand zur Hervorhebung von Textteilen
\usepackage{caption} % Korrektes Setzen einer mehrzeiligen float-Unterschrift bei neu definierten float-Umgebungen
\usepackage{floatrow}
% 2012-12-13
\usepackage{\seWaPathSty/bchart} % Kommandos zur Erzeugung von Balkendiagrammen
% 2013-01-27 
\usepackage[boxed,ngerman]{\seWaPathSty/algorithm2e}
%\usepackage[tworuled,vlined,ngerman]{\seWaPathSty/algorithm2e}


% Es wird jeweils die sty-Datei importiert und entsprechende Konfigurationseinstellungen werden vorgenommen

\usepackage{\seWaPathSty/se-jb-scrpage2} % Formatierung der Kopf- und Fu{\ss}zeilen
\usepackage{\seWaPathSty/se-jb-footmisc}    % Fussnoten besser formatieren

\usepackage{\seWaPathSty/se-jb-glossaries-v097} % Abk\"urzungsverzeichnis, Symbolverzeichnis, Glossar
   
\usepackage{\seWaPathSty/se-jb-floatrow}    % Definition und Konfiguration von float-Umgebungen (figure, table, die neue programm-Umgebung)
% Achtung: se-jb-varioref muss nach se-jb-floatrow importiert werden; 
% andernfalls ist der counter programm f\"ur die labelformat-Anweisung noch nicht definiert   
\usepackage{\seWaPathSty/se-jb-varioref-v097}   % Definition von Querverweisen
\usepackage{\seWaPathSty/se-jb-chngcntr}   % Kapitelweise oder globale Nummerierung von Abbildungen etc.
   
\usepackage{\seWaPathSty/se-jb-listen} % Definition neuer, besser formatierter Listen
% 2014-02-02
%\usepackage{\seWaPathSty/se-jb-kommandos-v097} % neue Kommandos f\"ur Seminar-, Projekt- und Bachelorarbeiten
%\usepackage{\seWaPathSty/se-jb-kommandos-v0971} % neue Kommandos f\"ur Seminar-, Projekt- und Bachelorarbeiten
% 2016-04-01
\usepackage{\seWaPathSty/se-jb-kommandos-v0972} % neue Kommandos f\"ur Seminar-, Projekt- und Bachelorarbeiten
% 2012-12-13
\ifthenelse{\equal{\seWaSprache}{englisch}}{\usepackage{\seWaPathSty/se-jb-kommandos-englisch-v0972}}{}

% 2016-12-18
\renewcommand{\sf}{\sffamily}
\renewcommand{\bf}{\bfseries}

