\section{Lunchapp - Proof of Concept}
\label{section:realisation:lunchapp}

\subsection{Beschreibung}

- Was ist die Idee

- Warum Lunch und nicht z.B. Dienstplan

\subsection{Lastenheft}

\begin{itemize}
\item Es ist eine mobile Anwendung zu erstellen, die das Lunchmenü für die nächsten 5 Tage anzeigt. Dies beinhaltet den Namen, den Preis sowie die Allergene und Zusatzstoffe der einzelnen Menüs.
\item Zusätzlich sollen die Öffnungszeiten der Kantine angezeigt werden. 
\item Außerdem soll es möglich sein zwischen verschiedenen Kantinen zu wählen
\item Wenn die Kantine geschlossen ist, soll diese Information anstelle der Öffnungszeiten und des Lunchmenüs angezeigt werden.
\item Außerdem soll die Umfragefunktionalität von PulseShift direkt innerhalb der Anwendung zur Verfügung stehen und kein Absprung nötig sein.
\item In der Anwendung sollen Banner angezeigt werden können, die den Nutzer zum Teilen der App oder der Teilnahme an einer Umfrage bewegen. Diese sollen in für den Nutzer als zufällig empfundenen Zeitabständen angezeigt werden.
\item Für die Umfragefunktionalität soll der Benutzer mindestens auf eine Gruppe von Personen eingegrenzt werden können.
\item Der Nutzer soll von der Anwendung aktiv über die Möglichkeit zur Umfrage sowie Essensangebote informiert werden.
\item Das Design der Anwendung soll sich an den von PulseShift entwickelten und bereitgestellten Mockups orientieren.
\end{itemize}

\subsection{Pflichtenheft}

\begin{itemize}
\item Es wird eine \gls{pwa} erstellt, die auf verschiedenen Plattformen lauffähig ist. 
\item Diese ermöglicht das Anzeigen des Lunchs für die nächsten 5 Tage mit dem Namen, dem Preis sowie den Allergenen und Zusatzstoffen der einzelnen Menüs. 
\item Außerdem werden die Öffnungszeiten sowie eine mögliche Schließung der Kantine angezeigt.
\item Die Daten zum Lunchmenü und den Öffnungszeiten werden lokal und hart kodiert in einer JSON Datei gemockt. Eine Anbindung an einen Datenserver erfolgt nicht. 
\item Die Umfragefunktionalität die von PulseShift entwickelt wurde, wird in die Anwendung hineingerendert werden. Das Anzeigen der Umfrage wird nicht selbst implementiert. 
\item Der Server für die Umfragefunktionalität wird austauschbar sein. Damit ist gemeint, dass die URL beliebig definierbar ist.
\item Die Benutzereingrenzung wird nach einer Absprache mit PulseShift über die konkrete gewünschte Ausprägung implementiert.
\item Die Detailansicht der Lunchmenüs soll dynamisch in die View der Übersicht der Lunchmenüs integriert werden.
\item Es wird eine Bannerfunktionalität bereitgestellt, die dynamisch ausgelöst werden soll. Diese weißt den Nutzer auf das Teilen der Anwendung sowie die Möglichkeit zur Umfrage hin.
\item Für Android Geräte werden Pushbenachrichtigungen implementiert. Für iOS ist dies aufgrund der Realisierung als \gls{pwa} nicht möglich.
\item Die Anwendung soll lokal gecached werden, damit sie auch ohne Internetverbindung genutzt werden kann.
\end{itemize}

\subsection{EPK: Ablauf der Anwendung}

\subsection{Architektur}

\subsection{Komponenten}
\subsubsection{Push Notification}
Die Push Notification soll ermöglichen, dass der Benutzer der Progressiven Webapp Benachrichtigungen auf sein beliebiges Endgerät erhält. So kann es ermöglicht werden, die Aufmerksamkeit des Nutzers auf die App und somit auch die Umfrage zu lenken. 

Vorab muss jedoch geprüft werden, ob das Endgerät oder auch der benutze Browser die Push Notifications unterstützen. 

\begin{itemize}
\item Zunächst wird geprüft, ob der Services Worker bereit ist. 
\item Danach wird geprüft, ob dieser Services Worker in der Lage ist Push Notifications zu verarbeiten.
\item Daraufhin wird geprüft, ob das geöffnete Fenster die Push Notifications überhaupt unterstützt.
\item Im nächsten Schritt wird geprüft, ob die Push Notifications schon abgelehnt wurden. Bei diesem Schritt kann darauf hin entweder erneut gefragt werden oder einfach ohne Push Notifiactions die App bedient werden.
\end{itemize} 

Wird die Seite aufgerufen und die Push Notification sind technisch möglich wird der Benutzer mit einem Pop-Up darauf hingewiesen, dass er die Möglichkeit hat diese zu erhalten. 

Hat der Benutzer sich dazu entschlossen die Push Notifications einzuschalten, können nun von außen Nachrichten an den Benutzer gesendet werden. Dazu wird die Entwicklungs-Plattform Firesbase genutzt. Mit Hilfe von Firebase können nun die Benachrichtigungen in bestimmten zeitlichen Abständen gesendet werden.