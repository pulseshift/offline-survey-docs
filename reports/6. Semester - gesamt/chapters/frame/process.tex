\section{Ablauf des Projekts}

Das Projektziel besteht wie in \ref{sec:introduction:goals} beschrieben aus zwei Teilzielen. Analog zu diesen Teilzielen gliedert sich das Projekt in zwei Phasen:

\begin{enumerate}
\item Die erste Phase wird im 5. Theoriesemester durchgeführt. Ihr Ziel ist die konzeptionelle Erarbeitung möglicher Umfragekanäle. Dazu wird zunächst die Persona eines Offline-Mitarbeiters erarbeitet. Anschließend werden daraus Ideen zu Umfragekanälen generiert. Dies ist in \ref{chapter:ideenfindung} beschrieben. Diese Ideen werden unter verschiedenen Gesichtspunkten weiter ausgearbeitet. Die Ergebnisse werden am Ende des 5. Semesters in einem Abschlussbericht zusammengefasst. Außerdem sind Sie in \ref{chap:Lösungsportfolio} dargestellt.
\item Die zweite Phase wird im 6. Theoriesemester durchgeführt. Ihr Ziel ist die Umsetzung der Kanäle mit dem höchsten Potential als \gls{poc}. Konkret werden dazu gemeinsam mit PulseShift die Kanäle Lunchapp, Captive Portal und Newsfeed App (siehe \ref{sec:events:definition_of_channels_for_poc}) ausgewählt. Die Lunchapp und das Captive Portal werden als \gls{poc} umgesetzt (siehe \ref{section:realisation:lunchapp} und \ref{section:realisation:captive_portal}). Eine Umsetzung der Newsfeed App wird vom Projektteam und PulseShift nicht als sinnvoll angesehen. Stattdessen wird hierzu ein Recherchebericht angefertigt, der entsprechende Produkte, Lösungen und Anwendungen, die sich bereits am Markt befinden analysiert (siehe \ref{section:realisation:newsfeed_app}). 
\end{enumerate}
