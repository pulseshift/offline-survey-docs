\section{Projektstrukturplan}

\subsection{5. Semester}

Für das fünfte Semester wurde ein \gls{psp} erstellt. Auf Grund dessen Größe ist dieser nicht in diesem Dokument sondern im Anhang eingefügt. Dieser \gls{psp} beinhaltet die in \vref{organigramm_semester5} abgebildeten Arbeitsgruppen, deren einzelne Aufgaben während des kompletten fünften Semesters, der geschätzte und der tatsächliche Arbeitsaufwand für die einzelnen Aufgaben.

\subsection{6. Semester}

Das sechste Semester wurde gleichermaßen in einem \gls{psp} zusammengefasst, dieser ist ebenfalls im Anhang abgelegt. Enthalten in diesem \gls{psp} sind die in \vref{organigramm_semester6} abgebildeten Arbeitsgruppen und deren Pflichten gegenüber des Projektteams. Nicht enthalten sind die einzelnen Arbeitspakete der vier Arbeitsgruppen (Single Purpose App, Newsfeed App Research, Captive Portal und Dokumentation). Diese wurden jeweils intern bearbeitet.

\subsection{Kanbanboard}

Nach der Erstellung des \gls{psp} wurde daraus innerhalb der Webanwendung Trello ein Kanbanboard erstellt (siehe \ref{fig:frame:kanban}). In diesem Kanbanboard werden jedem Arbeitspaket die verantwortlichen Personen, die benötigten Dateien, der Bearbeitungszeitraum und auch der Bearbeitungsstatus zugeordnet werden. Hierdurch ist der Fortschritt des Projekts und die zu bearbeitenden Aufgaben für alle Mitglieder einsehbar. Durch die Möglichkeit, Kommentare zu einzelnen Arbeitspaketen hinzuzufügen, kann direktes Feedback für Aufgaben anderer Teammitglieder gegeben werden und die gebrauchte Arbeitszeit eingetragen werden. Die gezielte Nutzung dieser Möglichkeit vereinfachte das Projektmanagement erheblich.

\begin{figure}[H]
\centering
\includegraphics[width=1\textwidth]{images/trello}
\caption[Bildschirmabgriff des Kanbanboards in Trello]{Bildschirmabgriff des Kanbanboards in Trello}
\label{fig:frame:kanban}
\end{figure}
