\section{Belohnungssysteme}

\subsection{Gründe für ein Belohnungssystem}
Basierend auf der Persona, die wir erstellt haben, gehen wir davon aus, dass der typische Bandarbeiter im Unternehmen sehr gestresst ist und in erster Linie darauf bedacht ist, seine Arbeit zu machen und Geld zu verdienen, um sich und seine Familie zu ernähren. Sein Interesse, an einer Unternehmensumfrage teilzunehmen, ist dementsprechend gering. Um einen sogenannten Offline-Mitarbeiter dennoch zur Teilnahme an einer Umfrage zu motivieren, halten wir ein Belohnungssystem für unabdingbar. Deshalb haben wir uns eine Reihe von Belohnungssystemen überlegt.

\subsection{Mögliche Belohnungssysteme}
Mögliche Anreize für den Mitarbeiter könnten sein:
\begin{itemize} 
\item Gratisfußballwette
\item Kostenlose Bildzeitung
\item Tipico-Guthaben
\item Ostereiersuche/Adventskalender
\item Jukebox (Mitarbeiter darf sich ein Lied wünschen)
\item Sammelobjekte (Fußballsammelbildchen, Sammelfiguren…)
\item Witze 
\item Firmenevent
\item Nach der Arbeit Interview/Kneipe/Bierabend
\item Gewinnspiel
\item Gratissnack
\item Gratis-Getränk
\item Adventskalender
\end{itemize}

\subsection{Vor- und Nachteile einer Umsetzung als Gewinnspiel}
Nach einer Abwägung der Möglichkeiten denken wir, dass ein Gewinnspiel die beste Lösung darstellt. Der Vorteil hierbei ist, dass ein Anreiz für eine Vielzahl an Mitarbeitern geschaffen werden kann, ohne zu hohe Aufwände und Kosten zu verursachen, da nur ein kleiner Teil der Mitarbeiter tatsächlich eine Belohnung erhält. 

Problematisch ist dabei jedoch, dass genau dadurch auch der Anreiz geringer sein könnte als bei anderen Belohnungssystemen. Ein weiterer Diskussionspunkt ist die Frage nach dem Geld: Ist das Unternehmen bereit, die Kosten zu übernehmen? Zudem wird Qualität gegen Quantität eingetauscht, da der Reiz der Belohnung dazu verführen kann, die Fragen nicht mehr gewissenhaft zu beantworten, sondern nur die Belohnung kassieren zu wollen. Da PulseShift nach eigener Aussage qualitativ hochwertige Antworten bevorzugt und auch John Deere die Finanzierung und Vergabe von Belohnungen in einem Abstimmungsmeeting ablehnte, haben wir uns generell gegen die Umsetzung eines solchen Anreizsystems entschieden.