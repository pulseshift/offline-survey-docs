\chapter{Abschluss}

\section{Zusammenfassung}

Das Ziel des zweisemestrigen Projekt war die Erarbeitung eines Lösungsportfolios, dass die Teilnahme von Offline-Mitarbeitern an den Umfragen von PulseShift ermöglicht. Dazu wurde die Durchführung des Projekts entsprechend der Semestern in zwei Phasen unterteilt. In der ersten Phase sollten die konzeptionellen Grundlagen des Lösungsportfolios erarbeitet werden. In der zweiten Phase sollten darauf aufbauend die Lösungen mit dem höchsten Potential näher betrachtet und umgesetzt werden.

Dazu wurden im 5. Semester als konzeptionelle Basis zuerst die Zielgruppe in Form einer Persona analysiert. Daraus wurden Ideen zu möglichen Lösungen generiert, die anschließend konzeptionell ausgearbeitet und bewertet wurden. Dieses konzeptionell ausgearbeitete Lösungsportfolio umfasst Zettelumfragen, Umfragen auf Tablets, Single Purpose Webapps, Captive Portale und Newsfeed Apps. Diese wurden in einem Zwischenbericht zusammenfassend dargestellt.

Im 6. Semester wurde dieses konzeptionell erarbeitete Portfolio mit PulseShift diskutiert. Es wurde gemeinsam entschieden, dass die Konzepte des Captive Portal, Single Purpose Webapp und Newsfeed App das höchste Potential aufweisen. Entsprechend dem Projektziel wurden das Captive Portal und die Single Purpose Webapp in Form einer Lunchapp als \gls{poc} umgesetzt. Für die Newsfeed App wurde gemeinsam mit PulseShift entschieden, dass eine Realisierung im Hinblick auf das Aufwand-Nutzen-Verhältnis nicht sinnvoll und im Rahmen dieses Projekts nicht realistisch ist. Stattdessen wurde hier vom Projektziel in sofern abgewichen, dass bestehende Lösungen anderer Softwareanbieter hinsichtlich ihrer Eignung zur Verwendung durch PulseShift analysiert wurden.

\section{Bewertung der Zielerreichung}

Zur Bewertung der Zielerreichung werden zuerst die beiden Teilziele betrachtet:

\begin{enumerate}
\item \textit{Es sollen mögliche Kanäle konzeptionell erarbeitet werden, um Offline-Mitarbeitern die Teilnahme an der Umfrage zu ermöglichen.}

$\Rightarrow$ Dieses Ziel wurde voll erreicht.

\item \textit{Die Kanäle die das größte Potential aufweisen sollen als \gls{poc} umgesetzt werden.}

$\Rightarrow$ Dieses Ziel wurde teilweise erreicht. Zwei der drei vielversprechendsten Kanäle wurden als \gls{poc} umgesetzt. 
\end{enumerate}

Somit wurden die Teilziele des Projekts in weiten Teilen erreicht. Lediglich die Newsfeed App wurde nicht als \gls{poc} umgesetzt, da dies im Rahmen dieses Projekts nicht realistisch gewesen wäre. Stattdessen wurden verschiedene am Markt befindliche Lösungen analysiert. Hier muss allerdings in sofern relativiert werden, dass diese Entscheidung gemeinsam mit PulseShift getroffen wurde. An dieser Stelle hat die durchgeführt Analyse für PulseShift einen höheren Wert als ein möglicher \gls{poc}. Abschließend kann somit, obwohl die Ziele nicht zu 100\% erreicht wurden von einem erfolgreichen Projekt gesprochen werden.

\section{Ausblick}

Die in der ersten Projektphase konzeptionell erarbeiteten Kanäle geben PulseShift einen Überblick, wie eine Umfrage an Offline-Mitarbeiter verteilt werden kann. Außerdem kann PulseShift die Ergebnisse nutzen, um ein besseres Verständnis hinsichtlich Kosten und Nutzen der einzelnen Kanäle zu erhalten. So kann teuren Investitionen in nicht geeignete Umfragekanäle vorgebeugt werden. Gleichzeitig kann gezielt in die Umsetzung von Kanälen mit hohem Potential investiert werden. 

Die beiden entwickelten \gls{poc}s können zum Einen von PulseShift zur Präsentation bei Kunden genutzt werden. Zum Anderen verdeutlichen sie, wie die Umsetzung einer Lunchapp und eines Captive Portals aussehen und technisch realisiert werden kann. Diese Vorlagen können für eine produktive Umsetzung durch PulseShift zu einem späteren Zeitpunkt als Basis dienen. Sie legen so den Grundstein für ein auch in Zukunft erfolgreiches und innovatives Lösungsportfolio des Unternehmens PulseShift.