\chapter{Wichtigste Ereignisse}
\label{chap:events}
\section{Kick-Off Meeting mit PulseShift - 06.10.2017}
Hierbei handelt es sich um das erste gemeinsame Treffen mit PulseShift, bei dem sich das Projektteam vorgestellt und genauere Informationen über die Arbeit von PulseShift erhalten hat. Zudem wurden mögliche Projekte erläutert, die im Rahmen des DHBW Projekts durchgeführt werden könnten. Hier bestand die Auswahl zwischen der Evaluation von Chatbots zur Umfrageerhebung, einem Dashboard, das aktuelle Technologie-Themen darstellt, und der Erstellung eines PoCs, um Mitarbeiter ohne Firmenmail zu befragen. Zudem wurden die Rahmenbedingungen des DHBW Projekts erklärt und das weitere Vorgehen festgelegt, was insbesondere die Rückmeldung einer Entscheidung für eines der möglichen Projekte einschließt.

\section{Team Planung - 06.10.2017}
Direkt im Anschluss an das Treffen mit PulseShift wurde dieses intern nachbereitet. Dabei wurde sich nochmal endgültig für PulseShift als zuverlässigen Partner und einstimmig für das Erstellen eines \gls{poc} für Umfragen an Mitarbeiter ohne Firmenmail entschieden. Für das Projekt spricht der betriebswirtschaftliche Hintergrund, das Potential der Generierung verschiedener Ideen und die Entwicklung und Evaluation verschiedener \gls{poc}s angesehen.

Weiterhin wurde eine Grobplanung des Projekts erstellt. Hier wurde für das 5. Semester die nicht-technische Ausarbeitung des Themas und für das 6. Semester die konkrete Implementierung der entwickelten Ideen festgelegt. Insbesondere ein Test der Eignung für die Endanwender ist für das 6. Semester geplant. Zudem fand auch die grundsätzliche Einteilung der Zuständigkeiten statt, die im Organigramm abgebildet ist.

Zum Abschluss wurde sich auf die zu verwendenden Tools Trello, Dropbox Paper, OneNote und Github geeinigt.

\section{Projektdefinition mit PulseShift - 12.10.2017}
Am 12.10.2017 fand ein erneutes Treffen mit PulseShift statt. Um einen höheren Endanwender-Bezug zu gewährleisten, wurde ein Gespräch mit John Deere am 15.11.2017 geplant, bei dem auch nach einer möglichen Werksbesichtigung gefragt werden soll. Alternativ zu einer Werksbesichtigung wurde empfohlen, im Bekanntenkreis nach Werks- und Wartungsmitarbeitern zu fragen, um einen besseren Eindruck von Lösungsansätzen zu erhalten.

Des Weiteren wurde der Zugriff auf ein Demosystem ermöglicht, um einen Eindruck von der Lösung von PulseShift (Umfrage-WebApp) zu erhalten.

Hinsichtlich des \gls{poc} der Umfrage für Werksmitarbeiter ohne Firmenmail wurden von PulseShift bereits einige Anregungen und Ideen mitgeteilt. Vorgeschlagen wurde etwa eine App zur Anzeige des Mittagessens in der Kantine, bei der regelmäßig Umfragen eingeblendet werden. Weiterhin soll keine App erstellt werden, die nur eine Umfrage darstellt und auch Hardware soll nicht eigens gebaut werden müssen. Insbesondere die Aspekte Kosten, Aufwand und verfügbare Ressourcen sollen bei der Ideenfindung miteinbezogen werden. Auch das Konzept von PulseShift, basierend auf unaufdringlichen Umfragen Aktionen mit Mehrwert für den Kunden zu finden, soll berücksichtigt werden.

\section{Teambesprechung - 16.10.2017}
Am 16.10.2017 wurde das letzte Treffen mit PulseShift nachbereitet. Das Treffen mit John Deere soll von Jason und Philipp wahrgenommen und als Feedbackmeeting für bis dahin ausgearbeitete Ideen genutzt werden.

Des Weiteren wurden die Aufgaben des Projektmanagements, wie die Formulierung eines konkreten Projektziels sowie das Erstellen eines Organigramms und eines Projektstrukturplans, verteilt.

Abschließend wurde eine Design Thinking Session vereinbart, um Ideen zu sammeln, und ein Treffen mit Herrn Prof. Dr. Holey arrangiert, um den aktuellen Fortschritt abzustimmen.

\section{Design Thinking - 23.10.2017}
In einem Treffen, das der Methode Design Thinking folgte, wurden Ideen für die Umsetzung der Umfrage ohne Firmenmail entwickelt. Dabei wurde zunächst eine Persona erstellt, die den typischen Endanwender des \gls{poc} darstellt. Hierdurch sollen Denkanstöße für die Ideensammlung und ein besseres Verständnis für die Situation entstehen. Im Anschluss fand die eigentliche Ideengenerierung in Form eines freien Brainstormings statt. Danach wurden die Ergebnisse gemeinsam besprochen, die Umsetzbarkeit abgeschätzt und sinnvolle Ideen zur genaueren Ausarbeitung unter den Teammitgliedern aufgeteilt (detaillierte Beschreibung siehe \vref{chapter:ideenfindung}).

\section{Treffen mit Herrn Prof. Dr. Holey - 26.10.2017}
Philipp, Sebastian und Florian haben den aktuellen Stand an Herrn Prof. Dr. Holey kommuniziert. Eine schriftliche Version wird per Mail von Sebastian an Herrn Prof. Dr. Holey weitergegeben. Herr Prof. Dr. Holey zeigte sich soweit mit dem Fortschritt des Projektes zufrieden. Abschließend wurde vereinbart, dass Herr Prof. Dr. Holey regelmäßig per Mail über Updates informiert und ein Abschlussmeeting zum Ende des 5. Semesters geplant wird.

\section{Besprechung der ausgearbeiteten Anwendungen - 03.11.2017}
Als erstes wurden drei generelle Ansätze als Umfrage App, die von jedem zuhause erarbeitet wurden, vorgestellt. Eine Newsfeed App (ähnlich zu Twitter) für Unternehmen, die aktuelle Nachrichten verteilt, Chats ermöglicht und Umfragen kaskadiert (siehe \vref{section:newsfeed_app}), eine WebApp die als Hauptinformationsquelle für Mitarbeiter dient und gleichzeitig die Teilnahme an Unternehmensumfragen ermöglicht (siehe \vref{section:lunchapp}) und abschließend eine Umfrage-App auf einem Tablet, das an stark frequentierten Orten innerhalb des Unternehmens aufgestellt werden kann (siehe \vref{section:tablets}). Des Weiteren wurde für alle drei Ansätze eine Diskussionsrunde eröffnet, in der sowohl Vorteile als auch Nachteile herausgearbeitet wurden. Abschließend wurden noch mögliche Fragen für das Treffen mit John Deere erarbeitet.

\section{Treffen mit John Deere - 15.11.2017}
Hierbei handelte es sich um ein Meeting mit drei Mitarbeitern der Organisationsabteilung von John Deere. Dabei wurden die erarbeiteten Ansätze vorgestellt, um direkt Feedback zu diesen zu erhalten. Die wichtigsten Verbesserungsvorschläge der Mitarbeiter von John Deere waren dabei, dass Belohnungen für abgeschlossene Umfragen höchstens passiv vergeben werden, Umfragen nicht erzwungen werden dürfen und ehrliche Antworten der Mitarbeiter extrem wichtig sind.

\section{Nachbearbeitung des Treffens mit John Deere - 17.11.2017}
Dieses Treffen war ein vorläufiges Abschlussmeeting für unser Projektteam. Das Meeting mit John Deere wurde besprochen und Arbeitspakete aus dem Feedback der John Deere Mitarbeiter erstellt. Weiterhin wurde festgelegt, welche finalen Schritte zum Abschluss der ersten Arbeitsphase (5. Semester) noch abgehandelt werden müssen und welche Dokumente zusammengefasst an Prof. Dr. Holey geschickt werden sollen.

\section{Abschluss des 5. Semesters und weiteres Vorgehen}
Zum Abschluss des 5. Semesters haben wir die wichtigsten Ergebnisse unserer Planungsphase zusammengefasst und diskutiert. Dabei wurden wichtige Ansätze zur Implementierung während des 6. Semesters erarbeitet. Dementsprechend ist die Arbeit des 5. Semesters als Projektstart und Projektplanung zu sehen, im 6. Semester erfolgt dann die Projektumsetzung basierend auf den Ergebnissen des 5. Semesters.

\section{Teammeeting - 22.02.2018}
Das Teammeeting diente als erneuter Kickoff des Projekts und initiierte die Umsetzungsphase. Dabei wurde festgelegt, dass alle Teammitglieder die im fünften Semester erstelle Dokumentation erneut durchlesen, um wieder mit der Thematik vertraut zu werden. Des Weiteren wurde das nächste Treffen mit PulseShift für den 08.03 festgesetzt, wobei bereits am 26.02 ein Treffen zur erneuten teaminternen Absprache erfolgt. Ferner wurde festgestellt, dass die bisherigen geschätzen Zeiten der einzelnen Arbeitspakete viel zu gering ausfielen und dementsprechend bessere Schätzungen erfolgen müssen. Abschließend wurde über die Firma \textit{iFeedback} und deren Umfragelösungen diskutiert.

\section{Diskussion der Umfragekanäle - 26.02.2018}
Um einen oder mehrere Umfragekanäle für die Entwicklung von Prototypen auszuwählen, wurden die verschiedenen Möglichkeiten in der Gruppe diskutiert. Zentrale Aspekte waren dabei, ob sich die Umsetzung hinsichtlich der Kosten und der tatsächlichen Nutzung durch die Mitarbeiter lohnt, welche Gruppenmitglieder welche Aufgaben übernehmen können und welche Technologien sich zur Umsetzung eignen.

\section{Treffen mit PulseShift - 08.03.2018}
\label{sec:events:definition_of_channels_for_poc}
Mit allen Teammitgliedern des Projektes und mehreren Vertretern von PulseShift wurde bei diesem Treffen die erarbeiteten Ideen vorgestellt. Dabei wurde von PulseShift detailliertes Feedback für die folgenden Umfragekanäle gegeben:
\begin{itemize}
\item Captive Portal
\item Newsfeed App
\item Single Purpose App
\item Tablet
\item QR-Code
\end{itemize}
Folgend wurden für den weiteren Projektverlauf von PulseShift folgende drei Ansätze am besten bewertet:
\begin{itemize}
\item Es soll eine \textbf{Captive Portal} Demo entwickelt werden.
\item Es soll eine Demo für eine \textbf{Single Purpose App} entwickelt werden.
\item Bezüglich der \textbf{Newsfeed App}  sollen die wichtigsten Lösungen am Markt analysiert werden.
\end{itemize}

\section{Nachbereitung PulseShift-Meeting und Aufgabenverteilung - 08.03.2018}
In Folge des Abstimmungsgesprächs mit PulseShift am 08.03.2018 wurden folgende Aufgabenbereiche definiert und unter den Gruppenmitgliedern aufgeteilt:
\begin{itemize}
\item Entwicklung einer Single Purpose App: Philipp, Jan, Julia(0,5)
\item Entwicklung eines Captive Portal: Henrik, Florian(0,5)
\item Evaluation/Research hinsichtlich der Newsfeed App: Jason, Julia(0,5)
\item Dokumentation und Projektmanagement der neu gebildeten Arbeitsgruppen: Sebastian, Florian(0,5)
\end{itemize}
Dieser Gruppenaufbau ist in \vref{organigramm_semester6} visualisiert.

\section{Besprechung eines Zwischenstands mit Herrn Holey - 26.03.2018}
Am 26.03 hat ein Treffen zwischen Projektteam und Herrn Holey stattgefunden. Dabei wurde der aktuelle Stand des Projektes Herrn Holey erläutert und die Zwischenergebnisse präsentiert. Von der Single Purpose App wurden mehrere \gls{epk}, ein Lastenheft, ein Pflichtenheft und alle Arbeitspakete präsentiert. Die Newsfeed App Research Arbeitsgruppe konnte erste Forschungsergebnisse und mögliche Einbindungsmöglichkeiten von Umfragen in diese Apps präsentieren. Für die Realisierung des Captive Portal wurden zwei Konzepte ausgearbeitet, allerdings wurde zu dieser Zeit auf die Bereitstellung der Hardware seitens PulseShift gewartet. Zur Dokumentation der einzelnen Arbeitsvorgänge wurde ein einheitlicher Rahmen vorgegeben und ein Latexdokument erstellt.

\section{Syncmeeting nach Arbeit an den Projekten - 10.04.2018}
Zur Präsentation der bisher geleisteten Ergebnisse, wurden am 10.04 alle Zwischenergebnisse der einzelnen Teams innerhalb des Projektteams besprochen. Die Captive Portal Funktionalität wurde auf einem Raspberry Pi implementiert, wobei noch wenige Anzeigefehler bei Samsung Handys auftreten und noch keine original Umfrage von PulseShift angezeigt wird. Die Newsfeed Forschung hat die Applikationen \textit{Staff Hub}, \textit{Xing}, \textit{linkedin} und \textit{Slack} untersucht, wobei eine genauere Analyse für \textit{Staff Hub} ausgearbeitet wurde. Die Single Purpose App wurde in Form einer Lunch App fast fertig gestellt, wobei die Umfragen von PulseShift durch ein iFrame noch nicht angezeigt werden können. Ferner wurde die Terminierung aller technischen Bearbeitungen auf den 16.04 und die Abfertigung der Dokumentation auf den 23.04 festgelegt.