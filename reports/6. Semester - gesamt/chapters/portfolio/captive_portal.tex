\section{Captive Portal}


\subsection{Beschreibung}
Ein sogenanntes Captive Portal wird vor allem in öffentlichen Bereichen eingesetzt, in denen Zugang zu einem WLAN-Netzwerk gewährt wird. Der Nutzer wird nach dem Verbindungsaufbau automatisch auf eine Webseite geleitet, auf der er z.B. Richtlinien akzeptieren muss. Dies können wir uns zu Nutze machen. Bietet das Unternehmen WLAN an, können sich die Mitarbeiter mit ihrem Smartphone verbinden und werden anschließend durch das Captive Portal zu den Fragen von PulseShift geleitet. Ein anderer Anwendungsfall ist der Einsatz bei einer Gesamtveranstaltung. Dabei werden die Router am Veranstaltungsort platziert und die Mitarbeiter anschließend gebeten, sich dort anzumelden und die Umfrage durchzuführen.

\subsection{Bewertung des Projektteams}
Das Potential liegt unserer Ansicht nach vor allem bei dem Einsatz während Gesamtveranstaltungen. Dadurch würden Zettel für Umfragen nicht mehr benötigt werden, jedoch könnten solche Umfragen auch nicht sehr häufig erfolgen. Der Einsatz in Kantinen oder Pausenräumen bietet sich auch an und ist mit geringem Aufwand umsetzbar, allerdings ist nach unseren Erkenntnissen die Motivation der Mitarbeiter äußerst gering, die Umfrage während der Pause zu machen. Zusätzlich gehen maximal 10\% der Werksmitarbeiter überhaupt zum Mittagessen in die Kantine.
Wir halten das Captive Portal für eine sinnvolle Ergänzung. Es kann aber aus oben genannten Gründen nicht als einziges Instrument zur Durchführung von Umfragen eingesetzt werden.

\subsection{Feedback und Bewertung durch PulseShift}
Das Captive Portal wurde von PulseShift enorm positiv bewertet. Darüber hinaus erhielt die Idee eines Captive Portals auch von HR-Experten überaus positives Feedback. Durch eine Erweiterung des Umfragenportfolios von PulseShift, könnten mit dem Captive Portal Nutzer über diverse Endgeräte an Umfragen teilnehmen. So könnte beispielsweise eine Hardwarelösung mit implementierten Captive Portal während einer Versammlung genutzt werden, um alle Teilnehmer der Versammlung über deren Endgeräte zu befragen. Das Captive Portal sollte entweder durch eine Weiterleitung zu den von PulseShift gehosteten Umfragen führen oder die entsprechenden Umfragen lokal auf der Hardwarelösung aufrufen. Ferner bot PulseShift an, eine entsprechende Hardwarelösung zur Umsetzung des Captive Portals bereitzustellen.

\subsection{Weiteres Vorgehen}
Nach dem Feedback von PulseShift hat sich das Projektteam dazu entschieden, eine Captive Portal Demo für genau eine Hardwarelösung zu realisieren. Zur Umsetzung wurde als Hardware ein Raspberry Pi festgelegt, da sich dieser als kostengünstigste und transportfähigste Alternative für die Implementierung eines Captive Portals erwiesen hat. Dabei gilt es zu beachten, dass unabhängig von den genutzten Endgeräten, eine Weiterleitung des Nutzers zu den PulseShift Umfragen durch das Captive Portal stattfinden muss. Die Realisation der Implementierung des Captive Portals auf einem Raspberry Pi, unter Beachtung der zuvor geschilderten Randbedingungen, wird in \vref{section:realisation:captive_portal} behandelt.
