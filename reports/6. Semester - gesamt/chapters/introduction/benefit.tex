\section{Erwarteter wirtschaftlicher Nutzen}

Die in diesem Projekt konzeptionell erarbeiteten Kanäle sollen PulseShift einen Überblick geben, wie eine Umfrage an Offline-Mitarbeiter verteilt werden kann. Daraus soll abgeleitet werden können, welche Umfragekanäle im Hinblick auf Kosten und Nutzen geeignet sind, um in das Lösungsportfolio von PulseShift aufgenommen zu werden. So kann teuren Investitionen in nicht geeignete Umfragekanäle vorgebeugt werden. Gleichzeitig kann gezielt in die Umsetzung von Kanälen mit hohem Potential investiert werden. 

Die entwickelten \gls{poc}s sollen zum Einen durch PulseShift zur Präsentation bei deren Kunden genutzt werden können. Zum Anderen sollen sie verdeutlichen, wie die Umsetzung der jeweiligen Kanäle aussehen kann. Wenn die Kanäle zu einem späteren Zeitpunkt durch PulseShift zum produktiven Einsatz weiter entwickelt werden, kann der Aufbau der \gls{poc}s als Vorlage dienen und gegebenenfalls Codeteile wiederverwendet werden. So kann PulseShift bei der Umsetzung Entwicklungskosten sparen.