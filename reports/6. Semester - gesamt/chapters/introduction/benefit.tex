\section{Erwarteter wirtschaftlicher Nutzen}

Die in diesem Projekt konzeptionell erarbeiteten Kanäle sollen PulseShift einen Überblick geben, wie eine Umfrage an Offline-Mitarbeiter verteilt werden kann. Daraus soll abgeleitet werden können, welche Umfragekanäle im Hinblick auf Kosten und Nutzen geeignet sind, um in das Lösungsportfolio von PulseShift aufgenommen zu werden. So kann teuren Investitionen in nicht geeignete Umfragekanäle vorgebeugt werden. Gleichzeitig kann gezielt in die Umsetzung von Kanälen mit hohem Potential investiert werden. 

Die entwickelten \gls{poc}s sollen zum einen für PulseShift als Demonstration


 aufzeigen, wie die Umsetzung der jeweiligen Kanäle aussehen kann. So 

 Zum Anderen



Die in dem Projekt entwickelten  sollen die Umfragemöglichkeiten für PulseShift bei Mitarbeitern ohne Email-Adresse gewährleisten.

 
 Weiterhin sollen die PoCs mit einer möglichst hohen Akzeptanz in der zu untersuchenden Zielgruppe anwendbar sein. 
 
 Der wirtschaftliche Mehrwert liegt somit einerseits in der kostengünstigen Umsetzung der Umfrage und der einhergehenden Kosteneinsparungen, andererseits auch in den möglichst aussagekräftigen Ergebnissen, die durch eine Steigerung der Anzahl an befragten Mitarbeitern erreicht werden. 
 
 Durch diese werden dem Unternehmen Handlungsmöglichkeiten zur besseren Durchführung von Digitalisierungsmaßnahmen aufgezeigt und somit weitere Optionen zur Kosteneinsparung dargelegt.