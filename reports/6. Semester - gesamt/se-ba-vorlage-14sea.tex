% Konfigurationsdatei f\"ur die Pfaddefinitionen einlesen
%  se-wa-pfade.tex
%
%
%  J\"org Baumgart
%  2012-12-20
%  
%  Pfaddefinitionen (Ordnerdefinitionen) f\"ur das Einlesen von
%  -- .sty-Dateien und
%  -- Textbaustenen f\"ur die Hinweise zur Verwendung von LaTeX
%  -- jpg-Bildern
%
\newcommand{\seWaPathSty}{se-wa-styles}
\newcommand{\seWaPathText}{se-wa-textbausteine-vorlagen}
\newcommand{\seWaPathJpg}{images}

%
%
% Festlegung der Sprache: 
\newcommand{\seWaSprache}{deutsch}
%\newcommand{\seWaSprache}{englisch}

%
% Einlesen der .sty-Dateien
%
%  se-wa-input-styles-v098.tex
%
%  Joerg Baumgart 01.08.2011
%
%  Zusammenfassung und Konfiguration wichtiger Styles f\"ur die 
%  Erzeugung von Seminar-, Projekt- und Bachelorarbeiten
%
%  2012-03-12: auf Version 0.94 umgestellt
%
%
% 2012-12-13: auf Version 0.95 umgestellt
%                     Sprachoptionen englisch/deutsch zusammengef\"uhrt
%                     bchart.sty hinzugenommen
%                 
%
% 2013-01-27: auf Version 0.96 umgestellt
%                     algorithm2e-Paket integriert
%
%
% 2013-07-08: auf Version 0.97 umgstellt 
%                     utf8, Fehlerkorrekturen bei pa1
%
% 2014-02-02: auf Version 0.971 umgestellt
%                     Workaround für Fehler im KOMAScript 3.2
%                     KOMAoption listof un documentclass übernommen
%
%
%
%
% 2014-07-22: etex-Paket hinzugenommen
%
%
% 2016-04-01: Sperrvermerk und Ehrenwörtliche Erklärung aus der PO 2015
%
%
%
% 2016-12-18: auf Version 0.98 umgestellt
%                     bchart.sty und algorithm2e haben veraltete Kommandos verwendet (\sf und \bf), 
%                     die unter MacTeX 2016 nicht mehr unterstützt werden 
%

\documentclass[12pt,BCOR=10mm,headinclude=on,footinclude=off,bibliography=totoc,listof=ignorechapter]{scrreprt}
\usepackage{etex}
\usepackage[T1]{fontenc}
\usepackage[utf8]{inputenc}
\usepackage{ifthen}
% 2012-12-13
\ifthenelse{\equal{\seWaSprache}{deutsch}}{% Deutsche Einstellungen
\usepackage[ngerman]{babel}% 
}{% Englische Einstellungen
\usepackage[english]{babel}% 
}

\usepackage{lmodern}

\usepackage{tikz} % Graphikpaket, das zu pdfLaTeX kompatibel ist
\usepackage{xkeyval} % Definition von Kommandos mit mehreren optionalen Argumenten
\usepackage{listings} % Formatierung von Programmlistings
\usepackage{graphicx} % Einbinden von Graphiken
\usepackage{color}
\usepackage{\seWaPathSty/slashbox} % Diagonalen in Tabellenfeldern
\usepackage{framed} % Erzeugung schwarzer Linien am linken Rand zur Hervorhebung von Textteilen
\usepackage{caption} % Korrektes Setzen einer mehrzeiligen float-Unterschrift bei neu definierten float-Umgebungen
\usepackage{floatrow}
% 2012-12-13
\usepackage{\seWaPathSty/bchart} % Kommandos zur Erzeugung von Balkendiagrammen
% 2013-01-27 
\usepackage[boxed,ngerman]{\seWaPathSty/algorithm2e}
%\usepackage[tworuled,vlined,ngerman]{\seWaPathSty/algorithm2e}


% Es wird jeweils die sty-Datei importiert und entsprechende Konfigurationseinstellungen werden vorgenommen

\usepackage{\seWaPathSty/se-jb-scrpage2} % Formatierung der Kopf- und Fu{\ss}zeilen
\usepackage{\seWaPathSty/se-jb-footmisc}    % Fussnoten besser formatieren

\usepackage{\seWaPathSty/se-jb-glossaries-v097} % Abk\"urzungsverzeichnis, Symbolverzeichnis, Glossar
   
\usepackage{\seWaPathSty/se-jb-floatrow}    % Definition und Konfiguration von float-Umgebungen (figure, table, die neue programm-Umgebung)
% Achtung: se-jb-varioref muss nach se-jb-floatrow importiert werden; 
% andernfalls ist der counter programm f\"ur die labelformat-Anweisung noch nicht definiert   
\usepackage{\seWaPathSty/se-jb-varioref-v097}   % Definition von Querverweisen
\usepackage{\seWaPathSty/se-jb-chngcntr}   % Kapitelweise oder globale Nummerierung von Abbildungen etc.
   
\usepackage{\seWaPathSty/se-jb-listen} % Definition neuer, besser formatierter Listen
% 2014-02-02
%\usepackage{\seWaPathSty/se-jb-kommandos-v097} % neue Kommandos f\"ur Seminar-, Projekt- und Bachelorarbeiten
%\usepackage{\seWaPathSty/se-jb-kommandos-v0971} % neue Kommandos f\"ur Seminar-, Projekt- und Bachelorarbeiten
% 2016-04-01
\usepackage{\seWaPathSty/se-jb-kommandos-v0972} % neue Kommandos f\"ur Seminar-, Projekt- und Bachelorarbeiten
% 2012-12-13
\ifthenelse{\equal{\seWaSprache}{englisch}}{\usepackage{\seWaPathSty/se-jb-kommandos-englisch-v0972}}{}

% 2016-12-18
\renewcommand{\sf}{\sffamily}
\renewcommand{\bf}{\bfseries}




%
% Individuelle Konfiguration des Dokumentes
%
%  Individuelle Konfiguration einer Projektarbeit/Bachelorarbeit
%
%
%
%

% 2012-10-27
%
% \"Anderung des Schrifttyps f\"ur das gesamte Dokument
%
% Das gesamte Dokument wird in einer serifenlosen Schrift gesetzt
%\renewcommand{\familydefault}{\sfdefault}
%
% Das gesamte Dokument wird in einer Serifenschrift gesetzt
% Achtung: serifenlose Schriften sind jetzt grunds\"atzlich nicht mehr nutzbar!
%
%\renewcommand{\sffamily}{\normalfont}

% 2012-12-05
%
% Verwendung des url-Pakets
% Durch den optionalen Paremeter hyphens wird eine Trennung 
% von URLs auch nach Bindestrichen erlaubt
\usepackage[hyphens]{url}


% 2012-10-27
%
%
% Literaturverzeichnis
% 
% Literaturverzeichnis gem\"ass den Vorgaben von Theisen aufbauen
\usepackage{\seWaPathSty/se-jb-jurabib-theisen} 
% Verwendung der Harvard-Zitierweise
%\usepackage{\seWaPathSty/se-jb-jurabib-harvard} 

% Weitere Optionseinstellungen f\"ur das Koma-Script
%
% Zwischen Abs\"atzen einen Abstand von 0.5 \baselineskip erzeugen
\KOMAoption{parskip}{full}
%
% Vergleiche Duden "Gliederung von Nummern, S.111" 
% DIN 5008 anschauen, wenn sie neu ver\"offentlicht wurde
\KOMAoption{numbers}{noendperiod}
%
%



%  Voreinstellungen f\"ur floats
%  Durch die verwendeten Parameter wird die Wahrscheinlichkeit deutlich kleiner, 
%  dass Gleitobjekte (z. B. Abbildungen) ans Ende des Dokumentes verschoben 
%  werden; 
%  Achtung: clearpage erzwingt die Ausgabe von Gleitobjekten
%
\renewcommand{\topfraction}{1}  % Gleitobjekte d\"urfen eine Seite zu 100% belegen 
\renewcommand{\bottomfraction}{1} % Entsprechender Wert f\"ur den unteren Teil der Seite
\renewcommand{\textfraction}{0} % Eine Seite darf auch ohne Fliesstext existieren
%%%\renewcommand{\floatpagefraction}{1} % Bedeutung unklar, daher keine Ver\"anderung des Vorgabewertes 
                                                                        % von 0.5; eventuell bringt ein \"Anderung auf 1 etwas, wenn 
                                                                         % Probleme mit floats auftreten
                                                                         
                                                                         
                                                                         
% Konfiguration von Programm-Listings
% 
% Achtung: hier gibt es nahezu beliebig viele weitere Konfigurationm\"oglichkeiten; vgl. Paketdokumentation
%
\lstset{language=Java,basicstyle=\ttfamily,keywordstyle=\color{blue},captionpos=b,aboveskip=0mm,belowskip=0mm,
          xleftmargin=0em}               
          
%
% Grundkonfiguration der Abs\"ande zwischen den Items der maximal f\"unf Verschachtelungsebenen der 
% neuen Listenumgebungen
%                                                                             
% Initialisierung der Abst\"ande zwischen den items f\"ur seList; Grundeinheit: 0.5\baselineskip; siehe se-jb-listen
\seSetlistbaselineskip{1}{0.75}{0.75}{0.75}{0.75}
% Initialisierung der Abst\"ande zwischen den items f\"ur seToplist; Grundeinheit: 0.5\baselineskip; siehe se-jb-listen
\seSettoplistbaselineskip{1}{0.75}{0.75}{0.75}{0.75}     


% Einlesen der sprachabh\"angigen Konfigurationsdatei
%
%
\ifthenelse{\equal{\seWaSprache}{deutsch}}{% deutsch
% wa-konfiguration-deutsch
%
% 2012-12-13
% 
% Diese Datei wird f\"ur die Sprachoption deutsch verwendet, d. h.  
% \newcommand{\seWaSprache}{deutsch}
%
%
% In dieser Datei k\"onnen Neudefinitionen vorgenommen werden f\"ur:
% -- Verzeichnisse
% -- Unter-/\"Uberschriften von Abbildungen, Tabellen und Listings
% -- Querverweise innerhalb des Textes

% 2013-01-26: Konfiguration des Algorithmenverzeichnis
%
%
%

% 2013-07-08: Querverweis auf Anhang hinzugenommen
%
%
%


%
%  Konfiguration der verschiedenen Verzeichnisse
%
%  abstandEintrag: Wert wird mit \baselineskip multipliziert
%

%
%  Abbildungsverzeichnis
%
\seKonfigurationAbb[
%verzeichnisname=Abbildungsverzeichnis,
labeltextLinks=, % kein Text links;
%labeltextRechts=:,
labelbreite=1cm,
%labeleinzug=1cm,
%abstandEintrag=1,
%newpage=ja,
%pnumwidth=20mm,
%dotsep=1000,
%tocrmarg=4.5cm,
%abstandVerzeichnis=-1mm
]

%
% LIstingverzeichnis
%
\seKonfigurationPrg[
%verzeichnisname=Listing-Verzeichnis,
labeltextLinks=,
%labeltextRechts=:,
labelbreite=1cm,
%labeleinzug=2cm,
%abstandEintrag=1,
%newpage=ja,
%pnumwidth=20mm,
%dotsep=1000,
%tocrmarg=4.5cm,
%abstandVerzeichnis=-10mm
]

% 2013-01-26
%
% Algorithmenverzeichnis
%
\seKonfigurationAlg[
%verzeichnisname=Algorithmen-Verzeichnis,
labeltextLinks=,
%labeltextRechts=:,
labelbreite=1cm,
%labeleinzug=2cm,
%abstandEintrag=1,
%newpage=ja,
%pnumwidth=20mm,
%dotsep=1000,
%tocrmarg=4.5cm,
%abstandVerzeichnis=-10mm
]




%
% Tabellenverzeichnis
%
\seKonfigurationTab[
%verzeichnisname=Liste der Tabellen,
labeltextLinks=,
%labeltextRechts=:,
labelbreite=1cm,
%labeleinzug=0.5cm,
%abstandEintrag=1,
%newpage=ja,
%pnumwidth=20mm,
%dotsep=1000,
%tocrmarg=4.5cm,
%abstandVerzeichnis=-10mm
]

%
% Abk\"urzungsverzeichnis
%
\seKonfigurationAbk[
%verzeichnisname=Liste der Abk\"urzungen,
%labelbreite=3cm,
%labeleinzug=0.5cm,
%abstandEintrag=1,
%newpage=ja,
%abstandVerzeichnis=-10mm
]

%
% Symbolverzeichnis
% 
\seKonfigurationSym[
%verzeichnisname=Liste der Symbole,
%labelbreite=4cm,
%labeleinzug=3.5cm,
%abstandEintrag=1,
%newpage=ja,
%abstandVerzeichnis=-10mm
]

%
% Glossar
%
\seKonfigurationGlo[
%verzeichnisname=Glossar,
%abstandEintrag=0,
]



% (eventuelle) Neudefinition f\"ur die Unter-/\"Uberschriften von Abbildungen, Tabellen und Listings
%
%
%\renewcommand{\seCaptionNameAbbildung}{Abb.}
%\renewcommand{\seCaptionNameTabelle}{Tab.}
%\renewcommand{\seCaptionNameProgramm}{Prg.}


% % (eventuelle) Neudefinition f\"ur Querverweise innerhalb des Textes
%
%
%
%\renewcommand{\seQuerverweisSeite}{Seite}
%\renewcommand{\seQuerverweisAbbildung}{Abb.}
%\renewcommand{\seQuerverweisTabelle}{Tab.}
%\renewcommand{\seQuerverweisProgramm}{Prg.}
%\renewcommand{\seQuerverweisGleichung}{Gl.}
%\renewcommand{\seQuerverweisAlgorithmus}{Alg.}
%
\renewcommand{\seQuerverweisChapter}{Kapitel}
% 2013-07-08
\renewcommand{\seQuerverweisAppendix}{Anhang}
\renewcommand{\seQuerverweisSection}{Kapitel}
\renewcommand{\seQuerverweisSubsection}{Kapitel}
\renewcommand{\seQuerverweisSubsubsection}{Kapitel}
\renewcommand{\seQuerverweisParagraph}{Kapitel}


%
% Kommandos f\"ur die Konfiguration von URL-Eintr\"agen im Literaturverzeichnis
%
\renewcommand*{\biburlprefix}{\jblangle{}URL: }
\renewcommand*{\biburlsuffix}{\jbrangle{}}
\renewcommand*{\bibbudcsep}{ -- }
\AddTo\bibsgerman{\renewcommand*{\urldatecomment}{Zugriff am }}


%
}{% englisch
% wa-konfiguration-englisch
%
% 2012-12-13
% 
% Diese Datei wird f\"ur die Sprachoption englisch verwendet, d. h.  
% \newcommand{\seWaSprache}{englisch}
%
%
% In dieser Datei k\"onnen Neudefinitionen vorgenommen werden f\"ur:
% -- Verzeichnisse
% -- Unter-/\"Uberschriften von Abbildungen, Tabellen und Listings
% -- Querverweise innerhalb des Textes

% 2013-07-08: Querverweis auf Anhang hinzugenommen
%
%
%


%
%  Konfiguration der verschiedenen Verzeichnisse
%
%  abstandEintrag: Wert wird mit \baselineskip multipliziert
%

%
%  Abbildungsverzeichnis
%
\seKonfigurationAbb[
verzeichnisname=List of Figures,
labeltextLinks=, % kein Text links;
%labeltextRechts=:,
labelbreite=1cm,
%labeleinzug=1cm,
%abstandEintrag=1,
%newpage=ja,
%pnumwidth=20mm,
%dotsep=1000,
%tocrmarg=4.5cm,
%abstandVerzeichnis=-1mm
]

%
% LIstingverzeichnis
%
\seKonfigurationPrg[
verzeichnisname=List of Program Listings,
labeltextLinks=,
%labeltextRechts=:,
labelbreite=1cm,
%labeleinzug=2cm,
%abstandEintrag=1,
%newpage=ja,
%%pnumwidth=20mm,
%dotsep=1000,
%tocrmarg=4.5cm,
%abstandVerzeichnis=-10mm
]


% 2013-01-26
%
% Algorithmenverzeichnis
%
\seKonfigurationAlg[
verzeichnisname=List of Algorithms,
labeltextLinks=,
%labeltextRechts=:,
labelbreite=1cm,
%labeleinzug=2cm,
%abstandEintrag=1,
%newpage=ja,
%pnumwidth=20mm,
%dotsep=1000,
%tocrmarg=4.5cm,
%abstandVerzeichnis=-10mm
]


%
% Tabellenverzeichnis
%
\seKonfigurationTab[
verzeichnisname=List of Tables,
labeltextLinks=,
%labeltextRechts=:,
labelbreite=1cm,
%labeleinzug=0.5cm,
%abstandEintrag=1,
%newpage=ja,
%pnumwidth=20mm,
%dotsep=1000,
%tocrmarg=4.5cm,
%abstandVerzeichnis=-10mm
]

%
% Abk\"urzungsverzeichnis
%
\seKonfigurationAbk[
verzeichnisname=List of Abbreviations,
%labelbreite=3cm,
%labeleinzug=0.5cm,
%abstandEintrag=1,
%newpage=ja,
%abstandVerzeichnis=-10mm
]

%
% Symbolverzeichnis
% 
\seKonfigurationSym[
verzeichnisname=List of Symbols,
%labelbreite=4cm,
%labeleinzug=3.5cm,
%abstandEintrag=1,
%newpage=ja,
%abstandVerzeichnis=-10mm
]

%
% Glossar
%
\seKonfigurationGlo[
verzeichnisname=Glossary,
%abstandEintrag=0,
]



% (eventuelle) Neudefinition f\"ur die Unter-/\"Uberschriften von Abbildungen, Tabellen und Listings
%
%
\renewcommand{\seCaptionNameAbbildung}{Figure}
\renewcommand{\seCaptionNameTabelle}{Table}
\renewcommand{\seCaptionNameProgramm}{Listing}
\renewcommand{\seCaptionNameAlgorithmus}{Algorithm}


% % (eventuelle) Neudefinition f\"ur Querverweise innerhalb des Textes
%
%
%
\renewcommand{\seQuerverweisSeite}{page}
\renewcommand{\seQuerverweisAbbildung}{figure}
\renewcommand{\seQuerverweisTabelle}{table}
\renewcommand{\seQuerverweisProgramm}{listing}
\renewcommand{\seQuerverweisGleichung}{equation}
\renewcommand{\seQuerverweisAlgorithmus}{algorithm}
%
\renewcommand{\seQuerverweisChapter}{chapter}
\renewcommand{\seQuerverweisAppendix}{appendix}
\renewcommand{\seQuerverweisSection}{chapter}
\renewcommand{\seQuerverweisSubsection}{chapter}
\renewcommand{\seQuerverweisSubsubsection}{chapter}
\renewcommand{\seQuerverweisParagraph}{chapter}


%
% Kommandos f\"ur die Konfiguration von URL-Eintr\"agen im Literaturverzeichnis
%
\renewcommand*{\biburlprefix}{\jblangle{}URL: }
\renewcommand*{\biburlsuffix}{\jbrangle{}}
\renewcommand*{\bibbudcsep}{ -- }
\AddTo\bibsenglish{\renewcommand*{\urldatecomment}{visited on }}


}

% Kommandos, die direkt nach \begin{document} ausgef\"uhrt werden m\"ussen
%
%
%
\AtBeginDocument{%
\renewcommand{\listfigurename}{\seAbbildungenVerzeichnisname}
\renewcommand{\listtablename}{\seTabellenVerzeichnisname}
\renewcommand{\figurename}{\seCaptionNameAbbildung}
\renewcommand{\tablename}{\seCaptionNameTabelle}
\labelformat{lstlisting}{\seQuerverweisProgramm{} #1}
\renewcommand{\thelstlisting}{\theprogramm}
\pagenumbering{roman}
}
                                                              
                                                                         

%
% Definition von Abk\"urzungen, Symbolen und eventuell Glossareintr\"agen
%
% 2012-03-22 Verwendung des optionalen Parameters f\"ur die Pluralform einer Abk\"urzung
%
% 2012-02-06 Umstellung auf die neuen Kommandos
%
%
%
%  J\"org Baumgart
%  Definition einiger Abk\"urzungen
%  


% Definition von Abk\"urzungen
%
% 1. Parameter: Schluessel (key) der Abkuerzung
% 2. Parameter: Abkuerzung
% 3. Parameter: Vollform
% 4. Parameter: Vollform im Plural (optional; falls nicht definiert, wird der Wert des dritten Parameters verwendet)
%
\seNewAcronymEntry{dm}{DM}{Diagonalmatrix}{Diagonalmatrizen}

\seNewAcronymEntry{dhbw}{DHBW}{Duale Hochschule Baden-W\"urttemberg}{}{}

\seNewAcronymEntry{usb}{USB}{Universal Serial Bus}{}

\seNewAcronymEntry{ctan}{CTAN}{Comprehensive \TeX{} Archive Network}{}


% 2012-03-24
% \"Uber den optionalen Parameter in eckigen Klammern wird die Pluralform f\"ur das erste 
% Auftreten der Abk\"urzung definiert

\seNewAcronymEntry[URLs]{url}{URL}{Uniform Resource Locator}%
{Uniform Resource Locators}


% Definition von Symbolen
%
% 1. Parameter: Schluessel (key) des Symbols
% 2. Parameter: Symbol
% 3. Parameter: Text, der die Sortierreihenfolge festlegt (optional; falls nicht definiert, wird der Wert des zweiten 
%                        Parameters verwendet)
% 4. Parameter: Beschreibung des Symbols
%

\seNewSymbolEntry{ND}{ND}{a}{Nutzungsdauer einer Maschine}

\seNewSymbolEntry{pi}{$\pi$}{b}{Die Kreiszahl}




% Definition von Glossareintraegen
%
% 1. Parameter: Schluessel (key) des Glossareintrags
% 2. Parameter: Begriff, der im Glossar definiert wird
% 3. Parameter: Pluralform des Begriffes (optional; falls nicht definiert, wird der Wert des zweiten Parameters verwendet)
%                        Achtung: Pluralform gilt nur fuer das erste Auftreten des Begriffes im Text
% 4. Parameter: Beschreibung des Glossareintrags
%
%
%

\seNewGlossaryEntry{glos:AD}{Active Directory}{Active Directories}
{Active Directory ist in einem Windows 2000/Windows
Server 2003-Netzwerk der Verzeichnisdienst, der die zentrale
Organisation und Verwaltung aller Netzwerkressourcen erlaubt. Es
erm\"oglicht den Benutzern \"uber eine einzige zentrale Anmeldung den
Zugriff auf alle Ressourcen und den Administratoren die zentral
organisierte Verwaltung, transparent von der Netzwerktopologie und
den eingesetzten Netzwerkprotokollen. Das daf\"ur ben\"otigte
Betriebssystem ist entweder Windows 2000 Server oder
Windows Server 2003, welches auf dem zentralen
Dom\"anencontroller installiert wird. Dieser h\"alt alle Daten des
Active Directory vor, wie z.\,B. Benutzernamen und
Kennw\"orter.\protect\footnote{Bedauerlicherweise wei{\ss} der Autor dieses Dokumentes nicht mehr, woher diese Information stammt -- das 
geht in einer richtigen wissenschaftlichen Arbeit nat\"urlich \"uberhaupt nicht!!!}}
%\protect\seFootcite{Vgl.}{S. 200}{Dud09}}


\seNewGlossaryEntry{glos:bs}{Betriebssystem}{Betriebssysteme}{Die Begriffsdefinition sollten Sie eigentlich kennen!}


% Definition von Glossareintraegen, die gleichzeitig im Abk�rzungsverzeichnis auftreten
%
% 1. Parameter: Schluessel (key) des Glossareintrags
% 2. Parameter: Abk\"urzung
% 3. Parameter: Vollform
% 4. Parameter: Vollform im Plural (optional; falls nicht definiert, wird der Wert des dritten Parameters verwendet)
% 5. Parameter: Beschreibung des Glossareintrags

\seNewAcronymGlossaryEntry{glos:ma}{MA}{Mobile Applikation}{Mobile Applikationen}
{Eine Applikation, die auf einem mobilen Endger\"at ausgef\"uhrt wird.}

% 2012-03-24
% \"Uber den optionalen Parameter in eckigen Klammern wird die Pluralform f\"ur die Abk\"urzung definiert

\seNewAcronymGlossaryEntry[TAen]{glos:ta}{TA}{Transaktion}%
{Transaktionen}%
{Was eine Transaktion ist, sollten Sie ebenfalls bereits wissen!}





% Alternative Definition von Abk\"urzungen; diese sollten nicht verwendet werden!!!
%
%\newacronym{dhbw}{DHBW}{Duale Hochschule Baden-W\"urttemberg}
%\newacronym{usb}{USB}{Universal Serial Bus}


% Alternative Definition von Symbolen
%
% Achtung: ohne sort wird nach Name sortiert
%\newglossaryentry{pi}{
%name=$\pi$,
%description={Die Kreiszahl},
%type=symbolslist,
%sort=b
%}
%
%\newglossaryentry{ND}{
%name=$\mbox{\textsl{ND}}$,
%description={Nutzungsdauer einer Maschine},
%type=symbolslist,%
%sort=a
%}



% Alternative Definition von Glossareintr\"agen
%
%\newglossaryentry{glos:AD}{
%first=Active Directory\textsuperscript{GL},
%name=Active Directory,
%description={Active Directory ist in einem Windows 2000/Windows
%Server 2003-Netzwerk der Verzeichnisdienst, der die zentrale
%Organisation und Verwaltung aller Netzwerkressourcen erlaubt. Es
%erm\"oglicht den Benutzern \"uber eine einzige zentrale Anmeldung den
%Zugriff auf alle Ressourcen und den Administratoren die zentral
%organisierte Verwaltung, transparent von der Netzwerktopologie und
%den eingesetzten Netzwerkprotokollen. Das daf\"ur ben\"otigte
%Betriebssystem ist entweder Windows 2000 Server oder
%Windows Server 2003, welches auf dem zentralen
%Dom\"anencontroller installiert wird. Dieser h\"alt alle Daten des
%Active Directory vor, wie z.\,B. Benutzernamen und
%Kennw\"orter.\protect\seFootcite{Vgl.}{S. 200}{Dud09}}
%}














 

% Eigene Kommandos
\usepackage[]{setspace}
\usepackage[]{textcomp}
\usepackage[output-decimal-marker={,}]{siunitx}
\usepackage{longtable}
\usepackage{pgfplots}
\usepackage{colortbl}
\usepackage{kantlipsum} % just for the example

%\usepackage[left=25mm,right=35mm,top=30mm,bottom=40mm]{geometry}

% Definition eines Kommandos für mathematische Formeln
\newcommand{\formel}[1]{\begin{center}$#1$\end{center}}
\newcommand{\formelleft}[1]{$#1$}

% Definition eines Kommandos für mathematische Variablen
\newcommand{\variable}[1]{$#1$}

% Definition für Zitate mit Übersetzung
\newcommand{\citeEnglish}[3]{\seFootcite{siehe}{Seite #1, Übersetzung: #2}{#3}}

\newcommand{\anmerkung}[1]{\footnote{Anmerkung: #1}}

\newcommand{\figref}[1]{\footnote{siehe \vref{#1}.}}

% Fußnote passend einrücken 
\setlength{\footnotemargin}{1.3em}

% kein Umbruch bei Fußnote
% siehe http://texwelt.de/wissen/fragen/2421/wie-kann-ich-verhindern-dass-funoten-auf-zwei-seiten-verteilt-ausgegeben-werden
\interfootnotelinepenalty=10000

%\KOMAoption{headings}{small}
\seNoChapterSkip[11.5mm]




%\seIstSeminararbeit{}

\newcommand{\version}{0.98}

% 
% Diese Redefinition ist nur f\"ur den Anhang B der  
% Vorlage (Hinweise zur Installation und \"Ubersetzung)
% notwendig; f\"ur Ihre Bachelorarbeit spielt sie keine Rolle
%
\renewcommand{\seVorlage}{\jobname}


\begin{document}

\begin{titlepage}

\centering

\vspace*{\stretch{2}}

\Large to be done Abschlussbericht Projekt PulseShift

\vspace{\stretch{1}}

\normalsize dauer

\vspace{\stretch{0.5}}

\begin{tabular}{ll}
Submission date & xx.yy.zzzz \\[2ex]
Student's name  & Name \\[2ex]
From            & Place
\end{tabular}

\vspace{\stretch{3}}

\begin{tabular*}{\textwidth}{@{}l@{\extracolsep{\fill}}l@{}}
My name & My professor's name \\[2ex]
        & My supervisor's name
\end{tabular*}

\vspace{\stretch{2}}
\end{titlepage}


%% Erzeugung des Titelblatts
%%
%%
%%
%\seTitelblattWissenschaftlicheArbeit[
%%hilfslinien=ja,
%%dhbwlogoSkalierung=0.5,
%%dhbwlogoDeltaX=2.4,
%%dhbwlogoDeltaY=-10,
%firmenlogo=firmenlogo,
%firmenlogoSkalierung=0.2,
%firmenlogoDeltaX=0,
%firmenlogoDeltaY=0,
%studiengang=\seWirtschaftsinformatik,
%studienrichtung=\seSoftwareEngineering,
%thema=Projekt PulseShift,
%verfasser=Sebastian Schütz Florian Finkel,
%matrikelnummer=9999999,
%kurs=WWI\,15\,SE\,A,
%firma=Ausbildungsfirma,
%% Da im Text ein Komma enthalten ist, muss der Text eingeklammert werden
%%abteilung={Wirtschaftsinformatik, Sales \& Consulting},
%abteilung={Wirtschaftsinformatik, Software Engineering},
%%studiengangsleiterin=,
%studiengangsleiter=Prof. Dr.-Ing. J\"org Baumgart,
%%studiengangsleiter=Prof. Dr. Thomas Holey,
%wissenschaftlicheBetreuerinName=Dr. Melanie Mustermann,
%wissenschaftlicheBetreuerinEmail=melanie.mustermann@musterfirma.de,
%wissenschaftlicheBetreuerinTelefon=0621/999999,
%%wissenschaftlicherBetreuerName=Prof. Dr.-Ing. J\"org Baumgart,
%%wissenschaftlicherBetreuerEmail=joerg.baumgart@dhbw-mannheim.de,
%%wissenschaftlicherBetreuerTelefon=0621/4105\,1216,
%%firmenbetreuerinName=Dipl.-Ing. Ariane Meistermann,
%%firmenbetreuerinEmail=a.meistermann@andere-musterfirma.de,
%%firmenbetreuerinTelefon=06151/88888,
%%firmenbetreuerName=,
%%firmenbetreuerEmail=,
%%firmenbetreuerTelefon=,
%bearbeitungszeitraumVon=28. November 2016,
%bearbeitungszeitraumBis=19. Februar 2017,
%%
%% Datum in englischer Schreibweise
%%bearbeitungszeitraumVon=28 November  2016,
%%bearbeitungszeitraumBis=19 February 2017,
%sperrvermerk=nein
%]
%



% Erzeugung des Titelblatts
%
%
%
\seTitelblattWissenschaftlicheArbeit[
%hilfslinien=ja,
%dhbwlogoSkalierung=0.5,
%dhbwlogoDeltaX=2.4,
%dhbwlogoDeltaY=-10,
firmenlogo=firmenlogo,
firmenlogoSkalierung=0.2,
firmenlogoDeltaX=0,
firmenlogoDeltaY=0,
studiengang=\seWirtschaftsinformatik,
studienrichtung=\seSoftwareEngineering,
thema=Projekt PulseShift,
verfasser=Sebastian Schütz Florian Finkel,
matrikelnummer=9999999,
kurs=WWI\,15\,SE\,A,
firma=Ausbildungsfirma,
% Da im Text ein Komma enthalten ist, muss der Text eingeklammert werden
%abteilung={Wirtschaftsinformatik, Sales \& Consulting},
abteilung={Wirtschaftsinformatik, Software Engineering},
%studiengangsleiterin=,
studiengangsleiter=Prof. Dr.-Ing. J\"org Baumgart,
%studiengangsleiter=Prof. Dr. Thomas Holey,
wissenschaftlicheBetreuerinName=Dr. Melanie Mustermann,
wissenschaftlicheBetreuerinEmail=melanie.mustermann@musterfirma.de,
wissenschaftlicheBetreuerinTelefon=0621/999999,
%wissenschaftlicherBetreuerName=Prof. Dr.-Ing. J\"org Baumgart,
%wissenschaftlicherBetreuerEmail=joerg.baumgart@dhbw-mannheim.de,
%wissenschaftlicherBetreuerTelefon=0621/4105\,1216,
%firmenbetreuerinName=Dipl.-Ing. Ariane Meistermann,
%firmenbetreuerinEmail=a.meistermann@andere-musterfirma.de,
%firmenbetreuerinTelefon=06151/88888,
%firmenbetreuerName=,
%firmenbetreuerEmail=,
%firmenbetreuerTelefon=,
bearbeitungszeitraumVon=28. November 2016,
bearbeitungszeitraumBis=19. Februar 2017,
%
% Datum in englischer Schreibweise
%bearbeitungszeitraumVon=28 November  2016,
%bearbeitungszeitraumBis=19 February 2017,
sperrvermerk=nein
]



% 2012-02-06 Inhaltsverzeichnis muss vor den weiteren Verzeichnisses kommen
%
%
% Ausgabe des Inhaltsverzeichnisses
%
%
\seInhaltsverzeichnis[%
einrueckung=ja,
gliederungsebenen=4
]




% Ausgabe der verschiedenen Verzeichnisse
% abk: Abk\"urzungsverzeichnis
% sym: Symbolverzeichnis
% abb: Abbildungsverzeichnis
% tab: Tabellenverzeichnis
% prg: Listingverzeichnis
% alg: Algorithmenverzeichnis
%
%
% Achtung: Abk\"urzungs- und Symbolverzeichnis werden nur ausgegeben, wenn mindest ein Symbol bzw. 
%                mindestens eine Abk\"urzung in der Arbeit verwendet wurden
%
%
% gliederungsebene:
% -- section: die Verzeichnisse werden einem Kapitel "Verzeichnisse" untergliedert
% -- chapter: die Verzeichnisse sind jeweils eigene Kapitel
% imInhaltsverzeichnis: ja/nein -- Sollen die Verzeichnisse im Inhaltsverzeichnis enthalten sein?
\seVerzeichnisse[gliederungsebene=section,imInhaltsverzeichnis=ja]{abk}{sym}{abb}{tab}{prg}{alg}



%-----------------------------
%- - - - - - - - - - - - - - -
%-----------------------------
%Hier für 1,5 Zeilenabstand
\onehalfspacing

%-----------------------------
%- - - - - - - - - - - - - - -
%-----------------------------

\pagenumbering{arabic}
\chapter{Einführung und Projektrahmen}
\chapter{Einführung und Projektrahmen}
\chapter{Einführung und Projektrahmen}
\input{chapters/introduction/introduction}
\input{chapters/introduction/goals}
\input{chapters/introduction/benefit}

\section{Ziel des Projekts}
\label{sec:introduction:goals}
Das Ziel des Projekts ist die Erarbeitung eines Lösungsportfolios, dass die Teilnahme von Offline-Mitarbeitern an den Umfragen von PulseShift ermöglicht. Dieses Ziel besteht aus zwei Teilzielen:

\begin{enumerate}
\item Es sollen mögliche Kanäle konzeptionell erarbeitet werden, um Offline-Mitarbeitern die Teilnahme an der Umfrage zu ermöglichen.
\item Die Kanäle die das größte Potential aufweisen sollen als \gls{poc} umgesetzt werden.
\end{enumerate}


\section{Erwarteter wirtschaftlicher Nutzen}

Die in diesem Projekt konzeptionell erarbeiteten Kanäle sollen PulseShift einen Überblick geben, wie eine Umfrage an Offline-Mitarbeiter verteilt werden kann. Daraus soll abgeleitet werden können, welche Umfragekanäle im Hinblick auf Kosten und Nutzen geeignet sind, um in das Lösungsportfolio von PulseShift aufgenommen zu werden. So kann teuren Investitionen in nicht geeignete Umfragekanäle vorgebeugt werden. Gleichzeitig kann gezielt in die Umsetzung von Kanälen mit hohem Potential investiert werden. 

Die entwickelten \gls{poc}s sollen zum einen für PulseShift als Demonstration


 aufzeigen, wie die Umsetzung der jeweiligen Kanäle aussehen kann. So 

 Zum Anderen



Die in dem Projekt entwickelten  sollen die Umfragemöglichkeiten für PulseShift bei Mitarbeitern ohne Email-Adresse gewährleisten.

 
 Weiterhin sollen die PoCs mit einer möglichst hohen Akzeptanz in der zu untersuchenden Zielgruppe anwendbar sein. 
 
 Der wirtschaftliche Mehrwert liegt somit einerseits in der kostengünstigen Umsetzung der Umfrage und der einhergehenden Kosteneinsparungen, andererseits auch in den möglichst aussagekräftigen Ergebnissen, die durch eine Steigerung der Anzahl an befragten Mitarbeitern erreicht werden. 
 
 Durch diese werden dem Unternehmen Handlungsmöglichkeiten zur besseren Durchführung von Digitalisierungsmaßnahmen aufgezeigt und somit weitere Optionen zur Kosteneinsparung dargelegt.

\section{Ziel des Projekts}
\label{sec:introduction:goals}
Das Ziel des Projekts ist die Erarbeitung eines Lösungsportfolios, dass die Teilnahme von Offline-Mitarbeitern an den Umfragen von PulseShift ermöglicht. Dieses Ziel besteht aus zwei Teilzielen:

\begin{enumerate}
\item Es sollen mögliche Kanäle konzeptionell erarbeitet werden, um Offline-Mitarbeitern die Teilnahme an der Umfrage zu ermöglichen.
\item Die Kanäle die das größte Potential aufweisen sollen als \gls{poc} umgesetzt werden.
\end{enumerate}


\section{Erwarteter wirtschaftlicher Nutzen}

Die in diesem Projekt konzeptionell erarbeiteten Kanäle sollen PulseShift einen Überblick geben, wie eine Umfrage an Offline-Mitarbeiter verteilt werden kann. Daraus soll abgeleitet werden können, welche Umfragekanäle im Hinblick auf Kosten und Nutzen geeignet sind, um in das Lösungsportfolio von PulseShift aufgenommen zu werden. So kann teuren Investitionen in nicht geeignete Umfragekanäle vorgebeugt werden. Gleichzeitig kann gezielt in die Umsetzung von Kanälen mit hohem Potential investiert werden. 

Die entwickelten \gls{poc}s sollen zum einen für PulseShift als Demonstration


 aufzeigen, wie die Umsetzung der jeweiligen Kanäle aussehen kann. So 

 Zum Anderen



Die in dem Projekt entwickelten  sollen die Umfragemöglichkeiten für PulseShift bei Mitarbeitern ohne Email-Adresse gewährleisten.

 
 Weiterhin sollen die PoCs mit einer möglichst hohen Akzeptanz in der zu untersuchenden Zielgruppe anwendbar sein. 
 
 Der wirtschaftliche Mehrwert liegt somit einerseits in der kostengünstigen Umsetzung der Umfrage und der einhergehenden Kosteneinsparungen, andererseits auch in den möglichst aussagekräftigen Ergebnissen, die durch eine Steigerung der Anzahl an befragten Mitarbeitern erreicht werden. 
 
 Durch diese werden dem Unternehmen Handlungsmöglichkeiten zur besseren Durchführung von Digitalisierungsmaßnahmen aufgezeigt und somit weitere Optionen zur Kosteneinsparung dargelegt.


% Mit markright kann eine verk\"urzte Version der \"Uberschrift f\"ur den Seitenkopf generiert werden
%
%
%\markright{Formaler Aufbau}
%
%\input{chapters/drought} 
%
%\input{chapters/surface_indices}
%
%\input{chapters/data}
%....


\chapter{Abschluss}

\section{Zusammenfassung}

- Was wurde alles gemacht

\section{Zielerreichung}

- Erreichung der Ziele auswerten auf basis von 1.2 und Randbedingungen berücksichtigen

\section{Ausblick}

Was kann pulseshift jetzt mit dem ganzen kladderradatsch anfangen?


% Anhang der Arbeit
% 
%
\seAppendix{}


%\chapter{Einige wichtige \LaTeX{}-Kommandos}

%  Testdatei f\"ur die Erzeugung von Literaturreferenzen, die den Regeln von Rene Theisen 
%  (Wissenschaftliches Arbeiten, 2009) folgen
%
%
%
\section{Kommandos f\"ur die Erzeugung von Literaturverweisen}

\subsection{Kurzzitierweise mit der Angabe eines Kurztitels}

Das Kommando \verb+\seCite{par1}{par2}{par3}+ erzeugt einen Literaturverweis im Text. 

\begin{seToplist}{\texttt{par1}:}
\item[\texttt{par1}:] Der erste Parameter  definiert einen optionalen Text, der vor dem eigentlichen Literaturverweis ausgegeben 
                               wird, typischerweise Vgl. oder vgl.
\item[\texttt{par2}:] Der zweite Parameter  wird verwendet, um (z.\,B.) zus\"atzliche Seitenangaben f\"ur den Literaturverweis 
                              vorzunehmen.
\item[\texttt{par2}:] Der dritte Parameter ist der entsprechende Schl\"ussel in der .bib-Datei, in der die Literaturquellen 
                              beschrieben sind (vgl. \texttt{wa.bib}).                                                                                       
\end{seToplist}

Als Beispiel f\"ur die Verwendung des \verb+\seCite+-Befehls dient folgendes Zitat: \glqq{}Die \textbf{Funktion} eines 
Anhangs einer wissenschaftlichen Arbeit wird sehr h\"aufig \textbf{missdeutet}, der Anhang selbst nicht selten \textbf{mi{\ss}braucht}.\grqq{} 
(\seCite{vgl.}{S. 170}{The:WA}).

Bei der von Theisen vorgeschlagenen Zitierweise erfolgt die Angabe der Literaturverweise in der Regel innerhalb einer Fu{\ss}note. 
Hierf\"ur kann das Kommando \verb+\seFootcite+ verwendet werden, das dieselben Parameter wie \verb+\seCite+ besitzt. 

Als Beispiel f\"ur ein indirektes Zitat l\"asst sich die Aussage von Theisen anf\"uhren, dass Hauptinhalte eines (berechtigten) Anhangs erg\"anzende 
Materialien und Dokumente sind, die weitere themenbezogene Informationen liefern k\"onnen.\seFootcite{Vgl.}{S. 171}{The:WA}

Weder das \verb+\seFootcite+- noch das \verb+\footnote+-Kommande k\"onnen bei Gleitobjekten (Verwendung der \verb+figure+-, \verb+table+- oder 
\verb+programm+-Umgebung) verwendet werden. Ein kleiner Workaround, um \LaTeX{} doch dazu zu bringen, Fu{\ss}noten bei Gleitobjekten 
zu akzeptieren, ist in \vref{gleitobjekte} zu finden.

\input{\seWaPathText/se-zitieren-harvard}

\subsection{Verwendung von URLs}

URLs k\"onnen in der \texttt{bib}-Datei mit \texttt{@WWW} definiert werden. Das Feld \texttt{author} ist zwar optional, sollte aber immer angegeben 
werden, da andernfalls im Literaturverzeichnis der Kurztitel nicht ausgegeben wird. Wenn kein Autor bekannt ist, wird die Abk\"urzung o.\ V. verwendet.
Beim Eintrag in der \texttt{bib}-Datei ist zu beachten, dass diese Abk\"urzung zus\"atzlich eingeklammert werden muss, d.\,h. sie ist in der 
Form \texttt{author = \{\{o.~V.\}\}} anzugeben. 

Das Layout der URL-Angabe im Literaturverzeichnis kann \"uber vier Parameter beeinflusst werden.  
In der Datei \texttt{wa-konfiguration-deutsch.tex} k\"onnen Redefinitionen vorgenommen werden.

\begin{seList}
\item \verb+\biburlprefix+ \newline Text, der vor dem eigentlichen URL-Eintrag ausgegeben wird \newline Standardwert: \glqq{}\jblangle{}URL: \grqq{}
\item \verb+\biburlsuffix+ \newline Text, der hinter dem eigentlichen URL-Eintrag ausgegeben wird \newline Standardwert: \glqq{}\jbrangle{}\grqq{}
\item \verb+\bibbudcsep+ \newline Text zwischen dem eigentlichen URL-Eintrag und der Datumsangabe f\"ur den letzten Zugriff auf die URL
                                         \newline Standardwert: \glqq{} -- \grqq{}
\item \verb+\urldatecomment+ \newline Text, der vor der Datumsangabe f\"ur den letzten Zugriff ausgegeben wird
                                          \newline Standardwert: \glqq{}Zugriff am\grqq{}
\end{seList}

Und hier kommen noch zwei Beispiele f\"ur die Angabe von Literaturreferenzen, deren Quelle eine URL ist:
Das Paket \texttt{jurabib.sty} wurde von Jens Berger entwickelt.\seFootcite{Vgl.}{}{Ber:Hoj} 
\glqq{}Google will seine Suche auch in Deutschland um eine Datenbank mit abgesicherten Fakten, Biografien und Bildern erweitern, 
den Knowledge Graph.\grqq{}\seFootcite{}{}{GKN}


\newcommand{\dateiAbk}{\texttt{wa-abkuerzungen.tex}}
%
% Ein kleiner Text, um Abk\"urzungen, Symbole und Glossareintr\"age zu testen
%
%
\section{Kommandos f\"ur die Erzeugung von Abk\"urzungen, Symbolen und Glos\-sar\-eint\-r\"a\-gen}

\subsection{Definition von Abk\"urzungen, Symbolen und Glossareintr\"agen}

Um eine einheitliche Darstellung von Abk\"urzungen, Symbolen und Glossareintr\"agen zu erreichen, 
werden vier neue Kommandos zur Verf\"ugung gestellt:

\begin{seList}
\item 
\verb+\seNewAcronymEntry+\newline
Definition einer neuen Abk\"urzung.
\item 
\verb+\seNewSymbolEntry+\newline
Definition eines neuen Symbols.
\item
\verb+\seNewGlossaryEntry+\newline
Definition eines neuen Eintrags im Glossar.
\item
\verb+\seNewAcronymGlossaryEntry+\newline
Definition eines neuen Eintrags im Glossar, wobei zus\"atzlich eine 
Abk\"urzung definiert wird, die dann auch in das Abk\"urzungsverzeichnis aufgenommen wird.
\end{seList}

Der Datei \dateiAbk{} lassen sich die zugeh\"origen \textbf{Pa\-ra\-me\-ter\-be\-schrei\-bun\-gen}  
entnehmen.
In dieser Datei sind auch Beispiele enthalten, wie Abk\"urzungen, Symbole und Glossareintr\"age mit den 
Standardkommandos definiert werden k\"onnen, was jedoch nicht empfohlen wird!

\subsection{Verwendung von Abk\"urzungen, Symbolen und Glossareintr\"agen im Text}

Innerhalb des Textes wird f\"ur Abk\"urzungen, Symbole und Glossareintr\"age das Kommando \verb+\gls{par1}+ 
verwendet.
\texttt{par1} stellt einen Schl\"ussel dar, der die entsprechende Definition identifiziert (vgl. den Inhalt der Datei
\dateiAbk{}). 

Mit dem Kommando \verb+\glspl+ ist es m\"oglich, beim
Auftreten eines Begriffes,  f\"ur den ein Glossareintrag existiert, bzw.\ beim (ersten) Auftreten einer 
Abk\"urzung f\"ur die Vollform die  \textbf{Pluralform} auszugeben.%
\footnote{Genauer gesagt wird derjenige Wert ausgegeben, der in den 
Kommandos \texttt{\textbackslash{}seNewAcronymEntry}, \texttt{\textbackslash{}seNewGlossaryEntry} bzw. \texttt{\textbackslash{}seNewAcronymGlossaryEntry}
als Pluralform definiert wurde. Die Pluralform k\"onnte man alternativ verwenden, um beispielsweise eine 
Genitivform zu definieren.}

Bei den Kommandos 
\begin{seList}
\item\verb+\seNewAcronymEntry+ und 
\item\verb+\seNewAcronymGlossaryEntry+
\end{seList}
kann durch die Verwendung des optionalen Parameters 
zu\-s\"atz\-lich eine Pluralform f\"ur die Abk\"urzung definiert werden (vgl. \dateiAbk{}).

\subsection{Anwendungsbeispiele}

\subsubsection{Abk\"urzungen}

Die dreimalige Anwendung von \verb+\gls{usb}+ liefert:

\begin{seList}
\item \gls{usb}
\item \gls{usb}
\item \gls{usb}
\end{seList}

Die Anwendung von \verb+\glspl{dm}+ \verb+\glspl{dm}+ \verb+\gls{dm}+ liefert:

\begin{seList}
\item \glspl{dm}
\item \gls{dm}
\item \gls{dm}
\end{seList}

Und auch die \gls{dhbw} soll noch erw\"ahnt werden, um das Abk\"urzungsverzeichnis ein wenig zu f\"ullen.

\subsubsection{Symbole}

Bei einem Symbol wird -- im Gegensatz zu Abk\"urzungen -- beim ersten Auftreten im Text nicht die 
zugeh\"orige Definition ausgegeben. Diese ist aber im Symbolverzeichnis zu finden.

Die zweimalige Anwendung von \verb+\gls{pi}+ liefert:

\begin{seList}
\item \gls{pi}
\item \gls{pi}
\end{seList}

Und jetzt kommt noch ein zweites Symbol f\"ur das Symbolverzeichnis: \gls{ND}

\subsubsection{Glossareintr\"age}

Bei einem Glossareintrag wird beim ersten Auftreten des Begriffes im Text dieser mit \textsuperscript{GL} markiert.
Im Glossar sind die Seitenzahlen angegeben, auf denen der Begriff verwendet wurde. 

Die dreimalige Anwendung von \verb+\gls{glos:AD}+ liefert:

\begin{seList}
\item \gls{glos:AD}
\item \gls{glos:AD}
\item \gls{glos:AD}
\end{seList}

Und hier kommt noch ein Beispiel f\"ur einen Glossareintrag, f\"ur den beim ersten und dritten Auftreten die Pluralform verwendet 
wird:  \verb+\glspl{glos:bs}+  \verb+\gls{glos:bs}+  \verb+\glspl{glos:bs}+ 

\begin{seList}
\item \glspl{glos:bs}
\item \gls{glos:bs}
\item \glspl{glos:bs}
\end{seList}

\subsubsection{Glossareintrag mit einem zus\"atzlichen Eintrag im Ab\-k\"ur\-zungs\-ver\-zeich\-nis}

Nach der ersten Anwendung des Begriffes, f\"ur den ein Glossareintrag erzeugt wurde, wird in der Folge 
jeweils nur noch die Abk\"urzung benutzt. 

Die Kommandoausf\"uhrungen \verb+\glspl{glos:ma}+ \verb+\gls{glos:ma}+  \verb+\glspl{glos:ma}+ haben als 
Ergebnis:

\begin{seList}
\item \glspl{glos:ma}
\item \gls{glos:ma}
\item \glspl{glos:ma}
\end{seList}

\newpage
Und jetzt wird auf einer neuen Seite nochmals \verb+\gls{glos:ma}+ verwendet, um im Glossar die neu hinzugekommene 
Seitennummer zu demonstrieren: \gls{glos:ma}

\subsubsection{Pluralform von Abk\"urzungen}

\textbf{\textsf{Definition einer Abk\"urzung}}

Der Eintrag wurde wie folgt definiert (vgl. \dateiAbk{}):

\vspace{-\baselineskip}
\begin{verbatim}
   \seNewAcronymEntry[URLs]{url}{URL}{Uniform Resource Locator}%
   {Uniform Resource Locators}
\end{verbatim}
\vspace{-\baselineskip}

Die Kommandoausf\"uhrungen \verb+\glspl{url}+ \verb+\gls{url}+  \verb+\glspl{url}+ haben als 
Ergebnis:

\begin{seList}
\item \glspl{url}
\item \gls{url}
\item \glspl{url}
\end{seList}

\seVsd
\textbf{\textsf{Definition eines Glossareintrags mit zus\"atzlicher Abk\"urzung}}

Der Eintrag wurde wie folgt definiert (vgl. \dateiAbk{}):

\vspace{-\baselineskip}
\begin{verbatim}
   \seNewAcronymGlossaryEntry[TAen]{glos:ta}{TA}{Transaktion}%
   {Transaktionen}%
   {Was eine Transaktion ist, sollten Sie ebenfalls bereits wissen!}
\end{verbatim}
\vspace{-\baselineskip}


Die Kommandoausf\"uhrungen \verb+\glspl{glos:ta}+ \verb+\gls{glos:ta}+  \verb+\glspl{glos:ta}+ haben als 
Ergebnis:

\begin{seList}
\item \glspl{glos:ta}
\item \gls{glos:ta}
\item \glspl{glos:ta}
\end{seList}

%2013-07-08
\subsubsection{Literaturverweise in Glossareinträgen}

Auch bei Glossareinträgen müssen natürlich Literaturverweise angegeben werden. 
Wird eine Literaturquelle erstmalig in einem Glossareintrag verwendet, dann tritt das Problem auf, 
dass sie von BibTeX nicht gefunden wird. Ein \textsl{Workaround} besteht darin, für die entsprechenden 
Literaturverweise \verb+\nocite{key}+-Kommandos anzugeben. \verb+key+ ist hierbei der zugehörige Schlüssel 
des Eintrags in der .bib-Datei.\footnote{Achtung: Ein \texttt{\textbackslash{}nocite}-Kommando sollte nur in absoluten Ausnahmefällen 
eingesetzt werden, da hiermit Einträge im Literaturverzeichnis erzeugt werden können, für die (möglicherweise) kein 
Literaturverweis innerhalb der Arbeit existiert.}

\newpage

 

\section{Fu{\ss}noten}

\subsection{Verwendung dreistelliger Fu{\ss}noten}

Bei dreistelligen Fu{\ss}noten tritt das Problem auf, dass der Abstand zwischen Fu{\ss}notennummer und folgendem Text nicht mehr ausreicht.
Der Abstand kann wie folgt vergr\"o{\ss}ert werden:

\begin{seList}
\item
In der Style-Datei \texttt{se-jb-footmisc.sty} wird \"uber \newline \texttt{\textbackslash{}setlength\textbackslash{}footnotemargin\{0.3cm\}} genau dieser Abstand definiert.
\item
\"Andert man den Wert z.\,B.\ auf 0.5cm, dann sollte es auch f\"ur dreistellige Fu{\ss}noten ausreichen.
\end{seList}

\subsection{Fu{\ss}noten in Abbildungen, Tabellen und Programmlistings}

\LaTeX{} erlaubt generell nicht die Verwendung des Kommandos \verb+\footnote+ in \textsl{Gleitobjekten} (\textsl{Floats}). 
Zu den Gleitobjekten geh\"oren \textsl{Abbildungen}, \textsl{Tabellen} und auch \textsl{Programmlistings}. In \vref{fussnote} findet ein 
kleiner \textsl{Workaround} Anwendung, wie man doch Fu{\ss}noten in Gleitobjekten angeben kann. 

\begin{seList}
\item Mit \verb+\footnotemark+ wird in dem \verb+\caption+-Kommando die \textsl{Fu{\ss}notennummer} erzeugt.
\item Mit dem Kommando \verb+\footnotetext+ wird au{\ss}erhalb der Umgebung, die das Gleitobjekt definiert (z.\,B. die 
\verb+figure+-Umgebung), der Text der Fu{\ss}note festgelegt. Hierbei ist zu beachten, dass ein Gleitobjekt auf die n\"achste Seite 
verschoben werden kann. In einem derartigen Fall sollte der Fu{\ss}notentext an einer Stelle im \LaTeX-Quelltext positioniert werden, die ebenfalls zu dieser Seite geh\"ort.\footnote{Standardm\"a{\ss}ig wird man \texttt{\textbackslash{}footnotetext} direkt hinter dem Gleitobjekt definieren, um sicherzustellen,
dass der Fu{\ss}notentext auch der richtigen Fu{\ss}notennummer zugeordnet wird.}
\end{seList}


\section{Abbildungen, Tabellen und Programmlistings\label{gleitobjekte}}

Ein Rechteck besitzt die in \vref{abb1} dargestellte Struktur.

\begin{figure}[htbp]
\centering
\setlength{\unitlength}{1mm}
\begin{picture}(100,30)
\put(0,0){\framebox(100,30){Ich bin kein Quadrat!}}
\end{picture}
\caption[Die Darstellung eines Rechtecks]{Die Darstellung eines Rechtecks\label{abb1}\footnotemark}
\end{figure}
%\footnotetext{\seCite{Vgl.}{S. 400}{The:WA}. Achtung: Dieser Literaturverweis ist  rein fiktiver Natur, 
%die Seite 400 existiert in \seCite{}{}{The:WA} nicht!}\label{fussnote}

Der optionale Parameter im folgenden \verb+\caption+-Kommando
\footnotetext{\seCite{Vgl.}{S. 400}{The:WA}. Achtung: Dieser Literaturverweis ist  rein fiktiver Natur, 
die Seite 400 existiert in \seCite{}{}{The:WA} nicht!}\label{fussnote}


\vspace*{-\baselineskip}
\begin{verbatim}
\caption[Die Darstellung eines Rechtecks]%
{Die Darstellung eines Rechtecks\label{abb1}\footnotemark}
\end{verbatim}
\vspace*{-\baselineskip}

definiert den Eintrag f\"ur das Abbildungsverzeichnis. Dort sollte die Fu{\ss}notennummer nicht auftauchen.
Nutzt man den optionalen Parameter nicht, ist es notwendig,  vor \verb+\footnotemark+ noch ein \verb+\protect+ 
einzuf\"ugen, da \LaTeX{} andernfalls die \"Ubersetzung mit einer Fehlermeldung abbricht. 

Eine Notentabelle kann wie in \vref{noten} dargestellt aussehen.

\begin{table}[htbp]%
\centering%
\begin{tabular}{| c | c |}
\hline
Matrikelnummer & Note \\
\hline
\hline
1234567 & 2,7 \\
\hline
2323456 & 3,5 \\
\hline
9865783 & 1,0 \\
\hline
\end{tabular} 
\caption{Ergebnisse der Klausur Programmierung I\label{noten}}
\end{table}


Eines der wichtigsten Java-Programme \"uberhaupt ist in \vref{hello} zu sehen.

\begin{programm}[htbp]
\begin{lstlisting}
public class HelloDHBW {
  public static void main ( String[] args ) {
    System.out.println ( "Hello DHBW" );
  } // main
} // HelloDHBW
\end{lstlisting}
\caption{Die Klasse \texttt{HelloDHBW}\label{hello}}
\end{programm}



\newpage
% J\"org Baumgart
% 2012-06-01
%
%
\section{Definition und Erzeugung von Querverweisen}

Die Grundlage f\"ur die Erzeugung eines Querverweises bildet die Definition eines 
\textbf{Labels}, z.\,B. \verb+\label{querverweis1}+\label{querverweis1}.

Mit dem Kommando \verb+\vref+, z.\,B. \verb+\vref{querverweis1}+, wird ein Querverweis mit 
den beiden folgenden Eigenschaften erzeugt:

\begin{seList}
\item 
Falls sich das Label auf eine \textsl{Abbildung}, eine \textsl{Tabelle}, ein \textsl{Listing} oder eine 
\textsl{Gleichung} bezieht, wird zus\"atzlich zur entsprechenden Nummer ein Text mit ausgegeben.
Beispielsweise erzeugt \verb+\vref{noten}+ \vref{noten}. Die zugeh\"origen Labels sind dann innerhalb 
der \verb+figure+-, \verb+table+-, \verb+programm-+ oder \verb+equation+-Umgebung definiert. 
Die auszugebenden Texte k\"onnen in der Datei\newline
\hspace*{\fill}\verb+wa-konfiguration-deutsch.tex+\hspace*{\fill}\newline  
umdefiniert werden.

Bezieht sich ein Label auf eine Textstelle, z.\,B. \verb+\label{querverweis1}+, dann wird die Kapitelnummer 
mit dem Zusatz \textsl{Kapitel} ausgegeben: \vref{querverweis1}\newline
F\"ur die Gliederungsebenen \verb+\chapter+, \verb+\section+, \verb+\subsection+, \verb+\subsubsection+ 
und \verb+\paragraph+ kann dieser \textsl{Zusatz} ebenfalls in der Datei \newline
\hspace*{\fill}\verb+wa-konfiguration-deutsch.tex+\hspace*{\fill}\newline 
umdefiniert werden. 
\item
Wenn sich der Querverweis auf die aktuelle Seite bezieht, dann wird keine Seitennummer ausgegeben.
\end{seList}

Bei der Verwendung des \verb+\vref+-Kommandos ist zu beachten, dass vor dem auszugebenden Text ein Leerzeichen 
eingef\"ugt wird. Im Normalfall hat dieses keine weitere Auswirkung. Wenn allerdings ein Absatz direkt mit einem 
\verb+\vref+-Kommando beginnt, dann wird der entsprechende Text nicht linksb\"undig ausgegeben, d.\,h. es liegt 
eine Verletzung des Blocksatzes vor.%

\vref{noten} stellt einen blocksatzverletzenden Querverweis dar.

Dieses \textsl{Problem} l\"asst sich durch die Anwendung des \verb+\vref*+-Kommandos vermeiden.\label{vrefstern} 

\vref*{noten} stellt einen nicht blocksatzverletzenden Querverweis dar.

Allerdings f\"uhrt die Verwendung des \verb+\vref*+-Kommandos innerhalb eines Satzes auch wieder 
zu einem nicht gew\"unschten Ergebnis: Das in \vref*{noten} dargestellte Klausurergebnis ... .%
\footnote{Der Grund, warum die Kommandos \texttt{\textbackslash{}vref} 
und \texttt{\textbackslash{}vref*}
in dieser Form definiert wurden, erschlie{\ss}t sich dem Autor dieses Dokuments allerdings nicht!}

Mit dem Kommande \verb+\pageref+ wird lediglich die Seitennummer ausgegeben, z.\,B. \verb+\pageref{noten}+ \pageref{noten}
oder \verb+\pageref{querverweis1}+ \pageref{querverweis1}.




\section{Definition und Anwendung von zwei neuen Listenumgebungen}

\subsection{Das Layout der Standardlistenumgebung von \LaTeX}

Stichpunktlisten werden in \LaTeX{} mit der \verb+itemize+-Umgebung erzeugt. 
Die Stichpunktliste 

\begin{itemize}
\item 1. Stichpunkt der ersten Ebene
\begin{itemize}
\item 1. Stichpunkt der zweiten Ebene
\item 2. Stichpunkt der zweiten Ebene
\begin{itemize}
\item 1. Stichpunkt der dritten Ebene
\item 2. Stichpunkt der dritten Ebene
\begin{itemize}
\item 1. Stichpunkt der vierten Ebene
\item 2. Stichpunkt der vierten Ebene
\end{itemize}
\end{itemize}
\end{itemize}
\item 2. Stichpunkt der ersten Ebene
\item 3. Stichpunkt der ersten Ebene
\end{itemize}

wird durch die folgenden Anweisungen erreicht:

\vspace*{-\baselineskip}

\begin{verbatim}
\begin{itemize}
\item 1. Stichpunkt der ersten Ebene
\begin{itemize}
\item 1. Stichpunkt der zweiten Ebene
\item 2. Stichpunkt der zweiten Ebene
\begin{itemize}
\item 1. Stichpunkt der dritten Ebene
\item 2. Stichpunkt der dritten Ebene
\begin{itemize}
\item 1. Stichpunkt der vierten Ebene
\item 2. Stichpunkt der vierten Ebene
\end{itemize}
\end{itemize}
\end{itemize}
\item 2. Stichpunkt der ersten Ebene
\item 3. Stichpunkt der ersten Ebene
\end{itemize}
\end{verbatim}

\subsection{Die neue Listenumgebung \texttt{seList} f\"ur Stichpunktlisten}

Weder die Einr\"uckung der einzelnen Ebenen noch die gro{\ss}en Abst\"ande zwischen den einzelnen Stichpunkten sind bei der \verb+itemize+-Umgebung 
bez\"uglich des Layouts sonderlich \"uberzeugend. 

Daher wird eine neue \verb+seList+-Umgebung zur Verf\"ugung gestellt. 

\begin{seList}
\item 1. Stichpunkt der ersten Ebene
\begin{seList}
\item 1. Stichpunkt der zweiten Ebene
\item 2. Stichpunkt der zweiten Ebene
\begin{seList}
\item 1. Stichpunkt der dritten Ebene
\item 2. Stichpunkt der dritten Ebene
\begin{seList}
\item 1. Stichpunkt der vierten Ebene
\item 2. Stichpunkt der vierten Ebene
\begin{seList}
\item 1. Stichpunkt der f\"unften Ebene
\item 2. Stichpunkt der f\"unften Ebene
\end{seList}
\end{seList}
\end{seList}
\end{seList}
\item 2. Stichpunkt der ersten Ebene
\item 3. Stichpunkt der ersten Ebene
\end{seList}

Der \LaTeX-Quelltext f\"ur diese Liste ist: 

\vspace*{-\baselineskip}
\begin{verbatim}
\begin{seList}
\item 1. Stichpunkt der ersten Ebene
\begin{seList}
\item 1. Stichpunkt der zweiten Ebene
\item 2. Stichpunkt der zweiten Ebene
\begin{seList}
\item 1. Stichpunkt der dritten Ebene
\item 2. Stichpunkt der dritten Ebene
\begin{seList}
\item 1. Stichpunkt der vierten Ebene
\item 2. Stichpunkt der vierten Ebene
\begin{seList}
\item 1. Stichpunkt der f\"unften Ebene
\item 2. Stichpunkt der f\"unften Ebene
\end{seList}
\end{seList}
\end{seList}
\end{seList}
\item 2. Stichpunkt der ersten Ebene
\item 3. Stichpunkt der ersten Ebene
\end{seList}
\end{verbatim}

\vspace*{-\baselineskip}
Neben der Eigenschaft, im Gegensatz zur \verb+itemize+-Umgebung f\"unf Verschachtelungsebenen angeben zu k\"onnen, ist es m\"oglich,
die Zeilenabst\"ande f\"ur die einzelnen Ebenen zu konfigurieren. 

Mit dem Kommando \newline 
\hspace*{\fill}\verb+\seSetlistbaselineskip{b1}{b2}{b3}{b4}{b5}+\hspace*{\fill}\newline 
kann f\"ur die Verschachtelungsebene $i$ der Grundlinienabstand \texttt{b$_{i}$} festgelegt 
werden. Als Einheit wird der Wert von \verb+\baselineskip+ (Grundlinienabstand des Dokuments) verwendet. Die folgenden Werte sind f\"ur ein Dokument voreingestellt:\newline
\hspace*{\fill}\verb+\seSetlistbaselineskip{1}{0.75}{0.75}{0.75}{0.75}+\hspace*{\fill}\newline\vspace*{-\baselineskip}

Mit dem Kommando \newline
\hspace*{\fill}\verb+\seResetlistbaselineskip{}+\hspace*{\fill}\newline
wird die letzte \"Anderung der Werte r\"uckg\"angig gemacht.

\newpage
\subsection{Die neue Listenumgebung \texttt{seToplist} f\"ur Listen mit einem Label und Aufz\"ahlungslisten}

Die neue Listenumgebung \verb+seToplist+ erlaubt es, jeden Stichpunkt mit einem Label zu versehen.
Die Liste\footnote{Die folgenden Werte sind frei erfunden.} 

\begin{seToplist}{Mercedes Benz:}
\item[Audi:] 400000 Gesamtverk\"aufe
\begin{seToplist}{3er Reihe:}
\item[A4:] 200000 Verk\"aufe
\item[A5:] 50000 Verk\"aufe
\item[A6:] 150000 Verk\"aufe
\end{seToplist}
\item[Mercedes Benz:] 500000 Gesamtverk\"aufe 
\item[BMW:] 650000 Gesamtverk\"aufe 
\begin{seToplist}{3er Reihe:}
\item[1er Reihe:] 100000 Verk\"aufe
\item[3er Reihe:] 300000 Verk\"aufe
\item[5er Reihe:] 250000 Verk\"aufe
\end{seToplist}
\end{seToplist}

wird durch die folgenden \LaTeX-Anweisungen erzeugt:

\vspace*{-\baselineskip}
\begin{verbatim}
\begin{seToplist}{Mercedes Benz:}
\item[Audi:] 400000 Gesamtverk\"aufe
\begin{seToplist}{3er Reihe:}
\item[A4:] 200000 Verk\"aufe
\item[A5:] 50000 Verk\"aufe
\item[A6:] 150000 Verk\"aufe
\end{seToplist}
\item[Mercedes Benz:] 500000 Gesamtverk\"aufe 
\item[BMW:] 650000 Gesamtverk\"aufe 
\begin{seToplist}{3er Reihe:}
\item[1er Reihe:] 100000 Verk\"aufe
\item[3er Reihe:] 300000 Verk\"aufe
\item[5er Reihe:] 250000 Verk\"aufe
\end{seToplist}
\end{seToplist}
\end{verbatim}

\vspace*{-\baselineskip}
Der Parameter \verb+par+ von \verb+\begin{seToplist}{par}+ definiert die Breite des Labels f\"ur die 
zugeh\"orige Liste.

F\"ur die \verb+seToplist+-Umgebung k\"onnen ebenfalls f\"unf Verschachtelungsebenen definiert werden. 
\"Uber die Kommandos \newline
\hspace*{\fill}\verb+\seSettoplistbaselineskip{b1}{b2}{b3}{b4}{b5}+\hspace*{\fill}\newline 
bzw. \newline
\hspace*{\fill}\verb+\seResettoplistbaselineskip{}+\hspace*{\fill}\newline
lassen sich analog zur \verb+seList+-Umgebung die Grundlinienabst\"ande der einzelnen Verschachtelungsebenen 
ver\"andern bzw. zur\"ucksetzen. Die folgenden Werte sind f\"ur ein Dokument voreingestellt:\newline
\hspace*{\fill}\verb+\seSettoplistbaselineskip{1}{0.75}{0.75}{0.75}{0.75}+\hspace*{\fill}\newline\vspace*{-\baselineskip}

Durch eine entsprechende Wahl der Labels k\"onnen Aufz\"ahlungslisten erzeugt werden:

\begin{seToplist}{a)}
\item[a)] Deutsche Automarken
\begin{seToplist}{1)}
\item[1)] Mercedes Benz
\item[2)] Audi 
\item[3)] VW
\item[4)] BMW 
\end{seToplist}
\item[b)] Japanische Automarken
\begin{seToplist}{1)}
\item[1)] Toyota
\item[2)] Honda
\item[3)] Mazda
\end{seToplist}
\end{seToplist}






\newpage
\section{\"Anderung der Schrifttypen im Dokument}

Standardm\"a{\ss}ig wird in dieser Vorlage f\"ur die \"Uberschriften, die Kopf- und Fu{\ss}zeilen 
sowie das Titelblatt eine serifenlose Schrift verwendet, w\"ahrend der Textteil in einer Serifenschrift 
gesetzt ist.

Soll das gesamte Dokument in einer \textsf{\textbf{serifenlosen Schrift}} gesetzt werden, dann ist in 
der Konfigurationsdatei \verb+wa-konfiguration.tex+ das Kommando \newline
\hspace*{\fill}\verb+\renewcommand{\familydefault}{\sfdefault}+\hspace*{\fill}\newline
zu verwenden.

Soll das gesamte Dokument in einer \textbf{Serifenschrift} gesetzt werden, dann ist in 
der Konfigurationsdatei \verb+wa-konfiguration.tex+ das Kommando \newline
\hspace*{\fill}\verb+\renewcommand{\sffamily}{\normalfont}+\hspace*{\fill}\newline
zu verwenden. Nach dieser \"Anderung ist es nicht mehr m\"oglich, \"uber das 
Kommando \verb+\textsf{}+ einen Textteil in einer serifenlosen Schrift zu setzen.




\section{Anpassungen des Gesamtlayouts}

\subsection{\"Anderung des vertikalen Zwischenraums beim Start eines neuen Kapitels}

Um den vertikalen Zwischenraum zu ver\"andern, den \LaTeX{} automatisch beim 
Start eines neuen Kapitels erzeugt, kann das Kommando \verb+\seNoChapterSkip+ 
verwendet werden. Dieses Kommando wird direkt vor \verb+\begin{document}+ eingef\"ugt.
Es besitzt einen optionalen Parameter, \"uber den ein Wert angegeben werden kann. Der 
Defaultwert ist \texttt{-14mm}. Damit wird erreicht, dass bei einem neuen Kapitel kein 
zus\"atzlicher vertikaler Zwischenraum eingef\"ugt wird. 

Beispiele:

\begin{seList}
\item
\verb+\seNoChapterSkip{}+\newline \verb+\begin{document}+ \newline
Es wird kein vertikaler Zwischenraum beim Beginn eines neuen Kapitels erzeugt.
\item
\verb+\seNoChapterSkip[11.5mm]+\newline \verb+\begin{document}+ \newline
Es wird der vertikale Zwischenraum erzeugt, der auch ohne Angabe dieses 
Kommandos Verwendung findet.
\item
\verb+\seNoChapterSkip[21.5mm]+\newline \verb+\begin{document}+ \newline
Im Vergleich zu dem standardm\"a{\ss}ig erzeugten vertikalen Zwischenraum 
wird ein 10\,mm gr\"o{\ss}erer Zwischenraum beim Start eines neuen Kapitels 
erzeugt.
\end{seList}

\subsection{M\"ogliche Layout-\"Anderungen f\"ur Seminararbeiten}

\subsubsection{Verwendung kleinerer Schriftgr\"o{\ss}en f\"ur \"Uberschriften}

Die Verwendung kleinerer Schriftgr\"o{\ss}en f\"ur \"Uberschriften wird durch 
die Angabe des Kommandos \verb+\KOMAoption{headings}{small}+ direkt 
vor \verb+\begin{document}+ erreicht.

Soll dieses Kommando bei Seminarbeiten mit \verb+\seNoChapterSkip{}+ kombiniert 
werden, ist die folgende Reihenfolge erforderlich:

\begin{seList}
\item[] \verb+\KOMAoption{headings}{small}+\newline
\verb+\seNoChapterSkip[-12.25mm]+\newline
\verb+\begin{document}+
\end{seList}

Da kleinere Schriftgr\"o{\ss}en f\"ur die \"Uberschriften verwendet werden, sollte das Kommando \verb+\seNoChapterSkip+ mit 
dem optionalen Parameter \texttt{-12.25mm} aufgerufen werden.

\subsubsection{Unterdr\"uckung des Seitenvorschubs f\"ur die folgenden Kapitel}

Das Kommando \verb+\seChaptersWithoutNewpage{}+ unterdr\"uckt den Seitenvorschub des \verb+\chapter+-Kom\-man\-dos f\"ur die folgenden Kapitel. 

Wenn dieses Kommando in Kombination mit \verb+\seNoChapterSkip{}+ benutzt wird, dann sollte nach jedem Kapitelende noch das Kommando 
\verb+\seChapterEndSkip{}+ ausgef\"uhrt werden, damit ein vern\"unftiger Abstand zur folgenden Kapitel\"uberschrift entsteht.

\verb+\seChapterNewpage{}+ erzeugt f\"ur die folgenden Kapitel wieder Seitenvorsch\"ube.






%
%  Erzeugung eines Glossars
%
% Achtung: Das Glossar wird nur ausgegeben, wenn mindestens ein Eintrag in der Arbeit 
%                definiert wurde
%
%
\newpage
\sePrintGlossary{}


%
% Literaturverzeichnisses
%
%\newpage
\sePrintBibliography{}

%  Erzeugung von Eintr\"agen im Literaturverzeichnis
%
%  Achtung: in einer Seminar-/Projekt/Bachelorarbeit darf da \nocite-Kommando nicht verwendet werden,
%                 da es einen Eintrag im Literaturverzeichnis erzeugt, ohne dass eine 
%                 entsprechende Literaturreferenz im Text der Arbeit angegeben wird
%
%
%
\nocite{DHBW:SG}
\nocite{KM:KS}
\nocite{Dud06}
\nocite{Dud09}
\nocite{Bri:WA}
\nocite{RP:WA}
\nocite{Sch:WAS}
\nocite{BSS:WA}
\nocite{Kor:WA}
\nocite{MK:GWA}
\nocite{ADG:WA}
\nocite{The:WA}
\nocite{BA:WA}
\nocite{Dij:CRT}
\nocite{BC:Cur}
\nocite{Par:ECP}
\nocite{Bro:SBE}
\nocite{GI:ADI}
\nocite{GI:AZI}
\nocite{Den:CD}
\nocite{LMS:Icb}
\nocite{Fre:SIF}




%
% Festlegung des grundlegenden Formatierungsstils des Literaturverzeichnis
%
\bibliographystyle{jurabib}

% Eigentliche Ausgabe der in der Arbeit verwendeten Quellen
%
%
% Angabe der bib-Dateien, in denen die Quellen beschrieben sind;
% die Angabe geht davon aus, dass eine wa.bib-Datei in demselben 
% Verzeichnis liegt, wie se-ba-vorlage.tex
%

% 2016-04-01
%
% Umbenennung von Quellen- in Literaturverzeichnis (nicht empfohlen, da sich 
% die 
% 
%\renewcommand*{\bibname}{Quellenverzeichnis}
\seBibliography{wa}


%
% Erzeugung der ehrenw\"ortlichen Erkl\"arung
%
% Der optionale Parameter kann verwendet werden, um f\"ur das Thema der Arbeit eine 
% andere Formatierung vorzunehmen; das sollte in der Regel nicht erforderlich sein;
% ausserdem besteht die Gefahr inkonsistenter Titel auf dem Titelblatt und in der 
% ehrenw\"ortlichen Erkl\"arung
%
\seEhrenwoertlicheErklaerung{} % dieses Kommando sollte standardm\"assig verwendet werden
%\seEhrenwoertlicheErklaerung[\LaTeX-Vorlage zur Anfertigung \seThemaWaArbeit{} (Version \version{})]


\end{document}











