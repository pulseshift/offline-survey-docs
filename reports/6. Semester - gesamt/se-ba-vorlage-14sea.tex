% Konfigurationsdatei f\"ur die Pfaddefinitionen einlesen
%  se-wa-pfade.tex
%
%
%  J\"org Baumgart
%  2012-12-20
%  
%  Pfaddefinitionen (Ordnerdefinitionen) f\"ur das Einlesen von
%  -- .sty-Dateien und
%  -- Textbaustenen f\"ur die Hinweise zur Verwendung von LaTeX
%  -- jpg-Bildern
%
\newcommand{\seWaPathSty}{se-wa-styles}
\newcommand{\seWaPathText}{se-wa-textbausteine-vorlagen}
\newcommand{\seWaPathJpg}{images}

%
%
% Festlegung der Sprache: 
\newcommand{\seWaSprache}{deutsch}
%\newcommand{\seWaSprache}{englisch}

%
% Einlesen der .sty-Dateien
%
%  se-wa-input-styles-v098.tex
%
%  Joerg Baumgart 01.08.2011
%
%  Zusammenfassung und Konfiguration wichtiger Styles f\"ur die 
%  Erzeugung von Seminar-, Projekt- und Bachelorarbeiten
%
%  2012-03-12: auf Version 0.94 umgestellt
%
%
% 2012-12-13: auf Version 0.95 umgestellt
%                     Sprachoptionen englisch/deutsch zusammengef\"uhrt
%                     bchart.sty hinzugenommen
%                 
%
% 2013-01-27: auf Version 0.96 umgestellt
%                     algorithm2e-Paket integriert
%
%
% 2013-07-08: auf Version 0.97 umgstellt 
%                     utf8, Fehlerkorrekturen bei pa1
%
% 2014-02-02: auf Version 0.971 umgestellt
%                     Workaround für Fehler im KOMAScript 3.2
%                     KOMAoption listof un documentclass übernommen
%
%
%
%
% 2014-07-22: etex-Paket hinzugenommen
%
%
% 2016-04-01: Sperrvermerk und Ehrenwörtliche Erklärung aus der PO 2015
%
%
%
% 2016-12-18: auf Version 0.98 umgestellt
%                     bchart.sty und algorithm2e haben veraltete Kommandos verwendet (\sf und \bf), 
%                     die unter MacTeX 2016 nicht mehr unterstützt werden 
%

\documentclass[12pt,BCOR=10mm,headinclude=on,footinclude=off,bibliography=totoc,listof=ignorechapter]{scrreprt}
\usepackage{etex}
\usepackage[T1]{fontenc}
\usepackage[utf8]{inputenc}
\usepackage{ifthen}
% 2012-12-13
\ifthenelse{\equal{\seWaSprache}{deutsch}}{% Deutsche Einstellungen
\usepackage[ngerman]{babel}% 
}{% Englische Einstellungen
\usepackage[english]{babel}% 
}

\usepackage{lmodern}

\usepackage{tikz} % Graphikpaket, das zu pdfLaTeX kompatibel ist
\usepackage{xkeyval} % Definition von Kommandos mit mehreren optionalen Argumenten
\usepackage{listings} % Formatierung von Programmlistings
\usepackage{graphicx} % Einbinden von Graphiken
\usepackage{color}
\usepackage{\seWaPathSty/slashbox} % Diagonalen in Tabellenfeldern
\usepackage{framed} % Erzeugung schwarzer Linien am linken Rand zur Hervorhebung von Textteilen
\usepackage{caption} % Korrektes Setzen einer mehrzeiligen float-Unterschrift bei neu definierten float-Umgebungen
\usepackage{floatrow}
% 2012-12-13
\usepackage{\seWaPathSty/bchart} % Kommandos zur Erzeugung von Balkendiagrammen
% 2013-01-27 
\usepackage[boxed,ngerman]{\seWaPathSty/algorithm2e}
%\usepackage[tworuled,vlined,ngerman]{\seWaPathSty/algorithm2e}


% Es wird jeweils die sty-Datei importiert und entsprechende Konfigurationseinstellungen werden vorgenommen

\usepackage{\seWaPathSty/se-jb-scrpage2} % Formatierung der Kopf- und Fu{\ss}zeilen
\usepackage{\seWaPathSty/se-jb-footmisc}    % Fussnoten besser formatieren

\usepackage{\seWaPathSty/se-jb-glossaries-v097} % Abk\"urzungsverzeichnis, Symbolverzeichnis, Glossar
   
\usepackage{\seWaPathSty/se-jb-floatrow}    % Definition und Konfiguration von float-Umgebungen (figure, table, die neue programm-Umgebung)
% Achtung: se-jb-varioref muss nach se-jb-floatrow importiert werden; 
% andernfalls ist der counter programm f\"ur die labelformat-Anweisung noch nicht definiert   
\usepackage{\seWaPathSty/se-jb-varioref-v097}   % Definition von Querverweisen
\usepackage{\seWaPathSty/se-jb-chngcntr}   % Kapitelweise oder globale Nummerierung von Abbildungen etc.
   
\usepackage{\seWaPathSty/se-jb-listen} % Definition neuer, besser formatierter Listen
% 2014-02-02
%\usepackage{\seWaPathSty/se-jb-kommandos-v097} % neue Kommandos f\"ur Seminar-, Projekt- und Bachelorarbeiten
%\usepackage{\seWaPathSty/se-jb-kommandos-v0971} % neue Kommandos f\"ur Seminar-, Projekt- und Bachelorarbeiten
% 2016-04-01
\usepackage{\seWaPathSty/se-jb-kommandos-v0972} % neue Kommandos f\"ur Seminar-, Projekt- und Bachelorarbeiten
% 2012-12-13
\ifthenelse{\equal{\seWaSprache}{englisch}}{\usepackage{\seWaPathSty/se-jb-kommandos-englisch-v0972}}{}

% 2016-12-18
\renewcommand{\sf}{\sffamily}
\renewcommand{\bf}{\bfseries}




%
% Individuelle Konfiguration des Dokumentes
%
%  Individuelle Konfiguration einer Projektarbeit/Bachelorarbeit
%
%
%
%

% 2012-10-27
%
% \"Anderung des Schrifttyps f\"ur das gesamte Dokument
%
% Das gesamte Dokument wird in einer serifenlosen Schrift gesetzt
%\renewcommand{\familydefault}{\sfdefault}
%
% Das gesamte Dokument wird in einer Serifenschrift gesetzt
% Achtung: serifenlose Schriften sind jetzt grunds\"atzlich nicht mehr nutzbar!
%
%\renewcommand{\sffamily}{\normalfont}

% 2012-12-05
%
% Verwendung des url-Pakets
% Durch den optionalen Paremeter hyphens wird eine Trennung 
% von URLs auch nach Bindestrichen erlaubt
\usepackage[hyphens]{url}


% 2012-10-27
%
%
% Literaturverzeichnis
% 
% Literaturverzeichnis gem\"ass den Vorgaben von Theisen aufbauen
\usepackage{\seWaPathSty/se-jb-jurabib-theisen} 
% Verwendung der Harvard-Zitierweise
%\usepackage{\seWaPathSty/se-jb-jurabib-harvard} 

% Weitere Optionseinstellungen f\"ur das Koma-Script
%
% Zwischen Abs\"atzen einen Abstand von 0.5 \baselineskip erzeugen
\KOMAoption{parskip}{full}
%
% Vergleiche Duden "Gliederung von Nummern, S.111" 
% DIN 5008 anschauen, wenn sie neu ver\"offentlicht wurde
\KOMAoption{numbers}{noendperiod}
%
%



%  Voreinstellungen f\"ur floats
%  Durch die verwendeten Parameter wird die Wahrscheinlichkeit deutlich kleiner, 
%  dass Gleitobjekte (z. B. Abbildungen) ans Ende des Dokumentes verschoben 
%  werden; 
%  Achtung: clearpage erzwingt die Ausgabe von Gleitobjekten
%
\renewcommand{\topfraction}{1}  % Gleitobjekte d\"urfen eine Seite zu 100% belegen 
\renewcommand{\bottomfraction}{1} % Entsprechender Wert f\"ur den unteren Teil der Seite
\renewcommand{\textfraction}{0} % Eine Seite darf auch ohne Fliesstext existieren
%%%\renewcommand{\floatpagefraction}{1} % Bedeutung unklar, daher keine Ver\"anderung des Vorgabewertes 
                                                                        % von 0.5; eventuell bringt ein \"Anderung auf 1 etwas, wenn 
                                                                         % Probleme mit floats auftreten
                                                                         
                                                                         
                                                                         
% Konfiguration von Programm-Listings
% 
% Achtung: hier gibt es nahezu beliebig viele weitere Konfigurationm\"oglichkeiten; vgl. Paketdokumentation
%
\lstset{language=Java,basicstyle=\ttfamily,keywordstyle=\color{blue},captionpos=b,aboveskip=0mm,belowskip=0mm,
          xleftmargin=0em}               
          
%
% Grundkonfiguration der Abs\"ande zwischen den Items der maximal f\"unf Verschachtelungsebenen der 
% neuen Listenumgebungen
%                                                                             
% Initialisierung der Abst\"ande zwischen den items f\"ur seList; Grundeinheit: 0.5\baselineskip; siehe se-jb-listen
\seSetlistbaselineskip{1}{0.75}{0.75}{0.75}{0.75}
% Initialisierung der Abst\"ande zwischen den items f\"ur seToplist; Grundeinheit: 0.5\baselineskip; siehe se-jb-listen
\seSettoplistbaselineskip{1}{0.75}{0.75}{0.75}{0.75}     


% Einlesen der sprachabh\"angigen Konfigurationsdatei
%
%
\ifthenelse{\equal{\seWaSprache}{deutsch}}{% deutsch
% wa-konfiguration-deutsch
%
% 2012-12-13
% 
% Diese Datei wird f\"ur die Sprachoption deutsch verwendet, d. h.  
% \newcommand{\seWaSprache}{deutsch}
%
%
% In dieser Datei k\"onnen Neudefinitionen vorgenommen werden f\"ur:
% -- Verzeichnisse
% -- Unter-/\"Uberschriften von Abbildungen, Tabellen und Listings
% -- Querverweise innerhalb des Textes

% 2013-01-26: Konfiguration des Algorithmenverzeichnis
%
%
%

% 2013-07-08: Querverweis auf Anhang hinzugenommen
%
%
%


%
%  Konfiguration der verschiedenen Verzeichnisse
%
%  abstandEintrag: Wert wird mit \baselineskip multipliziert
%

%
%  Abbildungsverzeichnis
%
\seKonfigurationAbb[
%verzeichnisname=Abbildungsverzeichnis,
labeltextLinks=, % kein Text links;
%labeltextRechts=:,
labelbreite=1cm,
%labeleinzug=1cm,
%abstandEintrag=1,
%newpage=ja,
%pnumwidth=20mm,
%dotsep=1000,
%tocrmarg=4.5cm,
%abstandVerzeichnis=-1mm
]

%
% LIstingverzeichnis
%
\seKonfigurationPrg[
%verzeichnisname=Listing-Verzeichnis,
labeltextLinks=,
%labeltextRechts=:,
labelbreite=1cm,
%labeleinzug=2cm,
%abstandEintrag=1,
%newpage=ja,
%pnumwidth=20mm,
%dotsep=1000,
%tocrmarg=4.5cm,
%abstandVerzeichnis=-10mm
]

% 2013-01-26
%
% Algorithmenverzeichnis
%
\seKonfigurationAlg[
%verzeichnisname=Algorithmen-Verzeichnis,
labeltextLinks=,
%labeltextRechts=:,
labelbreite=1cm,
%labeleinzug=2cm,
%abstandEintrag=1,
%newpage=ja,
%pnumwidth=20mm,
%dotsep=1000,
%tocrmarg=4.5cm,
%abstandVerzeichnis=-10mm
]




%
% Tabellenverzeichnis
%
\seKonfigurationTab[
%verzeichnisname=Liste der Tabellen,
labeltextLinks=,
%labeltextRechts=:,
labelbreite=1cm,
%labeleinzug=0.5cm,
%abstandEintrag=1,
%newpage=ja,
%pnumwidth=20mm,
%dotsep=1000,
%tocrmarg=4.5cm,
%abstandVerzeichnis=-10mm
]

%
% Abk\"urzungsverzeichnis
%
\seKonfigurationAbk[
%verzeichnisname=Liste der Abk\"urzungen,
%labelbreite=3cm,
%labeleinzug=0.5cm,
%abstandEintrag=1,
%newpage=ja,
%abstandVerzeichnis=-10mm
]

%
% Symbolverzeichnis
% 
\seKonfigurationSym[
%verzeichnisname=Liste der Symbole,
%labelbreite=4cm,
%labeleinzug=3.5cm,
%abstandEintrag=1,
%newpage=ja,
%abstandVerzeichnis=-10mm
]

%
% Glossar
%
\seKonfigurationGlo[
%verzeichnisname=Glossar,
%abstandEintrag=0,
]



% (eventuelle) Neudefinition f\"ur die Unter-/\"Uberschriften von Abbildungen, Tabellen und Listings
%
%
%\renewcommand{\seCaptionNameAbbildung}{Abb.}
%\renewcommand{\seCaptionNameTabelle}{Tab.}
%\renewcommand{\seCaptionNameProgramm}{Prg.}


% % (eventuelle) Neudefinition f\"ur Querverweise innerhalb des Textes
%
%
%
%\renewcommand{\seQuerverweisSeite}{Seite}
%\renewcommand{\seQuerverweisAbbildung}{Abb.}
%\renewcommand{\seQuerverweisTabelle}{Tab.}
%\renewcommand{\seQuerverweisProgramm}{Prg.}
%\renewcommand{\seQuerverweisGleichung}{Gl.}
%\renewcommand{\seQuerverweisAlgorithmus}{Alg.}
%
\renewcommand{\seQuerverweisChapter}{Kapitel}
% 2013-07-08
\renewcommand{\seQuerverweisAppendix}{Anhang}
\renewcommand{\seQuerverweisSection}{Kapitel}
\renewcommand{\seQuerverweisSubsection}{Kapitel}
\renewcommand{\seQuerverweisSubsubsection}{Kapitel}
\renewcommand{\seQuerverweisParagraph}{Kapitel}


%
% Kommandos f\"ur die Konfiguration von URL-Eintr\"agen im Literaturverzeichnis
%
\renewcommand*{\biburlprefix}{\jblangle{}URL: }
\renewcommand*{\biburlsuffix}{\jbrangle{}}
\renewcommand*{\bibbudcsep}{ -- }
\AddTo\bibsgerman{\renewcommand*{\urldatecomment}{Zugriff am }}


%
}{% englisch
% wa-konfiguration-englisch
%
% 2012-12-13
% 
% Diese Datei wird f\"ur die Sprachoption englisch verwendet, d. h.  
% \newcommand{\seWaSprache}{englisch}
%
%
% In dieser Datei k\"onnen Neudefinitionen vorgenommen werden f\"ur:
% -- Verzeichnisse
% -- Unter-/\"Uberschriften von Abbildungen, Tabellen und Listings
% -- Querverweise innerhalb des Textes

% 2013-07-08: Querverweis auf Anhang hinzugenommen
%
%
%


%
%  Konfiguration der verschiedenen Verzeichnisse
%
%  abstandEintrag: Wert wird mit \baselineskip multipliziert
%

%
%  Abbildungsverzeichnis
%
\seKonfigurationAbb[
verzeichnisname=List of Figures,
labeltextLinks=, % kein Text links;
%labeltextRechts=:,
labelbreite=1cm,
%labeleinzug=1cm,
%abstandEintrag=1,
%newpage=ja,
%pnumwidth=20mm,
%dotsep=1000,
%tocrmarg=4.5cm,
%abstandVerzeichnis=-1mm
]

%
% LIstingverzeichnis
%
\seKonfigurationPrg[
verzeichnisname=List of Program Listings,
labeltextLinks=,
%labeltextRechts=:,
labelbreite=1cm,
%labeleinzug=2cm,
%abstandEintrag=1,
%newpage=ja,
%%pnumwidth=20mm,
%dotsep=1000,
%tocrmarg=4.5cm,
%abstandVerzeichnis=-10mm
]


% 2013-01-26
%
% Algorithmenverzeichnis
%
\seKonfigurationAlg[
verzeichnisname=List of Algorithms,
labeltextLinks=,
%labeltextRechts=:,
labelbreite=1cm,
%labeleinzug=2cm,
%abstandEintrag=1,
%newpage=ja,
%pnumwidth=20mm,
%dotsep=1000,
%tocrmarg=4.5cm,
%abstandVerzeichnis=-10mm
]


%
% Tabellenverzeichnis
%
\seKonfigurationTab[
verzeichnisname=List of Tables,
labeltextLinks=,
%labeltextRechts=:,
labelbreite=1cm,
%labeleinzug=0.5cm,
%abstandEintrag=1,
%newpage=ja,
%pnumwidth=20mm,
%dotsep=1000,
%tocrmarg=4.5cm,
%abstandVerzeichnis=-10mm
]

%
% Abk\"urzungsverzeichnis
%
\seKonfigurationAbk[
verzeichnisname=List of Abbreviations,
%labelbreite=3cm,
%labeleinzug=0.5cm,
%abstandEintrag=1,
%newpage=ja,
%abstandVerzeichnis=-10mm
]

%
% Symbolverzeichnis
% 
\seKonfigurationSym[
verzeichnisname=List of Symbols,
%labelbreite=4cm,
%labeleinzug=3.5cm,
%abstandEintrag=1,
%newpage=ja,
%abstandVerzeichnis=-10mm
]

%
% Glossar
%
\seKonfigurationGlo[
verzeichnisname=Glossary,
%abstandEintrag=0,
]



% (eventuelle) Neudefinition f\"ur die Unter-/\"Uberschriften von Abbildungen, Tabellen und Listings
%
%
\renewcommand{\seCaptionNameAbbildung}{Figure}
\renewcommand{\seCaptionNameTabelle}{Table}
\renewcommand{\seCaptionNameProgramm}{Listing}
\renewcommand{\seCaptionNameAlgorithmus}{Algorithm}


% % (eventuelle) Neudefinition f\"ur Querverweise innerhalb des Textes
%
%
%
\renewcommand{\seQuerverweisSeite}{page}
\renewcommand{\seQuerverweisAbbildung}{figure}
\renewcommand{\seQuerverweisTabelle}{table}
\renewcommand{\seQuerverweisProgramm}{listing}
\renewcommand{\seQuerverweisGleichung}{equation}
\renewcommand{\seQuerverweisAlgorithmus}{algorithm}
%
\renewcommand{\seQuerverweisChapter}{chapter}
\renewcommand{\seQuerverweisAppendix}{appendix}
\renewcommand{\seQuerverweisSection}{chapter}
\renewcommand{\seQuerverweisSubsection}{chapter}
\renewcommand{\seQuerverweisSubsubsection}{chapter}
\renewcommand{\seQuerverweisParagraph}{chapter}


%
% Kommandos f\"ur die Konfiguration von URL-Eintr\"agen im Literaturverzeichnis
%
\renewcommand*{\biburlprefix}{\jblangle{}URL: }
\renewcommand*{\biburlsuffix}{\jbrangle{}}
\renewcommand*{\bibbudcsep}{ -- }
\AddTo\bibsenglish{\renewcommand*{\urldatecomment}{visited on }}


}

% Kommandos, die direkt nach \begin{document} ausgef\"uhrt werden m\"ussen
%
%
%
\AtBeginDocument{%
\renewcommand{\listfigurename}{\seAbbildungenVerzeichnisname}
\renewcommand{\listtablename}{\seTabellenVerzeichnisname}
\renewcommand{\figurename}{\seCaptionNameAbbildung}
\renewcommand{\tablename}{\seCaptionNameTabelle}
\labelformat{lstlisting}{\seQuerverweisProgramm{} #1}
\renewcommand{\thelstlisting}{\theprogramm}
\pagenumbering{roman}
}
                                                              
                                                                         

%
% Definition von Abk\"urzungen, Symbolen und eventuell Glossareintr\"agen
%
% 2012-03-22 Verwendung des optionalen Parameters f\"ur die Pluralform einer Abk\"urzung
%
% 2012-02-06 Umstellung auf die neuen Kommandos
%
%
%
%  J\"org Baumgart
%  Definition einiger Abk\"urzungen
%  


% Definition von Abk\"urzungen
%
% 1. Parameter: Schluessel (key) der Abkuerzung
% 2. Parameter: Abkuerzung
% 3. Parameter: Vollform
% 4. Parameter: Vollform im Plural (optional; falls nicht definiert, wird der Wert des dritten Parameters verwendet)
%

\seNewAcronymEntry{poc}{PoC}{Proof of Concept}{Proof of Concept}

\seNewAcronymEntry{pwa}{PWA}{Progressive Web App}{Progressive Web App}



% Definition von Symbolen
%
% 1. Parameter: Schluessel (key) des Symbols
% 2. Parameter: Symbol
% 3. Parameter: Text, der die Sortierreihenfolge festlegt (optional; falls nicht definiert, wird der Wert des zweiten 
%                        Parameters verwendet)
% 4. Parameter: Beschreibung des Symbols
%

\seNewSymbolEntry{ND}{ND}{a}{Nutzungsdauer einer Maschine}

\seNewSymbolEntry{pi}{$\pi$}{b}{Die Kreiszahl}




% Definition von Glossareintraegen
%
% 1. Parameter: Schluessel (key) des Glossareintrags
% 2. Parameter: Begriff, der im Glossar definiert wird
% 3. Parameter: Pluralform des Begriffes (optional; falls nicht definiert, wird der Wert des zweiten Parameters verwendet)
%                        Achtung: Pluralform gilt nur fuer das erste Auftreten des Begriffes im Text
% 4. Parameter: Beschreibung des Glossareintrags
%
%
%

\seNewGlossaryEntry{glos:offline_worker}{Offline-Mitarbeiter}{Offline-Mitarbeiter}{Mitarbeiter eines Unternehmens, dem keine Firmen-E-Mail-Adresse zugeordnet ist und der auch keinen anderen Zugang zu einem System hat, welches das Empfangen von mitarbeiterspezifischen digitalen Nachrichten erm\"oglicht.}




% Definition von Glossareintraegen, die gleichzeitig im Abk�rzungsverzeichnis auftreten
%
% 1. Parameter: Schluessel (key) des Glossareintrags
% 2. Parameter: Abk\"urzung
% 3. Parameter: Vollform
% 4. Parameter: Vollform im Plural (optional; falls nicht definiert, wird der Wert des dritten Parameters verwendet)
% 5. Parameter: Beschreibung des Glossareintrags

\seNewAcronymGlossaryEntry{glos:ma}{MA}{Mobile Applikation}{Mobile Applikationen}
{Eine Applikation, die auf einem mobilen Endger\"at ausgef\"uhrt wird.}

% 2012-03-24
% \"Uber den optionalen Parameter in eckigen Klammern wird die Pluralform f\"ur die Abk\"urzung definiert

\seNewAcronymGlossaryEntry[TAen]{glos:ta}{TA}{Transaktion}%
{Transaktionen}%
{Was eine Transaktion ist, sollten Sie ebenfalls bereits wissen!}

 

% Eigene Kommandos
\usepackage[]{setspace}
\usepackage[]{textcomp}
\usepackage[output-decimal-marker={,}]{siunitx}
\usepackage{longtable}
\usepackage{pgfplots}
\usepackage{colortbl}
\usepackage{kantlipsum} % just for the example

%\usepackage[left=25mm,right=35mm,top=30mm,bottom=40mm]{geometry}

% Definition eines Kommandos für mathematische Formeln
\newcommand{\formel}[1]{\begin{center}$#1$\end{center}}
\newcommand{\formelleft}[1]{$#1$}

% Definition eines Kommandos für mathematische Variablen
\newcommand{\variable}[1]{$#1$}

% Definition für Zitate mit Übersetzung
\newcommand{\citeEnglish}[3]{\seFootcite{siehe}{Seite #1, Übersetzung: #2}{#3}}

\newcommand{\anmerkung}[1]{\footnote{Anmerkung: #1}}

\newcommand{\figref}[1]{\footnote{siehe \vref{#1}.}}

% Fußnote passend einrücken 
\setlength{\footnotemargin}{1.3em}

% kein Umbruch bei Fußnote
% siehe http://texwelt.de/wissen/fragen/2421/wie-kann-ich-verhindern-dass-funoten-auf-zwei-seiten-verteilt-ausgegeben-werden
\interfootnotelinepenalty=10000

%\KOMAoption{headings}{small}
\seNoChapterSkip[11.5mm]




%\seIstSeminararbeit{}

\newcommand{\version}{0.98}

% 
% Diese Redefinition ist nur f\"ur den Anhang B der  
% Vorlage (Hinweise zur Installation und \"Ubersetzung)
% notwendig; f\"ur Ihre Bachelorarbeit spielt sie keine Rolle
%
\renewcommand{\seVorlage}{\jobname}


\begin{document}

\begin{titlepage}

\centering

\vspace*{\stretch{2}}

\Large to be done Abschlussbericht Projekt PulseShift

\vspace{\stretch{1}}

\normalsize dauer

\vspace{\stretch{0.5}}

\begin{tabular}{ll}
Submission date & xx.yy.zzzz \\[2ex]
Student's name  & Name \\[2ex]
From            & Place
\end{tabular}

\vspace{\stretch{3}}

\begin{tabular*}{\textwidth}{@{}l@{\extracolsep{\fill}}l@{}}
My name & My professor's name \\[2ex]
        & My supervisor's name
\end{tabular*}

\vspace{\stretch{2}}
\end{titlepage}


%% Erzeugung des Titelblatts
%%
%%
%%
%\seTitelblattWissenschaftlicheArbeit[
%%hilfslinien=ja,
%%dhbwlogoSkalierung=0.5,
%%dhbwlogoDeltaX=2.4,
%%dhbwlogoDeltaY=-10,
%firmenlogo=firmenlogo,
%firmenlogoSkalierung=0.2,
%firmenlogoDeltaX=0,
%firmenlogoDeltaY=0,
%studiengang=\seWirtschaftsinformatik,
%studienrichtung=\seSoftwareEngineering,
%thema=Projekt PulseShift,
%verfasser=Sebastian Schütz Florian Finkel,
%matrikelnummer=9999999,
%kurs=WWI\,15\,SE\,A,
%firma=Ausbildungsfirma,
%% Da im Text ein Komma enthalten ist, muss der Text eingeklammert werden
%%abteilung={Wirtschaftsinformatik, Sales \& Consulting},
%abteilung={Wirtschaftsinformatik, Software Engineering},
%%studiengangsleiterin=,
%studiengangsleiter=Prof. Dr.-Ing. J\"org Baumgart,
%%studiengangsleiter=Prof. Dr. Thomas Holey,
%wissenschaftlicheBetreuerinName=Dr. Melanie Mustermann,
%wissenschaftlicheBetreuerinEmail=melanie.mustermann@musterfirma.de,
%wissenschaftlicheBetreuerinTelefon=0621/999999,
%%wissenschaftlicherBetreuerName=Prof. Dr.-Ing. J\"org Baumgart,
%%wissenschaftlicherBetreuerEmail=joerg.baumgart@dhbw-mannheim.de,
%%wissenschaftlicherBetreuerTelefon=0621/4105\,1216,
%%firmenbetreuerinName=Dipl.-Ing. Ariane Meistermann,
%%firmenbetreuerinEmail=a.meistermann@andere-musterfirma.de,
%%firmenbetreuerinTelefon=06151/88888,
%%firmenbetreuerName=,
%%firmenbetreuerEmail=,
%%firmenbetreuerTelefon=,
%bearbeitungszeitraumVon=28. November 2016,
%bearbeitungszeitraumBis=19. Februar 2017,
%%
%% Datum in englischer Schreibweise
%%bearbeitungszeitraumVon=28 November  2016,
%%bearbeitungszeitraumBis=19 February 2017,
%sperrvermerk=nein
%]
%





% 2012-02-06 Inhaltsverzeichnis muss vor den weiteren Verzeichnisses kommen
%
%
% Ausgabe des Inhaltsverzeichnisses
%
%
\seInhaltsverzeichnis[%
einrueckung=ja,
gliederungsebenen=4
]




% Ausgabe der verschiedenen Verzeichnisse
% abk: Abk\"urzungsverzeichnis
% sym: Symbolverzeichnis
% abb: Abbildungsverzeichnis
% tab: Tabellenverzeichnis
% prg: Listingverzeichnis
% alg: Algorithmenverzeichnis
%
%
% Achtung: Abk\"urzungs- und Symbolverzeichnis werden nur ausgegeben, wenn mindest ein Symbol bzw. 
%                mindestens eine Abk\"urzung in der Arbeit verwendet wurden
%
%
% gliederungsebene:
% -- section: die Verzeichnisse werden einem Kapitel "Verzeichnisse" untergliedert
% -- chapter: die Verzeichnisse sind jeweils eigene Kapitel
% imInhaltsverzeichnis: ja/nein -- Sollen die Verzeichnisse im Inhaltsverzeichnis enthalten sein?
\seVerzeichnisse[gliederungsebene=section,imInhaltsverzeichnis=ja]{abk}{sym}{abb}{}{}%{tab}{prg}



%-----------------------------
%- - - - - - - - - - - - - - -
%-----------------------------
%Hier für 1,5 Zeilenabstand
\onehalfspacing

%-----------------------------
%- - - - - - - - - - - - - - -
%-----------------------------

\pagenumbering{arabic}

\chapter{Einführung und Projektrahmen}
\chapter{Einführung und Projektrahmen}
\chapter{Einführung und Projektrahmen}
\input{chapters/introduction/introduction}
\input{chapters/introduction/goals}
\input{chapters/introduction/benefit}

\section{Ziel des Projekts}
\label{sec:introduction:goals}
Das Ziel des Projekts ist die Erarbeitung eines Lösungsportfolios, dass die Teilnahme von Offline-Mitarbeitern an den Umfragen von PulseShift ermöglicht. Dieses Ziel besteht aus zwei Teilzielen:

\begin{enumerate}
\item Es sollen mögliche Kanäle konzeptionell erarbeitet werden, um Offline-Mitarbeitern die Teilnahme an der Umfrage zu ermöglichen.
\item Die Kanäle die das größte Potential aufweisen sollen als \gls{poc} umgesetzt werden.
\end{enumerate}


\section{Erwarteter wirtschaftlicher Nutzen}

Die in diesem Projekt konzeptionell erarbeiteten Kanäle sollen PulseShift einen Überblick geben, wie eine Umfrage an Offline-Mitarbeiter verteilt werden kann. Daraus soll abgeleitet werden können, welche Umfragekanäle im Hinblick auf Kosten und Nutzen geeignet sind, um in das Lösungsportfolio von PulseShift aufgenommen zu werden. So kann teuren Investitionen in nicht geeignete Umfragekanäle vorgebeugt werden. Gleichzeitig kann gezielt in die Umsetzung von Kanälen mit hohem Potential investiert werden. 

Die entwickelten \gls{poc}s sollen zum einen für PulseShift als Demonstration


 aufzeigen, wie die Umsetzung der jeweiligen Kanäle aussehen kann. So 

 Zum Anderen



Die in dem Projekt entwickelten  sollen die Umfragemöglichkeiten für PulseShift bei Mitarbeitern ohne Email-Adresse gewährleisten.

 
 Weiterhin sollen die PoCs mit einer möglichst hohen Akzeptanz in der zu untersuchenden Zielgruppe anwendbar sein. 
 
 Der wirtschaftliche Mehrwert liegt somit einerseits in der kostengünstigen Umsetzung der Umfrage und der einhergehenden Kosteneinsparungen, andererseits auch in den möglichst aussagekräftigen Ergebnissen, die durch eine Steigerung der Anzahl an befragten Mitarbeitern erreicht werden. 
 
 Durch diese werden dem Unternehmen Handlungsmöglichkeiten zur besseren Durchführung von Digitalisierungsmaßnahmen aufgezeigt und somit weitere Optionen zur Kosteneinsparung dargelegt.

\section{Ziel des Projekts}
\label{sec:introduction:goals}
Das Ziel des Projekts ist die Erarbeitung eines Lösungsportfolios, dass die Teilnahme von Offline-Mitarbeitern an den Umfragen von PulseShift ermöglicht. Dieses Ziel besteht aus zwei Teilzielen:

\begin{enumerate}
\item Es sollen mögliche Kanäle konzeptionell erarbeitet werden, um Offline-Mitarbeitern die Teilnahme an der Umfrage zu ermöglichen.
\item Die Kanäle die das größte Potential aufweisen sollen als \gls{poc} umgesetzt werden.
\end{enumerate}


\section{Erwarteter wirtschaftlicher Nutzen}

Die in diesem Projekt konzeptionell erarbeiteten Kanäle sollen PulseShift einen Überblick geben, wie eine Umfrage an Offline-Mitarbeiter verteilt werden kann. Daraus soll abgeleitet werden können, welche Umfragekanäle im Hinblick auf Kosten und Nutzen geeignet sind, um in das Lösungsportfolio von PulseShift aufgenommen zu werden. So kann teuren Investitionen in nicht geeignete Umfragekanäle vorgebeugt werden. Gleichzeitig kann gezielt in die Umsetzung von Kanälen mit hohem Potential investiert werden. 

Die entwickelten \gls{poc}s sollen zum einen für PulseShift als Demonstration


 aufzeigen, wie die Umsetzung der jeweiligen Kanäle aussehen kann. So 

 Zum Anderen



Die in dem Projekt entwickelten  sollen die Umfragemöglichkeiten für PulseShift bei Mitarbeitern ohne Email-Adresse gewährleisten.

 
 Weiterhin sollen die PoCs mit einer möglichst hohen Akzeptanz in der zu untersuchenden Zielgruppe anwendbar sein. 
 
 Der wirtschaftliche Mehrwert liegt somit einerseits in der kostengünstigen Umsetzung der Umfrage und der einhergehenden Kosteneinsparungen, andererseits auch in den möglichst aussagekräftigen Ergebnissen, die durch eine Steigerung der Anzahl an befragten Mitarbeitern erreicht werden. 
 
 Durch diese werden dem Unternehmen Handlungsmöglichkeiten zur besseren Durchführung von Digitalisierungsmaßnahmen aufgezeigt und somit weitere Optionen zur Kosteneinsparung dargelegt.


\chapter{Projektrahmen}

\section{Stakeholder}
Für dieses Projekt existieren drei Stakeholder:
\begin{itemize}
\item \textbf{PulseShift} ist der Auftraggeber der Aufgabe des Studentenprojektes und hat somit ein direktes Interesse an einer erfolgreichen Umsetzung und nutzbaren Ergebnissen des Projektes.
\item \textbf{Prof. Dr. Holey} betreut und bewertet das Studentenprojekt und hat damit ein direktes Interesse an dem gesamten Projektverlauf, wobei der Fokus auf dem angewandtem Projektmanagement und der Projektplanung liegt.
\item Das gesamte \textbf{Projektteam} setzt das Studentenprojekt um und strebt sowohl die Abgabe zufriedenstellender Ergebnisse an PulseShift als auch eine gute Bewertung des Projektes von Herrn Holey und PulseShift an.
\end{itemize}
\section{Randbedingungen}

Hier eignen sich Stichpunkte oder eine tabelle. Folgende Themen sollen insbesondere erwähnt werden:
- medienbruch und dass dieser möglichst gar nicht / früh / selten kommen soll 
- hohe Akzeptanz bei Mitarbeitern soll gegeben sein
- Ergebnisse sollen aussagekräftig sein
- ... Ihr könnt für weitere Punkte auch das Software-Architektur Skript zur Rate ziehen
\section{Prozess}

Die Erarbeitung unserer Ideen erfolgte im Rahmen eines Design Thinking Prozesses in zwei Schritten:

\begin{enumerate}
\item Bestimmung einer Persona
\item Ideengenerierung
\end{enumerate}

Die Leitung des Design Thinking Prozesses hat Philipp übernommen. Wir haben zunächst eine Persona aufgestellt, die aus zehn Kategorien (zum Beispiel demographische Daten, Interessen und Lifestyle) bestand. Dazu hat sich jedes Teammitglied individuell Gedanken gemacht und diese auf Post-Its festgehalten (\ref{fig:design_thinking:01}). Anschließend haben wir die Ideen gesammelt und in der Gruppe diskutiert (\ref{fig:design_thinking:02}). Zur Vollendung der Persona wurden dann die unserer Meinung nach wichtigsten Merkmale herausgefiltert.
 
\begin{figure}[h]
\centering
\includegraphics[width=0.8\textwidth]{images/design_thinking/01}
\caption[Erarbeitung der Eigenschaften zur Persona]{Erarbeitung der Eigenschaften zur Persona}
\label{fig:design_thinking:01}
\end{figure}
 
\begin{figure}[h]
\centering
\includegraphics[width=0.8\textwidth]{images/design_thinking/02}
\caption[Gesammelte Eigenschaften der Persona]{Gesammelte Eigenschaften der Persona}
\label{fig:design_thinking:02}
\end{figure}
 
Zum Zweck der Ideengenerierung wurde im Anschluss ein Brainstorming durchgeführt, wobei jeder seinen Gedanken freien Lauf lassen konnte. Jeder Einfall und jede Idee wurden, egal wie abstrus sie ist, auf Post-Ist an die Tafel geklebt (\ref{fig:design_thinking:03}). Daraufhin haben wir gemeinsam Gruppen aus den bislang ungeordneten Ideen gebildet und darüber diskutiert (\ref{fig:design_thinking:04}). Auf Basis der Gruppen konnten wir dann konkrete Konzepte für die Umfrage im Unternehmen entwickeln. Die Konzepte wurden anschließend in Teams aus zwei bis drei Leuten ausgearbeitet.
 
\begin{figure}[h]
\centering
\includegraphics[width=0.8\textwidth]{images/design_thinking/03}
\caption[Brainstorming]{Brainstorming}
\label{fig:design_thinking:03}
\end{figure} 

\begin{figure}[h]
\centering
\includegraphics[width=0.8\textwidth]{images/design_thinking/04}
\caption[Gruppenbildung]{Gruppenbildung}
\label{fig:design_thinking:04}
\end{figure} 
 
\section{Projektstrukturplan}

\subsection{5. Semester}

Für das fünfte Semester wurde ein \gls{psp} erstellt. Auf Grund dessen Größe ist dieser nicht in diesem Dokument sondern im Anhang eingefügt. Dieser \gls{psp} beinhaltet die in \vref{organigramm_semester5} abgebildeten Arbeitsgruppen, deren einzelne Aufgaben während des kompletten fünften Semesters, der geschätzte und der tatsächliche Arbeitsaufwand für die einzelnen Aufgaben.

\subsection{6. Semester}

Das sechste Semester wurde gleichermaßen in einem \gls{psp} zusammengefasst, dieser ist ebenfalls im Anhang abgelegt. Enthalten in diesem \gls{psp} sind die in \vref{organigramm_semester6} abgebildeten Arbeitsgruppen und deren Pflichten gegenüber des Projektteams. Nicht enthalten sind die einzelnen Arbeitspakete der vier Arbeitsgruppen (Single Purpose App, Newsfeed App Research, Captive Portal und Dokumentation). Diese wurden jeweils intern bearbeitet.

\subsection{Kanbanboard}

Nach der Erstellung des \gls{psp} wurde daraus innerhalb der Webanwendung Trello ein Kanbanboard erstellt (siehe \ref{fig:frame:kanban}). In diesem Kanbanboard werden jedem Arbeitspaket die verantwortlichen Personen, die benötigten Dateien, der Bearbeitungszeitraum und auch der Bearbeitungsstatus zugeordnet werden. Hierdurch ist der Fortschritt des Projekts und die zu bearbeitenden Aufgaben für alle Mitglieder einsehbar. Durch die Möglichkeit, Kommentare zu einzelnen Arbeitspaketen hinzuzufügen, kann direktes Feedback für Aufgaben anderer Teammitglieder gegeben werden und die gebrauchte Arbeitszeit eingetragen werden. Die gezielte Nutzung dieser Möglichkeit vereinfachte das Projektmanagement erheblich.

\begin{figure}[H]
\centering
\includegraphics[width=1\textwidth]{images/trello}
\caption[Bildschirmabgriff des Kanbanboards in Trello]{Bildschirmabgriff des Kanbanboards in Trello}
\label{fig:frame:kanban}
\end{figure}

\section{Organigramm}

In diesem Kapitel wird die Organisationsstruktur des kompletten Projektes erläutert. Dabei stellte das fünfte Semester die Konzeptionierungsphase und das sechste Semester die Umsetzungsphase dar. Dementsprechend wurden in beiden Phasen individuelle Arbeitsgruppen gebildet.

\subsection{5. Semester}
Während des fünften Semesters befand sich das Projekt in der Konzeptionierungsphase. Für diese wurde das Projektteam in vier Teams eingeteilt. Dementsprechend haben alle Teammitglieder an der Umsetzung der Aufgaben teilgenommen, die Verantwortung für einzelne Aufgaben wurde jedoch auf diese vier Teams verteilt. Im Folgenden werden diese Teams und deren dazugehörige Aufgabe dargestellt:

\begin{description}
\item[Kommunikation mit PulseShift\\]\hfill \\
- Schnittstelle zu PulseShift\\
- Organisieren und Leiten von Meetings mit PulseShift\\
- Abgleich der Anforderungen von PulseShift mit der Ausführung


\item[Ideengenerierung und Sammlung]\hfill \\
- Generierung von Ideen z.B. durch eine Design Thinking Session\\
- Sammeln von Ideen aus dem Team\\
- Festhalten und Ausarbeitung von Ansätzen

\item[Projektmanagementtools]\hfill \\
	- Auswahl von relevanten Projektmanagementtools und Diagrammen\\
	- Erstellung von Diagrammen

\item[Projektmanagement]\hfill \\
	\textbf{Scrum Master:}\\
	\phantom{hue}- Terminieren und Organisieren von Team-Meetings\\
	\phantom{hue}- Aufstellen einer Agenda\\
	\phantom{hue}- Trello Board verwalten\\\\
	\textbf{Qualitätsmanagement}\\
	\phantom{hue}- Protokollerstellung\\
	\phantom{hue}- Sicherstellung der Qualität von Arbeitsausführung und -ergebnissen\\
	\phantom{hue}- Kommunikation mit Herr Holey
\end{description}

In \vref{organigramm_semester5} sind die zuvor beschriebenen Teams und deren dazugehörigen Mitglieder grafisch abgebildet.

\begin{figure}[h]
\centering
\includegraphics[width=12cm]{images/organigramm_semester5}
\caption{Bildliche Darstellung der Organisation während der \\ Konzeptiomsphase im 5. Semester\protect}
\label{organigramm_semester5}
\end{figure}

\subsection{6. Semester}

Im anschließenden sechsten Semesters wurde auf Basis der Konzeptionierungsphase die Umsetzungsphase initiiert und vollendet. Dabei wurden sowohl die im fünften Semester etablierten Teams beibehalten und deren Aufgaben fortgesetzt als auch neue Arbeitsgruppen gebildet, die für die Umsetzung einzelner Konzeptionen verantwortlich waren. Nachfolgend werden diese neu entstandene Arbeitsgruppen und deren entsprechenden Pflichten dargelegt:

\begin{description}
\item[Lunchapp]\hfill \\
- Entwickelt Single-Purpose-Webapp\\
- Informationen über Essenspläne werden angezeigt\\
- Push-Notifications weißen auf mögliche Umfragen hin


\item[Newsfeed App Research]\hfill \\
- Untersucht bereits vorhandene Apps im Markt\\
- Überprüft ob PulseShift diese Apps für Umfragen nutzen kann

\item[Captive Portal]\hfill \\
- Untersucht Hardwarelösungen mit Captive Portal Funktionalität\\
- Realisiert die Captive Portal Funktionalität mit genau einer Hardwarelösung\\
- Ermöglichen Weiterleitung der Teilnehmer zur Umfrage

\item[Dokumentation \& Projektmanagement]\hfill \\
- Vorgabe für Dokumentation \& Protokollierung der Umsetzung\\
- Qualitätssicherung der einzelnen Arbeitsgruppen\\
- Erstellung einer Gesamtdokumentation
\end{description}

In \vref{organigramm_semester5} ist der zuvor erläuterte Aufbau der Organisation visualisiert.

\begin{figure}[h]
\centering
\includegraphics[width=9cm]{images/organigramm_semester6}
\caption{Bildliche Darstellung der Organisation während der Umsetzungsphase im 6. Semester\protect}
\label{organigramm_semester6}
\end{figure}


\chapter{Wichtigste Ereignisse}
\label{chap:events}
\section{Kick-Off Meeting mit PulseShift - 06.10.2017}
Hierbei handelt es sich um das erste gemeinsame Treffen mit PulseShift, bei dem sich das Projektteam vorgestellt und genauere Informationen über die Arbeit von PulseShift erhalten hat. Zudem wurden mögliche Projekte erläutert, die im Rahmen des DHBW Projekts durchgeführt werden könnten. Hier bestand die Auswahl zwischen der Evaluation von Chatbots zur Umfrageerhebung, einem Dashboard, das aktuelle Technologie-Themen darstellt, und der Erstellung eines PoCs, um Mitarbeiter ohne Firmenmail zu befragen. Zudem wurden die Rahmenbedingungen des DHBW Projekts erklärt und das weitere Vorgehen festgelegt, was insbesondere die Rückmeldung einer Entscheidung für eines der möglichen Projekte einschließt.

\section{Team Planung - 06.10.2017}
Direkt im Anschluss an das Treffen mit PulseShift wurde dieses intern nachbereitet. Dabei wurde sich nochmal endgültig für PulseShift als zuverlässigen Partner und einstimmig für das Erstellen eines \gls{poc} für Umfragen an Mitarbeiter ohne Firmenmail entschieden. Für das Projekt spricht der betriebswirtschaftliche Hintergrund, das Potential der Generierung verschiedener Ideen und die Entwicklung und Evaluation verschiedener \gls{poc}s angesehen.

Weiterhin wurde eine Grobplanung des Projekts erstellt. Hier wurde für das 5. Semester die nicht-technische Ausarbeitung des Themas und für das 6. Semester die konkrete Implementierung der entwickelten Ideen festgelegt. Insbesondere ein Test der Eignung für die Endanwender ist für das 6. Semester geplant. Zudem fand auch die grundsätzliche Einteilung der Zuständigkeiten statt, die im Organigramm abgebildet ist.

Zum Abschluss wurde sich auf die zu verwendenden Tools Trello, Dropbox Paper, OneNote und Github geeinigt.

\section{Projektdefinition mit PulseShift - 12.10.2017}
Am 12.10.2017 fand ein erneutes Treffen mit PulseShift statt. Um einen höheren Endanwender-Bezug zu gewährleisten, wurde ein Gespräch mit John Deere am 15.11.2017 geplant, bei dem auch nach einer möglichen Werksbesichtigung gefragt werden soll. Alternativ zu einer Werksbesichtigung wurde empfohlen, im Bekanntenkreis nach Werks- und Wartungsmitarbeitern zu fragen, um einen besseren Eindruck von Lösungsansätzen zu erhalten.

Des Weiteren wurde der Zugriff auf ein Demosystem ermöglicht, um einen Eindruck von der Lösung von PulseShift (Umfrage-WebApp) zu erhalten.

Hinsichtlich des \gls{poc} der Umfrage für Werksmitarbeiter ohne Firmenmail wurden von PulseShift bereits einige Anregungen und Ideen mitgeteilt. Vorgeschlagen wurde etwa eine App zur Anzeige des Mittagessens in der Kantine, bei der regelmäßig Umfragen eingeblendet werden. Weiterhin soll keine App erstellt werden, die nur eine Umfrage darstellt und auch Hardware soll nicht eigens gebaut werden müssen. Insbesondere die Aspekte Kosten, Aufwand und verfügbare Ressourcen sollen bei der Ideenfindung miteinbezogen werden. Auch das Konzept von PulseShift, basierend auf unaufdringlichen Umfragen Aktionen mit Mehrwert für den Kunden zu finden, soll berücksichtigt werden.

\section{Teambesprechung - 16.10.2017}
Am 16.10.2017 wurde das letzte Treffen mit PulseShift nachbereitet. Das Treffen mit John Deere soll von Jason und Philipp wahrgenommen und als Feedbackmeeting für bis dahin ausgearbeitete Ideen genutzt werden.

Des Weiteren wurden die Aufgaben des Projektmanagements, wie die Formulierung eines konkreten Projektziels sowie das Erstellen eines Organigramms und eines Projektstrukturplans, verteilt.

Abschließend wurde eine Design Thinking Session vereinbart, um Ideen zu sammeln, und ein Treffen mit Herrn Prof. Dr. Holey arrangiert, um den aktuellen Fortschritt abzustimmen.

\section{Design Thinking - 23.10.2017}
In einem Treffen, das der Methode Design Thinking folgte, wurden Ideen für die Umsetzung der Umfrage ohne Firmenmail entwickelt. Dabei wurde zunächst eine Persona erstellt, die den typischen Endanwender des \gls{poc} darstellt. Hierdurch sollen Denkanstöße für die Ideensammlung und ein besseres Verständnis für die Situation entstehen. Im Anschluss fand die eigentliche Ideengenerierung in Form eines freien Brainstormings statt. Danach wurden die Ergebnisse gemeinsam besprochen, die Umsetzbarkeit abgeschätzt und sinnvolle Ideen zur genaueren Ausarbeitung unter den Teammitgliedern aufgeteilt (detaillierte Beschreibung siehe \vref{chapter:ideenfindung}).

\section{Treffen mit Herrn Prof. Dr. Holey - 26.10.2017}
Philipp, Sebastian und Florian haben den aktuellen Stand an Herrn Prof. Dr. Holey kommuniziert. Eine schriftliche Version wird per Mail von Sebastian an Herrn Prof. Dr. Holey weitergegeben. Herr Prof. Dr. Holey zeigte sich soweit mit dem Fortschritt des Projektes zufrieden. Abschließend wurde vereinbart, dass Herr Prof. Dr. Holey regelmäßig per Mail über Updates informiert und ein Abschlussmeeting zum Ende des 5. Semesters geplant wird.

\section{Besprechung der ausgearbeiteten Anwendungen - 03.11.2017}
Als erstes wurden drei generelle Ansätze als Umfrage App, die von jedem zuhause erarbeitet wurden, vorgestellt. Eine Newsfeed App (ähnlich zu Twitter) für Unternehmen, die aktuelle Nachrichten verteilt, Chats ermöglicht und Umfragen kaskadiert (siehe \vref{section:newsfeed_app}), eine WebApp die als Hauptinformationsquelle für Mitarbeiter dient und gleichzeitig die Teilnahme an Unternehmensumfragen ermöglicht (siehe \vref{section:lunchapp}) und abschließend eine Umfrage-App auf einem Tablet, das an stark frequentierten Orten innerhalb des Unternehmens aufgestellt werden kann (siehe \vref{section:tablets}). Des Weiteren wurde für alle drei Ansätze eine Diskussionsrunde eröffnet, in der sowohl Vorteile als auch Nachteile herausgearbeitet wurden. Abschließend wurden noch mögliche Fragen für das Treffen mit John Deere erarbeitet.

\section{Treffen mit John Deere - 15.11.2017}
Hierbei handelte es sich um ein Meeting mit drei Mitarbeitern der Organisationsabteilung von John Deere. Dabei wurden die erarbeiteten Ansätze vorgestellt, um direkt Feedback zu diesen zu erhalten. Die wichtigsten Verbesserungsvorschläge der Mitarbeiter von John Deere waren dabei, dass Belohnungen für abgeschlossene Umfragen höchstens passiv vergeben werden, Umfragen nicht erzwungen werden dürfen und ehrliche Antworten der Mitarbeiter extrem wichtig sind.

\section{Nachbearbeitung des Treffens mit John Deere - 17.11.2017}
Dieses Treffen war ein vorläufiges Abschlussmeeting für unser Projektteam. Das Meeting mit John Deere wurde besprochen und Arbeitspakete aus dem Feedback der John Deere Mitarbeiter erstellt. Weiterhin wurde festgelegt, welche finalen Schritte zum Abschluss der ersten Arbeitsphase (5. Semester) noch abgehandelt werden müssen und welche Dokumente zusammengefasst an Prof. Dr. Holey geschickt werden sollen.

\section{Abschluss des 5. Semesters und weiteres Vorgehen}
Zum Abschluss des 5. Semesters haben wir die wichtigsten Ergebnisse unserer Planungsphase zusammengefasst und diskutiert. Dabei wurden wichtige Ansätze zur Implementierung während des 6. Semesters erarbeitet. Dementsprechend ist die Arbeit des 5. Semesters als Projektstart und Projektplanung zu sehen, im 6. Semester erfolgt dann die Projektumsetzung basierend auf den Ergebnissen des 5. Semesters.

\section{Teammeeting - 22.02.2018}
Das Teammeeting diente als erneuter Kickoff des Projekts und initiierte die Umsetzungsphase. Dabei wurde festgelegt, dass alle Teammitglieder die im fünften Semester erstelle Dokumentation erneut durchlesen, um wieder mit der Thematik vertraut zu werden. Des Weiteren wurde das nächste Treffen mit PulseShift für den 08.03 festgesetzt, wobei bereits am 26.02 ein Treffen zur erneuten teaminternen Absprache erfolgt. Ferner wurde festgestellt, dass die bisherigen geschätzen Zeiten der einzelnen Arbeitspakete viel zu gering ausfielen und dementsprechend bessere Schätzungen erfolgen müssen. Abschließend wurde über die Firma \textit{iFeedback} und deren Umfragelösungen diskutiert.

\section{Diskussion der Umfragekanäle - 26.02.2018}
Um einen oder mehrere Umfragekanäle für die Entwicklung von Prototypen auszuwählen, wurden die verschiedenen Möglichkeiten in der Gruppe diskutiert. Zentrale Aspekte waren dabei, ob sich die Umsetzung hinsichtlich der Kosten und der tatsächlichen Nutzung durch die Mitarbeiter lohnt, welche Gruppenmitglieder welche Aufgaben übernehmen können und welche Technologien sich zur Umsetzung eignen.

\section{Treffen mit PulseShift - 08.03.2018}
\label{sec:events:definition_of_channels_for_poc}
Mit allen Teammitgliedern des Projektes und mehreren Vertretern von PulseShift wurde bei diesem Treffen die erarbeiteten Ideen vorgestellt. Dabei wurde von PulseShift detailliertes Feedback für die folgenden Umfragekanäle gegeben:
\begin{itemize}
\item Captive Portal
\item Newsfeed App
\item Single Purpose App
\item Tablet
\item QR-Code
\end{itemize}
Folgend wurden für den weiteren Projektverlauf von PulseShift folgende drei Ansätze am besten bewertet:
\begin{itemize}
\item Es soll eine \textbf{Captive Portal} Demo entwickelt werden.
\item Es soll eine Demo für eine \textbf{Single Purpose App} entwickelt werden.
\item Bezüglich der \textbf{Newsfeed App}  sollen die wichtigsten Lösungen am Markt analysiert werden.
\end{itemize}

\section{Nachbereitung PulseShift-Meeting und Aufgabenverteilung - 08.03.2018}
In Folge des Abstimmungsgesprächs mit PulseShift am 08.03.2018 wurden folgende Aufgabenbereiche definiert und unter den Gruppenmitgliedern aufgeteilt:
\begin{itemize}
\item Entwicklung einer Single Purpose App: Philipp, Jan, Julia(0,5)
\item Entwicklung eines Captive Portal: Henrik, Florian(0,5)
\item Evaluation/Research hinsichtlich der Newsfeed App: Jason, Julia(0,5)
\item Dokumentation und Projektmanagement der neu gebildeten Arbeitsgruppen: Sebastian, Florian(0,5)
\end{itemize}
Dieser Gruppenaufbau ist in \vref{organigramm_semester6} visualisiert.

\section{Besprechung eines Zwischenstands mit Herrn Holey - 26.03.2018}
Am 26.03 hat ein Treffen zwischen Projektteam und Herrn Holey stattgefunden. Dabei wurde der aktuelle Stand des Projektes Herrn Holey erläutert und die Zwischenergebnisse präsentiert. Von der Single Purpose App wurden mehrere \gls{epk}, ein Lastenheft, ein Pflichtenheft und alle Arbeitspakete präsentiert. Die Newsfeed App Research Arbeitsgruppe konnte erste Forschungsergebnisse und mögliche Einbindungsmöglichkeiten von Umfragen in diese Apps präsentieren. Für die Realisierung des Captive Portal wurden zwei Konzepte ausgearbeitet, allerdings wurde zu dieser Zeit auf die Bereitstellung der Hardware seitens PulseShift gewartet. Zur Dokumentation der einzelnen Arbeitsvorgänge wurde ein einheitlicher Rahmen vorgegeben und ein Latexdokument erstellt.

\section{Syncmeeting nach Arbeit an den Projekten - 10.04.2018}
Zur Präsentation der bisher geleisteten Ergebnisse, wurden am 10.04 alle Zwischenergebnisse der einzelnen Teams innerhalb des Projektteams besprochen. Die Captive Portal Funktionalität wurde auf einem Raspberry Pi implementiert, wobei noch wenige Anzeigefehler bei Samsung Handys auftreten und noch keine original Umfrage von PulseShift angezeigt wird. Die Newsfeed Forschung hat die Applikationen \textit{Staff Hub}, \textit{Xing}, \textit{linkedin} und \textit{Slack} untersucht, wobei eine genauere Analyse für \textit{Staff Hub} ausgearbeitet wurde. Die Single Purpose App wurde in Form einer Lunch App fast fertig gestellt, wobei die Umfragen von PulseShift durch ein iFrame noch nicht angezeigt werden können. Ferner wurde die Terminierung aller technischen Bearbeitungen auf den 16.04 und die Abfertigung der Dokumentation auf den 23.04 festgelegt.

\chapter{Ideenfindung}
\label{chapter:ideenfindung}
\section{Prozess}

Die Erarbeitung unserer Ideen erfolgte im Rahmen eines Design Thinking Prozesses in zwei Schritten:

\begin{enumerate}
\item Bestimmung einer Persona
\item Ideengenerierung
\end{enumerate}

Die Leitung des Design Thinking Prozesses hat Philipp übernommen. Wir haben zunächst eine Persona aufgestellt, die aus zehn Kategorien (zum Beispiel demographische Daten, Interessen und Lifestyle) bestand. Dazu hat sich jedes Teammitglied individuell Gedanken gemacht und diese auf Post-Its festgehalten (\ref{fig:design_thinking:01}). Anschließend haben wir die Ideen gesammelt und in der Gruppe diskutiert (\ref{fig:design_thinking:02}). Zur Vollendung der Persona wurden dann die unserer Meinung nach wichtigsten Merkmale herausgefiltert.
 
\begin{figure}[h]
\centering
\includegraphics[width=0.8\textwidth]{images/design_thinking/01}
\caption[Erarbeitung der Eigenschaften zur Persona]{Erarbeitung der Eigenschaften zur Persona}
\label{fig:design_thinking:01}
\end{figure}
 
\begin{figure}[h]
\centering
\includegraphics[width=0.8\textwidth]{images/design_thinking/02}
\caption[Gesammelte Eigenschaften der Persona]{Gesammelte Eigenschaften der Persona}
\label{fig:design_thinking:02}
\end{figure}
 
Zum Zweck der Ideengenerierung wurde im Anschluss ein Brainstorming durchgeführt, wobei jeder seinen Gedanken freien Lauf lassen konnte. Jeder Einfall und jede Idee wurden, egal wie abstrus sie ist, auf Post-Ist an die Tafel geklebt (\ref{fig:design_thinking:03}). Daraufhin haben wir gemeinsam Gruppen aus den bislang ungeordneten Ideen gebildet und darüber diskutiert (\ref{fig:design_thinking:04}). Auf Basis der Gruppen konnten wir dann konkrete Konzepte für die Umfrage im Unternehmen entwickeln. Die Konzepte wurden anschließend in Teams aus zwei bis drei Leuten ausgearbeitet.
 
\begin{figure}[h]
\centering
\includegraphics[width=0.8\textwidth]{images/design_thinking/03}
\caption[Brainstorming]{Brainstorming}
\label{fig:design_thinking:03}
\end{figure} 

\begin{figure}[h]
\centering
\includegraphics[width=0.8\textwidth]{images/design_thinking/04}
\caption[Gruppenbildung]{Gruppenbildung}
\label{fig:design_thinking:04}
\end{figure} 
 
\section{Persona}
\textbf{Bernd Bandarbeiter}
\begin{itemize}
\item Geschlecht: männlich
\item Alter: 50 Jahre
\item Familienstand: verheiratet, 2 Kinder
\item Einkommen: 2500€ Brutto/Monat
\item Gesellschaftlicher Stand: untere Mittelschicht
\item Job: Schichtarbeit am Band, 8-Stunden-Schichten
\item Lifestyle und Hobbys: fußballinteressiert, Alkohol, Raucher, Glücksspiel, lebt in den Tag
\item Motivation: Geld verdienen, Familie ernähren, Akzeptanz im direkten Umfeld
\item Affinität im digitalen Bereich: Smartphone, (älterer) Computer mit Internet (Mails, YouTube…), technisch nicht versiert
\item Informationsquellen: Herrensitzung, Kneipe, RTL, Bildzeitung
\item Herausforderungen und Ängste: Familie ernähren, Job behalten, außerordentliche Rechnungen bezahlen, Ansehensverlust, Krankheiten und Verletzungen
\end{itemize}

\section{Belohnungssysteme}

\subsection{Gründe für ein Belohnungssystem}
Basierend auf der Persona, die wir erstellt haben, gehen wir davon aus, dass der typische Bandarbeiter im Unternehmen sehr gestresst ist und in erster Linie darauf bedacht ist, seine Arbeit zu machen und Geld zu verdienen, um sich und seine Familie zu ernähren. Sein Interesse, an einer Unternehmensumfrage teilzunehmen, ist dementsprechend gering. Um einen sogenannten Offline-Mitarbeiter dennoch zur Teilnahme an einer Umfrage zu motivieren, halten wir ein Belohnungssystem für unabdingbar. Deshalb haben wir uns eine Reihe von Belohnungssystemen überlegt.

\subsection{Mögliche Belohnungssysteme}
Mögliche Anreize für den Mitarbeiter könnten sein:
\begin{itemize} 
\item Gratisfußballwette
\item Kostenlose Bildzeitung
\item Tipico-Guthaben
\item Ostereiersuche/Adventskalender
\item Jukebox (Mitarbeiter darf sich ein Lied wünschen)
\item Sammelobjekte (Fußballsammelbildchen, Sammelfiguren…)
\item Witze 
\item Firmenevent
\item Nach der Arbeit Interview/Kneipe/Bierabend
\item Gewinnspiel
\item Gratissnack
\item Gratis-Getränk
\item Adventskalender
\end{itemize}

\subsection{Vor- und Nachteile einer Umsetzung als Gewinnspiel}
Nach einer Abwägung der Möglichkeiten denken wir, dass ein Gewinnspiel die beste Lösung darstellt. Der Vorteil hierbei ist, dass ein Anreiz für eine Vielzahl an Mitarbeitern geschaffen werden kann, ohne zu hohe Aufwände und Kosten zu verursachen, da nur ein kleiner Teil der Mitarbeiter tatsächlich eine Belohnung erhält. 

Problematisch ist dabei jedoch, dass genau dadurch auch der Anreiz geringer sein könnte als bei anderen Belohnungssystemen. Ein weiterer Diskussionspunkt ist die Frage nach dem Geld: Ist das Unternehmen bereit, die Kosten zu übernehmen? Zudem wird Qualität gegen Quantität eingetauscht, da der Reiz der Belohnung dazu verführen kann, die Fragen nicht mehr gewissenhaft zu beantworten, sondern nur die Belohnung kassieren zu wollen. Da PulseShift nach eigener Aussage qualitativ hochwertige Antworten bevorzugt und auch John Deere die Finanzierung und Vergabe von Belohnungen in einem Abstimmungsmeeting ablehnte, haben wir uns generell gegen die Umsetzung eines solchen Anreizsystems entschieden.

\chapter{Lösungsportfolio}
\section{Zettelumfrage}

\subsection{Beschreibung}
Hier werden einfache Papierzettel als Umfragemedium genutzt. Auf diesen stehen spezifische Fragen für einzelne Abteilungen oder Zielgruppen eines Unternehmens. Dabei können verschiedene Umfragebögen für diverse Abteilungen oder Gruppen erstellt und bei diesen explizit ausgelegt werden. Bei der Auslage der Zettel ist darauf zu achten, dass diese immer ausschließlich für die gewählte Zielgruppe erreichbar, jedoch gleichzeitig gut zugänglich sind. Die abgedruckten Fragen können direkt von der bestehenden PulseShift-Applikation entnommen werden. Durch diese Vorgehensweise ist eine Authentisierung der Mitarbeiter nicht mehr notwendig, da nur die gewünschten Zielpersonen Zugang zu den entsprechenden Zetteln haben. Fraglich ist, wie die Mitarbeiter dazu motiviert werden, solche Zettel auszufüllen. Ein weiteres Problem ergibt sich bei der Auswertung der Zettel. So müsste entweder ein spezielles Programm zur Auswertung geschrieben werden oder eine händische Auswertung erfolgen. Beide Optionen erweisen sich als kostenintensiv und sind mit einem hohen Aufwand verbunden.

\subsection{Mockups}

\begin{figure}[H]
\centering
\includegraphics[width=0.9\textwidth]{images/portfolio/paper_survey}
\caption[Mockup Zettelumfrage]{Mockup Zettelumfrage}
\label{fig:portfolio:paper_survey}
\end{figure}

\subsection{Kostenfaktoren}

\begin{itemize}
\item Papier (z.B. Amazon Versando:	6 € / 500 Stück)
\item Tinte (z.B. Catridge Duck Black: 50 € / 500 Stück)
\end{itemize}

Hierbei handelt es sich nur um die Materialkosten für die gedruckten Zettel. Zusätzlich zu beachten sind die Kosten, die bei der Auswertung der Zettel anfallen. Solche sind beispielsweise Kosten zur Erstellung einer Auswertungs-Software oder Kosten für die manuelle Bearbeitung der Zettel. Diese sind aber sehr schwer zu kalkulieren und werden daher vorerst nicht geschätzt.

\subsection{Beurteilung des Projektteams}
\subsubsection{Vorteile}

\begin{itemize}
\item Keine IT-Implementierung seitens des Unternehmens benötigt
\item Schnelle und einfache Beantwortung der Zettel durch Mitarbeiter
\item Geringe Kosten und einfacher Prozess für die Erstellung der Zettel
\item Kein Authentisierungsprozess benötigt
\end{itemize}

\subsubsection{Nachteile}

\begin{itemize}
\item Aufwändiger Auswertungsprozess
\item Hohe Kosten bei der Auswertung der Zettel
\item Zielgruppen müssen lokal von anderen Gruppen trennbar sein $\rightarrow$ Wo werden die Zettel ausgelegt?
\item Zettel müssen gedruckt werden
\item Die entsprechenden Fragen müssen händisch in ein Dokument eingetragen werden
\end{itemize}

\subsubsection{Bewertung und Potential}
Für die Mitarbeiter ist die Umfrage schnell und einfach durchzuführen. Es müssen keine weiteren Applikationen installiert oder sonstige technische Voraussetzungen geschaffen werden. Allerdings erfordert die Umfrage einen hohen Arbeitsaufwand in der Vorbereitung und Auswertung. Die geringen Kosten für die Zettel selbst sind jedoch zu vernachlässigen. Insbesondere durch den Medienbruch in der Auswertung und durch den fehlenden Anreiz für die Mitarbeiter stellen die Zettel keinen geeigneten Umfragekanal dar.

\subsection{Feedback und Beurteilung durch PulseShift}

PulseShift sieht die Umsetzung einer Zettelumfrage nicht als sinnvoll an, da der Aufwand zur Auswertung der Zettel zu hoch ist. Außerdem wird der Kanal Zettelumfrage nicht als innovativ genug betrachtet, um sich von Wettbewerbern abzugrenzen.

\subsection{Weiteres Vorgehen}

Der Kanal Zettelumfrage wird sowohl vom Projektteam als auch von PulseShift kritisch betrachtet. Er wird nicht weiter verfolgt.

\section{Tablets}
\label{section:tablets}
\section{Single-Purpose-Webapp: Lunchapp}
\label{section:lunchapp}

\subsection{Beschreibung}

%- Was ist die Idee einer single purpose app -> bitte mit glossareintrag für single purpose app (bei fragen an sebbel wenden)
%- Das es ne Lunchapp wird erst unter Weiteres Vorgehen Beschreiben
%- siehe 5. semester word dokument

Bei diesem Prototyp handelt es sich um die Ausarbeitung einer \gls{glos:single_purpose_web_app}. Diese soll einen relevanten Inhalt, wie den Schichtplan oder das Lunchmenü, für die Mitarbeiter bereithalten. Somit ist ein Anreiz für die Mitarbeiter gegeben, diese Webseite aufzurufen. Auf dieser Seite soll ein Banner zu der Umfrage führen. Dieses soll dabei schlicht und nicht zu aufdringlich wirken, aber trotzdem die Aufmerksamkeit des Nutzers wecken. Klickt man auf das Banner, so gelangt man zur Umfrage. Um die relevanten Informationen zu schützen, könnte man einen Authentifizierungsmechanismus einrichten. Damit wird bezweckt, dass Mitarbeiter nur die für sie bestimmten Informationen, wie den Schichtplan, einsehen können. Durch diese Authentifizierung braucht sich der Mitarbeiter auch bei der Umfrage nicht mehr zusätzlich anmelden, sondern wird direkt vom System eingeordnet.

\subsection{Mockups}

\ref{wamu1} bis \ref{wamu3} zeigen drei verschiedene Design-Mockups, die wir für die Single-Purpose-Webapp entwickelt haben:

\begin{figure}[H] 
\centering 
\includegraphics[scale=0.72]{images/lunchapp_mockups/mockup1} 
\caption[Mockup: Webapp]{Mockup: Webapp} 
\label{wamu1} 
\end{figure}

\begin{figure}[H] 
\centering 
\includegraphics[scale=0.3]{images/lunchapp_mockups/mockup3} 
\caption[Mockup: Webapp mit Popup]{Mockup: Webapp mit Popup} 
\label{wamu2} 
\end{figure}

\begin{figure}[H] 
\centering 
\includegraphics[scale=0.3]{images/lunchapp_mockups/mockup2} 
\caption[Mockup: Webapp mit alternativem Popup]{Mockup: Webapp mit alternativem Popup} 
\label{wamu3} 
\end{figure}



\subsection{Kostenfaktoren}

\paragraph{Initiale Kosten}
Bei der Erstellung der Webanwendung fallen einmalige Kosten an. Diese können unterschiedlich hoch ausfallen, je nachdem von wem sie durchgeführt wird. Ein wichtiger Aspekt bei der Erstellung ist die Implementierung eines sicheren Authentifizierungsmechanismus, um sowohl ausreichende Datensicherheit als auch Datenschutz zu gewährleisten. Die Kosten könnten daher zwischen 350 € und 1200 € liegen.

\paragraph{Regelmäßige Kosten}
Die Webapp muss fortlaufend mit den neusten Informationen versorgt werden und auf dem aktuellen Stand der Technik bleiben. Somit müsste es einen Mitarbeiter geben, der für die Instandhaltung und Aktualisierung der Seite verantwortlich ist. Dieser müsste schätzungsweise 8 Stunden pro Monat mit dem Aktualisieren der App verbringen, wodurch mit Kosten von mindestens 80 Euro pro Monat zu rechnen ist.

\subsection{Beurteilung des Projektteams}

\paragraph{Vorteile}
\begin{itemize} 
\item Der initiale Anreiz, die Webseite zu besuchen, wird durch die relevanten Informationen gegeben.
\item Es wird kein Druck auf die Mitarbeiter ausgeübt, sodass sie die Umfrage aus eigner Entscheidung starten. Somit sind die Umfrageantworten realistisch und qualitativ hochwertig.
\item Ist beispielsweise der Schichtplan die relevante Information und der Mitarbeiter ist mit seinem nicht zufrieden, könnte ihn das noch mehr anregen, an der Umfrage teilzunehmen, um etwas zu verbessern.
\end{itemize}


\paragraph{Nachteile}
\begin{itemize}
\item Es kann durchaus sein, dass die Mitarbeiter die Umfrage nicht nutzen, da sie keine Lust oder Zeit haben und nur die Informationen auf der Webseite einsehen wollen.
\item Dauerhafter administrativer Aufwand, da die Seite gut geschützt werden muss und die Informationen immer wieder aktualisiert werden müssen.
\end{itemize}


\paragraph{Bewertung und Potential}%$~~$\\
Das Kosten-Leistungs-Verhältnis ist hier in einem angemessenen Rahmen, daher sollte dieser Kanal weiterverfolgt werden. Die einmalige Implementierung erzeugt zwar erst einmal hohe Kosten, jedoch sind die laufenden Kosten relativ gering und können durch den Vorteil, die die Umfragen bringen können, aufgewogen werden.

\subsection{Feedback und Beurteilung durch PulseShift}

Von Seiten PulseShifts stieß die Single-Purpose-Webapp auf äußerst positives Feedback. Das Unternehmen erachtete für den Anfang eine hybride App oder \gls{pwa} als sinnvoll. Der Vorteil einer solchen Lösung ist, dass plattformübergreifend Push-Benachrichtigungen unterstützt werden, beispielsweise dann, wenn dem Mitarbeiter eine neue Umfrage zur Verfügung steht. Bezüglich der konkreten Technologie gibt es seitens PulseShift keine Vorgaben. Jedoch soll die Umfrage, beispielsweise in einem iFrame, in die App gerendert werden können.


\subsection{Weiteres Vorgehen}

Für die Umsetzung haben wir uns zwei alternative Single-Purposes überlegt, die die Applikation dem Mitarbeiter bieten soll:

\begin{itemize}
\item Anzeige des Schichtplans eines Mitarbeiters
\item Anzeige des Lunchmenüs für verschiedene Kantinen und Wochentage
\end{itemize}

Beides bietet unserer Ansicht nach dem Mitarbeiter genügend Anreiz, die Anwendung regelmäßig zu nutzen und dabei gelegentlich an einer Umfrage teilzunehmen. Letztendlich haben wir uns für die Anzeige des Lunchmenüs entschieden, da viele Unternehmen die Schichtpläne ihrer Mitarbeiter aus Datenschutzgründen nicht an PulseShift weitergeben dürfen. Die Realisation ist in \vref{section:realisation:lunchapp} beschrieben.
\section{Captive Portal}


\subsection{Beschreibung}
Ein sogenanntes Captive Portal wird vor allem in öffentlichen Bereichen eingesetzt, in denen Zugang zu einem WLAN-Netzwerk gewährt wird. Der Nutzer wird nach dem Verbindungsaufbau automatisch auf eine Webseite geleitet, auf der er z.B. Richtlinien akzeptieren muss. Dies können wir uns zu Nutze machen. Bietet das Unternehmen WLAN an, können sich die Mitarbeiter mit ihrem Smartphone verbinden und werden anschließend durch das Captive Portal zu den Fragen von PulseShift geleitet. Ein anderer Anwendungsfall ist der Einsatz bei einer Gesamtveranstaltung. Dabei werden die Router am Veranstaltungsort platziert und die Mitarbeiter anschließend gebeten, sich dort anzumelden und die Umfrage durchzuführen.

\subsection{Bewertung des Projektteams}
Das Potential liegt unserer Ansicht nach vor allem bei dem Einsatz während Gesamtveranstaltungen. Dadurch würden Zettel für Umfragen nicht mehr benötigt werden, jedoch könnten solche Umfragen auch nicht sehr häufig erfolgen. Der Einsatz in Kantinen oder Pausenräumen bietet sich auch an und ist mit geringem Aufwand umsetzbar, allerdings ist nach unseren Erkenntnissen die Motivation der Mitarbeiter äußerst gering, die Umfrage während der Pause zu machen. Zusätzlich gehen maximal 10\% der Werksmitarbeiter überhaupt zum Mittagessen in die Kantine.
Wir halten das Captive Portal für eine sinnvolle Ergänzung. Es kann aber aus oben genannten Gründen nicht als einziges Instrument zur Durchführung von Umfragen eingesetzt werden.

\subsection{Feedback und Bewertung durch PulseShift}

siehe Protokoll 08.03.2018 - Meeting PulseShift

\subsection{Weiteres Vorgehen}
Nach dem Feedback von PulseShift hat sich das Projektteam dazu entschieden, eine Captive Portal Demo für genau eine Hardwarelösung zu realisieren. Zur Umsetzung wurde als Hardware ein Raspberry Pi festgelegt, da sich dieser als kostengünstigste und transportfähigste Alternative für die Implementierung eines Captive Portals erwiesen hat. Dabei gilt es zu beachten, dass unabhängig von den genutzten Endgeräten, eine Weiterleitung des Nutzers zu den PulseShift Umfragen durch das Captive Portal stattfinden muss. Die Realisation der Implementierung des Captive Portals auf einem Raspberry Pi, unter Beachtung der zuvor geschilderten Randbedingungen, wird in \vref{section:realisation:captive_portal} behandelt.

\section{Newsfeed App}
\label{section:newsfeed_app}

\subsection{Beschreibung}

Die native Newsfeed-App ist inspiriert von Twitter. Hier sollen verschiedene News des Unternehmens, des Standorts, in dem der Mitarbeiter arbeitet sowie dessen eigener Abteilung angezeigt werden. In diese Newsfeed-App soll die Umfrage automatisch eingebettet werden, entweder als Banner an der Seite oder als eigener Eintrag im Feed. So können die Mitarbeiter die aktuellen News einsehen und aus eigener Initiative die Umfrage starten.

\subsection{Mockup}

\begin{figure}[H] 
\centering 
\includegraphics[scale=0.72]{images/5mockup} 
\caption[Mockup: Native Newsfeed-App]{Mockup: Native Newsfeed-App\protect} 
\label{ws} 
\end{figure}

\subsection{Kostenkalkulation}

\subsubsection{Initiale Kosten}

Einmalige Kosten fallen bei der Erstellung der nativen Anwendung an. Diese soll so implementiert werden, dass sie auf allen Endgeräten, insbesondere auf dem Smartphone, genutzt werden kann. Dabei ist es auch hier ein Authentifizierungsmechanismus entscheidend, sodass der Nutzer nur die für ihn bestimmten News erhält.
Schätzung: zwischen 350€ und 1200€.

\subsubsection{Regelmäßige Kosten}

Die native Anwendung muss möglichst jeden Tag bzw. jede Woche um neue News ergänzt werden, sodass der Nutzer auch den Ansporn hat, sie sich anzuschauen.

\begin{figure}[H] 
\centering 
\includegraphics[scale=0.82]{images/5kosten} 
\caption[Kostenübersicht]{Kostenübersicht\protect} 
\label{ws} 
\end{figure}

\subsection{Beurteilung des Projektteams}

\paragraph{Vorteile}

\begin{itemize}
\item	Mitarbeiter erhalten News über das Unternehmen, ggf. bessere Corporate Identity.
\item Die Umfrage ist direkt integriert und somit nicht zu aufdringlich.
\item Die News können die Mitarbeiter anregen, an den Umfragen teilzunehmen, da sie vielleicht positiven oder negativen Inhalt erhalten, zu dem sich der Mitarbeiter äußern möchte
\end{itemize}

\paragraph{Nachteile}

\begin{itemize}
\item Die Nutzung der Newsfeed App ist fragwürdig, da es wahrscheinlich nur wenige Mitarbeiter gibt, die sich außerhalb der Arbeitszeit damit beschäftigen wollen.
\item Der durchschnittliche Werksmitarbeiter ist womöglich technisch nicht so interessiert, dass er sich eine solche Newsfeed-App herunterladen möchte.
\item Der Kostenaufwand, um die News möglichst aktuell zu halten, ist relativ hoch.
\end{itemize}

\paragraph{Bewertung und Potential}

Die Newsfeed-App wäre gegebenenfalls für technisch interessierte und engagierte Mitarbeiter geeignet. Da dies aber auf einen Großteil der Werksmitarbeiter nicht zutrifft und auch die Möglichkeiten, viele interessante News zu verfassen, relativ gering sind, schätzen wir das Potential einer Newsfeed-App für diesen Use-Case als gering ein. Auch ist der Aufwand, explizit News zu verfassen, unverhältnismäßig groß im Vergleich zu dem, wofür die App eigentlich dienen soll, nämlich dem Anregen, an einer Umfrage teilzunehmen.

\subsection{Feedback und Beurteilung durch PulseShift}

Die Newsfeed App wird  insgesamt von PulseShift positiv bewertet, aber wahrscheinlich vom Kunden kritisch, da diese Content pflegen müssen. Hier gibt es bereits Lösungen von anderen Unternehmen (z.B. MS Staffhub). Es ist sinnvoller, diese zu analysieren und zu evaluieren anstelle eine eigene Anwendung zu entwickeln. 

Folgende Aspekte sollten bei einer Analyse auf jeden Fall berücksichtigt werden: 

\begin{itemize}
\item Flexibilität und Freiraum
\item Anreize zur Nutzung des Bandarbeiters
\item API die durch PulseShift angesprochen werden kann
\item Umfragefunktionalitäten
\item Referenzkunden
\end{itemize}

\subsection{Weiteres Vorgehen}

Es sollen die wichtigsten Lösungen am Markt insbesondere nach den oben genannten Kriterien analysiert und evaluiert werden. Diese Herangehensweise findet in Kapitel 6.3 statt.



\chapter{Realisierung vielversprechender Umfragekanäle}
\label{chapter:realisation}
\section{Single-Purpose-Webapp: Lunchapp}
\label{section:lunchapp}

\subsection{Beschreibung}

%- Was ist die Idee einer single purpose app -> bitte mit glossareintrag für single purpose app (bei fragen an sebbel wenden)
%- Das es ne Lunchapp wird erst unter Weiteres Vorgehen Beschreiben
%- siehe 5. semester word dokument

Bei diesem Prototyp handelt es sich um die Ausarbeitung einer \gls{glos:single_purpose_web_app}. Diese soll einen relevanten Inhalt, wie den Schichtplan oder das Lunchmenü, für die Mitarbeiter bereithalten. Somit ist ein Anreiz für die Mitarbeiter gegeben, diese Webseite aufzurufen. Auf dieser Seite soll ein Banner zu der Umfrage führen. Dieses soll dabei schlicht und nicht zu aufdringlich wirken, aber trotzdem die Aufmerksamkeit des Nutzers wecken. Klickt man auf das Banner, so gelangt man zur Umfrage. Um die relevanten Informationen zu schützen, könnte man einen Authentifizierungsmechanismus einrichten. Damit wird bezweckt, dass Mitarbeiter nur die für sie bestimmten Informationen, wie den Schichtplan, einsehen können. Durch diese Authentifizierung braucht sich der Mitarbeiter auch bei der Umfrage nicht mehr zusätzlich anmelden, sondern wird direkt vom System eingeordnet.

\subsection{Mockups}

\ref{wamu1} bis \ref{wamu3} zeigen drei verschiedene Design-Mockups, die wir für die Single-Purpose-Webapp entwickelt haben:

\begin{figure}[H] 
\centering 
\includegraphics[scale=0.72]{images/lunchapp_mockups/mockup1} 
\caption[Mockup: Webapp]{Mockup: Webapp} 
\label{wamu1} 
\end{figure}

\begin{figure}[H] 
\centering 
\includegraphics[scale=0.3]{images/lunchapp_mockups/mockup3} 
\caption[Mockup: Webapp mit Popup]{Mockup: Webapp mit Popup} 
\label{wamu2} 
\end{figure}

\begin{figure}[H] 
\centering 
\includegraphics[scale=0.3]{images/lunchapp_mockups/mockup2} 
\caption[Mockup: Webapp mit alternativem Popup]{Mockup: Webapp mit alternativem Popup} 
\label{wamu3} 
\end{figure}



\subsection{Kostenfaktoren}

\paragraph{Initiale Kosten}
Bei der Erstellung der Webanwendung fallen einmalige Kosten an. Diese können unterschiedlich hoch ausfallen, je nachdem von wem sie durchgeführt wird. Ein wichtiger Aspekt bei der Erstellung ist die Implementierung eines sicheren Authentifizierungsmechanismus, um sowohl ausreichende Datensicherheit als auch Datenschutz zu gewährleisten. Die Kosten könnten daher zwischen 350 € und 1200 € liegen.

\paragraph{Regelmäßige Kosten}
Die Webapp muss fortlaufend mit den neusten Informationen versorgt werden und auf dem aktuellen Stand der Technik bleiben. Somit müsste es einen Mitarbeiter geben, der für die Instandhaltung und Aktualisierung der Seite verantwortlich ist. Dieser müsste schätzungsweise 8 Stunden pro Monat mit dem Aktualisieren der App verbringen, wodurch mit Kosten von mindestens 80 Euro pro Monat zu rechnen ist.

\subsection{Beurteilung des Projektteams}

\paragraph{Vorteile}
\begin{itemize} 
\item Der initiale Anreiz, die Webseite zu besuchen, wird durch die relevanten Informationen gegeben.
\item Es wird kein Druck auf die Mitarbeiter ausgeübt, sodass sie die Umfrage aus eigner Entscheidung starten. Somit sind die Umfrageantworten realistisch und qualitativ hochwertig.
\item Ist beispielsweise der Schichtplan die relevante Information und der Mitarbeiter ist mit seinem nicht zufrieden, könnte ihn das noch mehr anregen, an der Umfrage teilzunehmen, um etwas zu verbessern.
\end{itemize}


\paragraph{Nachteile}
\begin{itemize}
\item Es kann durchaus sein, dass die Mitarbeiter die Umfrage nicht nutzen, da sie keine Lust oder Zeit haben und nur die Informationen auf der Webseite einsehen wollen.
\item Dauerhafter administrativer Aufwand, da die Seite gut geschützt werden muss und die Informationen immer wieder aktualisiert werden müssen.
\end{itemize}


\paragraph{Bewertung und Potential}%$~~$\\
Das Kosten-Leistungs-Verhältnis ist hier in einem angemessenen Rahmen, daher sollte dieser Kanal weiterverfolgt werden. Die einmalige Implementierung erzeugt zwar erst einmal hohe Kosten, jedoch sind die laufenden Kosten relativ gering und können durch den Vorteil, die die Umfragen bringen können, aufgewogen werden.

\subsection{Feedback und Beurteilung durch PulseShift}

Von Seiten PulseShifts stieß die Single-Purpose-Webapp auf äußerst positives Feedback. Das Unternehmen erachtete für den Anfang eine hybride App oder \gls{pwa} als sinnvoll. Der Vorteil einer solchen Lösung ist, dass plattformübergreifend Push-Benachrichtigungen unterstützt werden, beispielsweise dann, wenn dem Mitarbeiter eine neue Umfrage zur Verfügung steht. Bezüglich der konkreten Technologie gibt es seitens PulseShift keine Vorgaben. Jedoch soll die Umfrage, beispielsweise in einem iFrame, in die App gerendert werden können.


\subsection{Weiteres Vorgehen}

Für die Umsetzung haben wir uns zwei alternative Single-Purposes überlegt, die die Applikation dem Mitarbeiter bieten soll:

\begin{itemize}
\item Anzeige des Schichtplans eines Mitarbeiters
\item Anzeige des Lunchmenüs für verschiedene Kantinen und Wochentage
\end{itemize}

Beides bietet unserer Ansicht nach dem Mitarbeiter genügend Anreiz, die Anwendung regelmäßig zu nutzen und dabei gelegentlich an einer Umfrage teilzunehmen. Letztendlich haben wir uns für die Anzeige des Lunchmenüs entschieden, da viele Unternehmen die Schichtpläne ihrer Mitarbeiter aus Datenschutzgründen nicht an PulseShift weitergeben dürfen. Die Realisation ist in \vref{section:realisation:lunchapp} beschrieben.
\section{Captive Portal}


\subsection{Beschreibung}
Ein sogenanntes Captive Portal wird vor allem in öffentlichen Bereichen eingesetzt, in denen Zugang zu einem WLAN-Netzwerk gewährt wird. Der Nutzer wird nach dem Verbindungsaufbau automatisch auf eine Webseite geleitet, auf der er z.B. Richtlinien akzeptieren muss. Dies können wir uns zu Nutze machen. Bietet das Unternehmen WLAN an, können sich die Mitarbeiter mit ihrem Smartphone verbinden und werden anschließend durch das Captive Portal zu den Fragen von PulseShift geleitet. Ein anderer Anwendungsfall ist der Einsatz bei einer Gesamtveranstaltung. Dabei werden die Router am Veranstaltungsort platziert und die Mitarbeiter anschließend gebeten, sich dort anzumelden und die Umfrage durchzuführen.

\subsection{Bewertung des Projektteams}
Das Potential liegt unserer Ansicht nach vor allem bei dem Einsatz während Gesamtveranstaltungen. Dadurch würden Zettel für Umfragen nicht mehr benötigt werden, jedoch könnten solche Umfragen auch nicht sehr häufig erfolgen. Der Einsatz in Kantinen oder Pausenräumen bietet sich auch an und ist mit geringem Aufwand umsetzbar, allerdings ist nach unseren Erkenntnissen die Motivation der Mitarbeiter äußerst gering, die Umfrage während der Pause zu machen. Zusätzlich gehen maximal 10\% der Werksmitarbeiter überhaupt zum Mittagessen in die Kantine.
Wir halten das Captive Portal für eine sinnvolle Ergänzung. Es kann aber aus oben genannten Gründen nicht als einziges Instrument zur Durchführung von Umfragen eingesetzt werden.

\subsection{Feedback und Bewertung durch PulseShift}

siehe Protokoll 08.03.2018 - Meeting PulseShift

\subsection{Weiteres Vorgehen}
Nach dem Feedback von PulseShift hat sich das Projektteam dazu entschieden, eine Captive Portal Demo für genau eine Hardwarelösung zu realisieren. Zur Umsetzung wurde als Hardware ein Raspberry Pi festgelegt, da sich dieser als kostengünstigste und transportfähigste Alternative für die Implementierung eines Captive Portals erwiesen hat. Dabei gilt es zu beachten, dass unabhängig von den genutzten Endgeräten, eine Weiterleitung des Nutzers zu den PulseShift Umfragen durch das Captive Portal stattfinden muss. Die Realisation der Implementierung des Captive Portals auf einem Raspberry Pi, unter Beachtung der zuvor geschilderten Randbedingungen, wird in \vref{section:realisation:captive_portal} behandelt.

\section{Newsfeed App}
\label{section:newsfeed_app}

\subsection{Beschreibung}

Die native Newsfeed-App ist inspiriert von Twitter. Hier sollen verschiedene News des Unternehmens, des Standorts, in dem der Mitarbeiter arbeitet sowie dessen eigener Abteilung angezeigt werden. In diese Newsfeed-App soll die Umfrage automatisch eingebettet werden, entweder als Banner an der Seite oder als eigener Eintrag im Feed. So können die Mitarbeiter die aktuellen News einsehen und aus eigener Initiative die Umfrage starten.

\subsection{Mockup}

\begin{figure}[H] 
\centering 
\includegraphics[scale=0.72]{images/5mockup} 
\caption[Mockup: Native Newsfeed-App]{Mockup: Native Newsfeed-App\protect} 
\label{ws} 
\end{figure}

\subsection{Kostenkalkulation}

\subsubsection{Initiale Kosten}

Einmalige Kosten fallen bei der Erstellung der nativen Anwendung an. Diese soll so implementiert werden, dass sie auf allen Endgeräten, insbesondere auf dem Smartphone, genutzt werden kann. Dabei ist es auch hier ein Authentifizierungsmechanismus entscheidend, sodass der Nutzer nur die für ihn bestimmten News erhält.
Schätzung: zwischen 350€ und 1200€.

\subsubsection{Regelmäßige Kosten}

Die native Anwendung muss möglichst jeden Tag bzw. jede Woche um neue News ergänzt werden, sodass der Nutzer auch den Ansporn hat, sie sich anzuschauen.

\begin{figure}[H] 
\centering 
\includegraphics[scale=0.82]{images/5kosten} 
\caption[Kostenübersicht]{Kostenübersicht\protect} 
\label{ws} 
\end{figure}

\subsection{Beurteilung des Projektteams}

\paragraph{Vorteile}

\begin{itemize}
\item	Mitarbeiter erhalten News über das Unternehmen, ggf. bessere Corporate Identity.
\item Die Umfrage ist direkt integriert und somit nicht zu aufdringlich.
\item Die News können die Mitarbeiter anregen, an den Umfragen teilzunehmen, da sie vielleicht positiven oder negativen Inhalt erhalten, zu dem sich der Mitarbeiter äußern möchte
\end{itemize}

\paragraph{Nachteile}

\begin{itemize}
\item Die Nutzung der Newsfeed App ist fragwürdig, da es wahrscheinlich nur wenige Mitarbeiter gibt, die sich außerhalb der Arbeitszeit damit beschäftigen wollen.
\item Der durchschnittliche Werksmitarbeiter ist womöglich technisch nicht so interessiert, dass er sich eine solche Newsfeed-App herunterladen möchte.
\item Der Kostenaufwand, um die News möglichst aktuell zu halten, ist relativ hoch.
\end{itemize}

\paragraph{Bewertung und Potential}

Die Newsfeed-App wäre gegebenenfalls für technisch interessierte und engagierte Mitarbeiter geeignet. Da dies aber auf einen Großteil der Werksmitarbeiter nicht zutrifft und auch die Möglichkeiten, viele interessante News zu verfassen, relativ gering sind, schätzen wir das Potential einer Newsfeed-App für diesen Use-Case als gering ein. Auch ist der Aufwand, explizit News zu verfassen, unverhältnismäßig groß im Vergleich zu dem, wofür die App eigentlich dienen soll, nämlich dem Anregen, an einer Umfrage teilzunehmen.

\subsection{Feedback und Beurteilung durch PulseShift}

Die Newsfeed App wird  insgesamt von PulseShift positiv bewertet, aber wahrscheinlich vom Kunden kritisch, da diese Content pflegen müssen. Hier gibt es bereits Lösungen von anderen Unternehmen (z.B. MS Staffhub). Es ist sinnvoller, diese zu analysieren und zu evaluieren anstelle eine eigene Anwendung zu entwickeln. 

Folgende Aspekte sollten bei einer Analyse auf jeden Fall berücksichtigt werden: 

\begin{itemize}
\item Flexibilität und Freiraum
\item Anreize zur Nutzung des Bandarbeiters
\item API die durch PulseShift angesprochen werden kann
\item Umfragefunktionalitäten
\item Referenzkunden
\end{itemize}

\subsection{Weiteres Vorgehen}

Es sollen die wichtigsten Lösungen am Markt insbesondere nach den oben genannten Kriterien analysiert und evaluiert werden. Diese Herangehensweise findet in Kapitel 6.3 statt.


\chapter{Abschluss}

\section{Zusammenfassung}

- Was wurde alles gemacht

\section{Zielerreichung}

- Erreichung der Ziele auswerten auf basis von 1.2 und Randbedingungen berücksichtigen

\section{Ausblick}

Was kann pulseshift jetzt mit dem ganzen kladderradatsch anfangen?


% Anhang der Arbeit
% 
%
\seAppendix{}


%\chapter{Einige wichtige \LaTeX{}-Kommandos}

%  Testdatei f\"ur die Erzeugung von Literaturreferenzen, die den Regeln von Rene Theisen 
%  (Wissenschaftliches Arbeiten, 2009) folgen
%
%
%
\section{Kommandos f\"ur die Erzeugung von Literaturverweisen}

\subsection{Kurzzitierweise mit der Angabe eines Kurztitels}

Das Kommando \verb+\seCite{par1}{par2}{par3}+ erzeugt einen Literaturverweis im Text. 

\begin{seToplist}{\texttt{par1}:}
\item[\texttt{par1}:] Der erste Parameter  definiert einen optionalen Text, der vor dem eigentlichen Literaturverweis ausgegeben 
                               wird, typischerweise Vgl. oder vgl.
\item[\texttt{par2}:] Der zweite Parameter  wird verwendet, um (z.\,B.) zus\"atzliche Seitenangaben f\"ur den Literaturverweis 
                              vorzunehmen.
\item[\texttt{par2}:] Der dritte Parameter ist der entsprechende Schl\"ussel in der .bib-Datei, in der die Literaturquellen 
                              beschrieben sind (vgl. \texttt{wa.bib}).                                                                                       
\end{seToplist}

Als Beispiel f\"ur die Verwendung des \verb+\seCite+-Befehls dient folgendes Zitat: \glqq{}Die \textbf{Funktion} eines 
Anhangs einer wissenschaftlichen Arbeit wird sehr h\"aufig \textbf{missdeutet}, der Anhang selbst nicht selten \textbf{mi{\ss}braucht}.\grqq{} 
(\seCite{vgl.}{S. 170}{The:WA}).

Bei der von Theisen vorgeschlagenen Zitierweise erfolgt die Angabe der Literaturverweise in der Regel innerhalb einer Fu{\ss}note. 
Hierf\"ur kann das Kommando \verb+\seFootcite+ verwendet werden, das dieselben Parameter wie \verb+\seCite+ besitzt. 

Als Beispiel f\"ur ein indirektes Zitat l\"asst sich die Aussage von Theisen anf\"uhren, dass Hauptinhalte eines (berechtigten) Anhangs erg\"anzende 
Materialien und Dokumente sind, die weitere themenbezogene Informationen liefern k\"onnen.\seFootcite{Vgl.}{S. 171}{The:WA}

Weder das \verb+\seFootcite+- noch das \verb+\footnote+-Kommande k\"onnen bei Gleitobjekten (Verwendung der \verb+figure+-, \verb+table+- oder 
\verb+programm+-Umgebung) verwendet werden. Ein kleiner Workaround, um \LaTeX{} doch dazu zu bringen, Fu{\ss}noten bei Gleitobjekten 
zu akzeptieren, ist in \vref{gleitobjekte} zu finden.

\input{\seWaPathText/se-zitieren-harvard}

\subsection{Verwendung von URLs}

URLs k\"onnen in der \texttt{bib}-Datei mit \texttt{@WWW} definiert werden. Das Feld \texttt{author} ist zwar optional, sollte aber immer angegeben 
werden, da andernfalls im Literaturverzeichnis der Kurztitel nicht ausgegeben wird. Wenn kein Autor bekannt ist, wird die Abk\"urzung o.\ V. verwendet.
Beim Eintrag in der \texttt{bib}-Datei ist zu beachten, dass diese Abk\"urzung zus\"atzlich eingeklammert werden muss, d.\,h. sie ist in der 
Form \texttt{author = \{\{o.~V.\}\}} anzugeben. 

Das Layout der URL-Angabe im Literaturverzeichnis kann \"uber vier Parameter beeinflusst werden.  
In der Datei \texttt{wa-konfiguration-deutsch.tex} k\"onnen Redefinitionen vorgenommen werden.

\begin{seList}
\item \verb+\biburlprefix+ \newline Text, der vor dem eigentlichen URL-Eintrag ausgegeben wird \newline Standardwert: \glqq{}\jblangle{}URL: \grqq{}
\item \verb+\biburlsuffix+ \newline Text, der hinter dem eigentlichen URL-Eintrag ausgegeben wird \newline Standardwert: \glqq{}\jbrangle{}\grqq{}
\item \verb+\bibbudcsep+ \newline Text zwischen dem eigentlichen URL-Eintrag und der Datumsangabe f\"ur den letzten Zugriff auf die URL
                                         \newline Standardwert: \glqq{} -- \grqq{}
\item \verb+\urldatecomment+ \newline Text, der vor der Datumsangabe f\"ur den letzten Zugriff ausgegeben wird
                                          \newline Standardwert: \glqq{}Zugriff am\grqq{}
\end{seList}

Und hier kommen noch zwei Beispiele f\"ur die Angabe von Literaturreferenzen, deren Quelle eine URL ist:
Das Paket \texttt{jurabib.sty} wurde von Jens Berger entwickelt.\seFootcite{Vgl.}{}{Ber:Hoj} 
\glqq{}Google will seine Suche auch in Deutschland um eine Datenbank mit abgesicherten Fakten, Biografien und Bildern erweitern, 
den Knowledge Graph.\grqq{}\seFootcite{}{}{GKN}


\newcommand{\dateiAbk}{\texttt{wa-abkuerzungen.tex}}
%
% Ein kleiner Text, um Abk\"urzungen, Symbole und Glossareintr\"age zu testen
%
%
\section{Kommandos f\"ur die Erzeugung von Abk\"urzungen, Symbolen und Glos\-sar\-eint\-r\"a\-gen}

\subsection{Definition von Abk\"urzungen, Symbolen und Glossareintr\"agen}

Um eine einheitliche Darstellung von Abk\"urzungen, Symbolen und Glossareintr\"agen zu erreichen, 
werden vier neue Kommandos zur Verf\"ugung gestellt:

\begin{seList}
\item 
\verb+\seNewAcronymEntry+\newline
Definition einer neuen Abk\"urzung.
\item 
\verb+\seNewSymbolEntry+\newline
Definition eines neuen Symbols.
\item
\verb+\seNewGlossaryEntry+\newline
Definition eines neuen Eintrags im Glossar.
\item
\verb+\seNewAcronymGlossaryEntry+\newline
Definition eines neuen Eintrags im Glossar, wobei zus\"atzlich eine 
Abk\"urzung definiert wird, die dann auch in das Abk\"urzungsverzeichnis aufgenommen wird.
\end{seList}

Der Datei \dateiAbk{} lassen sich die zugeh\"origen \textbf{Pa\-ra\-me\-ter\-be\-schrei\-bun\-gen}  
entnehmen.
In dieser Datei sind auch Beispiele enthalten, wie Abk\"urzungen, Symbole und Glossareintr\"age mit den 
Standardkommandos definiert werden k\"onnen, was jedoch nicht empfohlen wird!

\subsection{Verwendung von Abk\"urzungen, Symbolen und Glossareintr\"agen im Text}

Innerhalb des Textes wird f\"ur Abk\"urzungen, Symbole und Glossareintr\"age das Kommando \verb+\gls{par1}+ 
verwendet.
\texttt{par1} stellt einen Schl\"ussel dar, der die entsprechende Definition identifiziert (vgl. den Inhalt der Datei
\dateiAbk{}). 

Mit dem Kommando \verb+\glspl+ ist es m\"oglich, beim
Auftreten eines Begriffes,  f\"ur den ein Glossareintrag existiert, bzw.\ beim (ersten) Auftreten einer 
Abk\"urzung f\"ur die Vollform die  \textbf{Pluralform} auszugeben.%
\footnote{Genauer gesagt wird derjenige Wert ausgegeben, der in den 
Kommandos \texttt{\textbackslash{}seNewAcronymEntry}, \texttt{\textbackslash{}seNewGlossaryEntry} bzw. \texttt{\textbackslash{}seNewAcronymGlossaryEntry}
als Pluralform definiert wurde. Die Pluralform k\"onnte man alternativ verwenden, um beispielsweise eine 
Genitivform zu definieren.}

Bei den Kommandos 
\begin{seList}
\item\verb+\seNewAcronymEntry+ und 
\item\verb+\seNewAcronymGlossaryEntry+
\end{seList}
kann durch die Verwendung des optionalen Parameters 
zu\-s\"atz\-lich eine Pluralform f\"ur die Abk\"urzung definiert werden (vgl. \dateiAbk{}).

\subsection{Anwendungsbeispiele}

\subsubsection{Abk\"urzungen}

Die dreimalige Anwendung von \verb+\gls{usb}+ liefert:

\begin{seList}
\item \gls{usb}
\item \gls{usb}
\item \gls{usb}
\end{seList}

Die Anwendung von \verb+\glspl{dm}+ \verb+\glspl{dm}+ \verb+\gls{dm}+ liefert:

\begin{seList}
\item \glspl{dm}
\item \gls{dm}
\item \gls{dm}
\end{seList}

Und auch die \gls{dhbw} soll noch erw\"ahnt werden, um das Abk\"urzungsverzeichnis ein wenig zu f\"ullen.

\subsubsection{Symbole}

Bei einem Symbol wird -- im Gegensatz zu Abk\"urzungen -- beim ersten Auftreten im Text nicht die 
zugeh\"orige Definition ausgegeben. Diese ist aber im Symbolverzeichnis zu finden.

Die zweimalige Anwendung von \verb+\gls{pi}+ liefert:

\begin{seList}
\item \gls{pi}
\item \gls{pi}
\end{seList}

Und jetzt kommt noch ein zweites Symbol f\"ur das Symbolverzeichnis: \gls{ND}

\subsubsection{Glossareintr\"age}

Bei einem Glossareintrag wird beim ersten Auftreten des Begriffes im Text dieser mit \textsuperscript{GL} markiert.
Im Glossar sind die Seitenzahlen angegeben, auf denen der Begriff verwendet wurde. 

Die dreimalige Anwendung von \verb+\gls{glos:AD}+ liefert:

\begin{seList}
\item \gls{glos:AD}
\item \gls{glos:AD}
\item \gls{glos:AD}
\end{seList}

Und hier kommt noch ein Beispiel f\"ur einen Glossareintrag, f\"ur den beim ersten und dritten Auftreten die Pluralform verwendet 
wird:  \verb+\glspl{glos:bs}+  \verb+\gls{glos:bs}+  \verb+\glspl{glos:bs}+ 

\begin{seList}
\item \glspl{glos:bs}
\item \gls{glos:bs}
\item \glspl{glos:bs}
\end{seList}

\subsubsection{Glossareintrag mit einem zus\"atzlichen Eintrag im Ab\-k\"ur\-zungs\-ver\-zeich\-nis}

Nach der ersten Anwendung des Begriffes, f\"ur den ein Glossareintrag erzeugt wurde, wird in der Folge 
jeweils nur noch die Abk\"urzung benutzt. 

Die Kommandoausf\"uhrungen \verb+\glspl{glos:ma}+ \verb+\gls{glos:ma}+  \verb+\glspl{glos:ma}+ haben als 
Ergebnis:

\begin{seList}
\item \glspl{glos:ma}
\item \gls{glos:ma}
\item \glspl{glos:ma}
\end{seList}

\newpage
Und jetzt wird auf einer neuen Seite nochmals \verb+\gls{glos:ma}+ verwendet, um im Glossar die neu hinzugekommene 
Seitennummer zu demonstrieren: \gls{glos:ma}

\subsubsection{Pluralform von Abk\"urzungen}

\textbf{\textsf{Definition einer Abk\"urzung}}

Der Eintrag wurde wie folgt definiert (vgl. \dateiAbk{}):

\vspace{-\baselineskip}
\begin{verbatim}
   \seNewAcronymEntry[URLs]{url}{URL}{Uniform Resource Locator}%
   {Uniform Resource Locators}
\end{verbatim}
\vspace{-\baselineskip}

Die Kommandoausf\"uhrungen \verb+\glspl{url}+ \verb+\gls{url}+  \verb+\glspl{url}+ haben als 
Ergebnis:

\begin{seList}
\item \glspl{url}
\item \gls{url}
\item \glspl{url}
\end{seList}

\seVsd
\textbf{\textsf{Definition eines Glossareintrags mit zus\"atzlicher Abk\"urzung}}

Der Eintrag wurde wie folgt definiert (vgl. \dateiAbk{}):

\vspace{-\baselineskip}
\begin{verbatim}
   \seNewAcronymGlossaryEntry[TAen]{glos:ta}{TA}{Transaktion}%
   {Transaktionen}%
   {Was eine Transaktion ist, sollten Sie ebenfalls bereits wissen!}
\end{verbatim}
\vspace{-\baselineskip}


Die Kommandoausf\"uhrungen \verb+\glspl{glos:ta}+ \verb+\gls{glos:ta}+  \verb+\glspl{glos:ta}+ haben als 
Ergebnis:

\begin{seList}
\item \glspl{glos:ta}
\item \gls{glos:ta}
\item \glspl{glos:ta}
\end{seList}

%2013-07-08
\subsubsection{Literaturverweise in Glossareinträgen}

Auch bei Glossareinträgen müssen natürlich Literaturverweise angegeben werden. 
Wird eine Literaturquelle erstmalig in einem Glossareintrag verwendet, dann tritt das Problem auf, 
dass sie von BibTeX nicht gefunden wird. Ein \textsl{Workaround} besteht darin, für die entsprechenden 
Literaturverweise \verb+\nocite{key}+-Kommandos anzugeben. \verb+key+ ist hierbei der zugehörige Schlüssel 
des Eintrags in der .bib-Datei.\footnote{Achtung: Ein \texttt{\textbackslash{}nocite}-Kommando sollte nur in absoluten Ausnahmefällen 
eingesetzt werden, da hiermit Einträge im Literaturverzeichnis erzeugt werden können, für die (möglicherweise) kein 
Literaturverweis innerhalb der Arbeit existiert.}

\newpage

 

\section{Fu{\ss}noten}

\subsection{Verwendung dreistelliger Fu{\ss}noten}

Bei dreistelligen Fu{\ss}noten tritt das Problem auf, dass der Abstand zwischen Fu{\ss}notennummer und folgendem Text nicht mehr ausreicht.
Der Abstand kann wie folgt vergr\"o{\ss}ert werden:

\begin{seList}
\item
In der Style-Datei \texttt{se-jb-footmisc.sty} wird \"uber \newline \texttt{\textbackslash{}setlength\textbackslash{}footnotemargin\{0.3cm\}} genau dieser Abstand definiert.
\item
\"Andert man den Wert z.\,B.\ auf 0.5cm, dann sollte es auch f\"ur dreistellige Fu{\ss}noten ausreichen.
\end{seList}

\subsection{Fu{\ss}noten in Abbildungen, Tabellen und Programmlistings}

\LaTeX{} erlaubt generell nicht die Verwendung des Kommandos \verb+\footnote+ in \textsl{Gleitobjekten} (\textsl{Floats}). 
Zu den Gleitobjekten geh\"oren \textsl{Abbildungen}, \textsl{Tabellen} und auch \textsl{Programmlistings}. In \vref{fussnote} findet ein 
kleiner \textsl{Workaround} Anwendung, wie man doch Fu{\ss}noten in Gleitobjekten angeben kann. 

\begin{seList}
\item Mit \verb+\footnotemark+ wird in dem \verb+\caption+-Kommando die \textsl{Fu{\ss}notennummer} erzeugt.
\item Mit dem Kommando \verb+\footnotetext+ wird au{\ss}erhalb der Umgebung, die das Gleitobjekt definiert (z.\,B. die 
\verb+figure+-Umgebung), der Text der Fu{\ss}note festgelegt. Hierbei ist zu beachten, dass ein Gleitobjekt auf die n\"achste Seite 
verschoben werden kann. In einem derartigen Fall sollte der Fu{\ss}notentext an einer Stelle im \LaTeX-Quelltext positioniert werden, die ebenfalls zu dieser Seite geh\"ort.\footnote{Standardm\"a{\ss}ig wird man \texttt{\textbackslash{}footnotetext} direkt hinter dem Gleitobjekt definieren, um sicherzustellen,
dass der Fu{\ss}notentext auch der richtigen Fu{\ss}notennummer zugeordnet wird.}
\end{seList}


\section{Abbildungen, Tabellen und Programmlistings\label{gleitobjekte}}

Ein Rechteck besitzt die in \vref{abb1} dargestellte Struktur.

\begin{figure}[htbp]
\centering
\setlength{\unitlength}{1mm}
\begin{picture}(100,30)
\put(0,0){\framebox(100,30){Ich bin kein Quadrat!}}
\end{picture}
\caption[Die Darstellung eines Rechtecks]{Die Darstellung eines Rechtecks\label{abb1}\footnotemark}
\end{figure}
%\footnotetext{\seCite{Vgl.}{S. 400}{The:WA}. Achtung: Dieser Literaturverweis ist  rein fiktiver Natur, 
%die Seite 400 existiert in \seCite{}{}{The:WA} nicht!}\label{fussnote}

Der optionale Parameter im folgenden \verb+\caption+-Kommando
\footnotetext{\seCite{Vgl.}{S. 400}{The:WA}. Achtung: Dieser Literaturverweis ist  rein fiktiver Natur, 
die Seite 400 existiert in \seCite{}{}{The:WA} nicht!}\label{fussnote}


\vspace*{-\baselineskip}
\begin{verbatim}
\caption[Die Darstellung eines Rechtecks]%
{Die Darstellung eines Rechtecks\label{abb1}\footnotemark}
\end{verbatim}
\vspace*{-\baselineskip}

definiert den Eintrag f\"ur das Abbildungsverzeichnis. Dort sollte die Fu{\ss}notennummer nicht auftauchen.
Nutzt man den optionalen Parameter nicht, ist es notwendig,  vor \verb+\footnotemark+ noch ein \verb+\protect+ 
einzuf\"ugen, da \LaTeX{} andernfalls die \"Ubersetzung mit einer Fehlermeldung abbricht. 

Eine Notentabelle kann wie in \vref{noten} dargestellt aussehen.

\begin{table}[htbp]%
\centering%
\begin{tabular}{| c | c |}
\hline
Matrikelnummer & Note \\
\hline
\hline
1234567 & 2,7 \\
\hline
2323456 & 3,5 \\
\hline
9865783 & 1,0 \\
\hline
\end{tabular} 
\caption{Ergebnisse der Klausur Programmierung I\label{noten}}
\end{table}


Eines der wichtigsten Java-Programme \"uberhaupt ist in \vref{hello} zu sehen.

\begin{programm}[htbp]
\begin{lstlisting}
public class HelloDHBW {
  public static void main ( String[] args ) {
    System.out.println ( "Hello DHBW" );
  } // main
} // HelloDHBW
\end{lstlisting}
\caption{Die Klasse \texttt{HelloDHBW}\label{hello}}
\end{programm}



\newpage
% J\"org Baumgart
% 2012-06-01
%
%
\section{Definition und Erzeugung von Querverweisen}

Die Grundlage f\"ur die Erzeugung eines Querverweises bildet die Definition eines 
\textbf{Labels}, z.\,B. \verb+\label{querverweis1}+\label{querverweis1}.

Mit dem Kommando \verb+\vref+, z.\,B. \verb+\vref{querverweis1}+, wird ein Querverweis mit 
den beiden folgenden Eigenschaften erzeugt:

\begin{seList}
\item 
Falls sich das Label auf eine \textsl{Abbildung}, eine \textsl{Tabelle}, ein \textsl{Listing} oder eine 
\textsl{Gleichung} bezieht, wird zus\"atzlich zur entsprechenden Nummer ein Text mit ausgegeben.
Beispielsweise erzeugt \verb+\vref{noten}+ \vref{noten}. Die zugeh\"origen Labels sind dann innerhalb 
der \verb+figure+-, \verb+table+-, \verb+programm-+ oder \verb+equation+-Umgebung definiert. 
Die auszugebenden Texte k\"onnen in der Datei\newline
\hspace*{\fill}\verb+wa-konfiguration-deutsch.tex+\hspace*{\fill}\newline  
umdefiniert werden.

Bezieht sich ein Label auf eine Textstelle, z.\,B. \verb+\label{querverweis1}+, dann wird die Kapitelnummer 
mit dem Zusatz \textsl{Kapitel} ausgegeben: \vref{querverweis1}\newline
F\"ur die Gliederungsebenen \verb+\chapter+, \verb+\section+, \verb+\subsection+, \verb+\subsubsection+ 
und \verb+\paragraph+ kann dieser \textsl{Zusatz} ebenfalls in der Datei \newline
\hspace*{\fill}\verb+wa-konfiguration-deutsch.tex+\hspace*{\fill}\newline 
umdefiniert werden. 
\item
Wenn sich der Querverweis auf die aktuelle Seite bezieht, dann wird keine Seitennummer ausgegeben.
\end{seList}

Bei der Verwendung des \verb+\vref+-Kommandos ist zu beachten, dass vor dem auszugebenden Text ein Leerzeichen 
eingef\"ugt wird. Im Normalfall hat dieses keine weitere Auswirkung. Wenn allerdings ein Absatz direkt mit einem 
\verb+\vref+-Kommando beginnt, dann wird der entsprechende Text nicht linksb\"undig ausgegeben, d.\,h. es liegt 
eine Verletzung des Blocksatzes vor.%

\vref{noten} stellt einen blocksatzverletzenden Querverweis dar.

Dieses \textsl{Problem} l\"asst sich durch die Anwendung des \verb+\vref*+-Kommandos vermeiden.\label{vrefstern} 

\vref*{noten} stellt einen nicht blocksatzverletzenden Querverweis dar.

Allerdings f\"uhrt die Verwendung des \verb+\vref*+-Kommandos innerhalb eines Satzes auch wieder 
zu einem nicht gew\"unschten Ergebnis: Das in \vref*{noten} dargestellte Klausurergebnis ... .%
\footnote{Der Grund, warum die Kommandos \texttt{\textbackslash{}vref} 
und \texttt{\textbackslash{}vref*}
in dieser Form definiert wurden, erschlie{\ss}t sich dem Autor dieses Dokuments allerdings nicht!}

Mit dem Kommande \verb+\pageref+ wird lediglich die Seitennummer ausgegeben, z.\,B. \verb+\pageref{noten}+ \pageref{noten}
oder \verb+\pageref{querverweis1}+ \pageref{querverweis1}.




\section{Definition und Anwendung von zwei neuen Listenumgebungen}

\subsection{Das Layout der Standardlistenumgebung von \LaTeX}

Stichpunktlisten werden in \LaTeX{} mit der \verb+itemize+-Umgebung erzeugt. 
Die Stichpunktliste 

\begin{itemize}
\item 1. Stichpunkt der ersten Ebene
\begin{itemize}
\item 1. Stichpunkt der zweiten Ebene
\item 2. Stichpunkt der zweiten Ebene
\begin{itemize}
\item 1. Stichpunkt der dritten Ebene
\item 2. Stichpunkt der dritten Ebene
\begin{itemize}
\item 1. Stichpunkt der vierten Ebene
\item 2. Stichpunkt der vierten Ebene
\end{itemize}
\end{itemize}
\end{itemize}
\item 2. Stichpunkt der ersten Ebene
\item 3. Stichpunkt der ersten Ebene
\end{itemize}

wird durch die folgenden Anweisungen erreicht:

\vspace*{-\baselineskip}

\begin{verbatim}
\begin{itemize}
\item 1. Stichpunkt der ersten Ebene
\begin{itemize}
\item 1. Stichpunkt der zweiten Ebene
\item 2. Stichpunkt der zweiten Ebene
\begin{itemize}
\item 1. Stichpunkt der dritten Ebene
\item 2. Stichpunkt der dritten Ebene
\begin{itemize}
\item 1. Stichpunkt der vierten Ebene
\item 2. Stichpunkt der vierten Ebene
\end{itemize}
\end{itemize}
\end{itemize}
\item 2. Stichpunkt der ersten Ebene
\item 3. Stichpunkt der ersten Ebene
\end{itemize}
\end{verbatim}

\subsection{Die neue Listenumgebung \texttt{seList} f\"ur Stichpunktlisten}

Weder die Einr\"uckung der einzelnen Ebenen noch die gro{\ss}en Abst\"ande zwischen den einzelnen Stichpunkten sind bei der \verb+itemize+-Umgebung 
bez\"uglich des Layouts sonderlich \"uberzeugend. 

Daher wird eine neue \verb+seList+-Umgebung zur Verf\"ugung gestellt. 

\begin{seList}
\item 1. Stichpunkt der ersten Ebene
\begin{seList}
\item 1. Stichpunkt der zweiten Ebene
\item 2. Stichpunkt der zweiten Ebene
\begin{seList}
\item 1. Stichpunkt der dritten Ebene
\item 2. Stichpunkt der dritten Ebene
\begin{seList}
\item 1. Stichpunkt der vierten Ebene
\item 2. Stichpunkt der vierten Ebene
\begin{seList}
\item 1. Stichpunkt der f\"unften Ebene
\item 2. Stichpunkt der f\"unften Ebene
\end{seList}
\end{seList}
\end{seList}
\end{seList}
\item 2. Stichpunkt der ersten Ebene
\item 3. Stichpunkt der ersten Ebene
\end{seList}

Der \LaTeX-Quelltext f\"ur diese Liste ist: 

\vspace*{-\baselineskip}
\begin{verbatim}
\begin{seList}
\item 1. Stichpunkt der ersten Ebene
\begin{seList}
\item 1. Stichpunkt der zweiten Ebene
\item 2. Stichpunkt der zweiten Ebene
\begin{seList}
\item 1. Stichpunkt der dritten Ebene
\item 2. Stichpunkt der dritten Ebene
\begin{seList}
\item 1. Stichpunkt der vierten Ebene
\item 2. Stichpunkt der vierten Ebene
\begin{seList}
\item 1. Stichpunkt der f\"unften Ebene
\item 2. Stichpunkt der f\"unften Ebene
\end{seList}
\end{seList}
\end{seList}
\end{seList}
\item 2. Stichpunkt der ersten Ebene
\item 3. Stichpunkt der ersten Ebene
\end{seList}
\end{verbatim}

\vspace*{-\baselineskip}
Neben der Eigenschaft, im Gegensatz zur \verb+itemize+-Umgebung f\"unf Verschachtelungsebenen angeben zu k\"onnen, ist es m\"oglich,
die Zeilenabst\"ande f\"ur die einzelnen Ebenen zu konfigurieren. 

Mit dem Kommando \newline 
\hspace*{\fill}\verb+\seSetlistbaselineskip{b1}{b2}{b3}{b4}{b5}+\hspace*{\fill}\newline 
kann f\"ur die Verschachtelungsebene $i$ der Grundlinienabstand \texttt{b$_{i}$} festgelegt 
werden. Als Einheit wird der Wert von \verb+\baselineskip+ (Grundlinienabstand des Dokuments) verwendet. Die folgenden Werte sind f\"ur ein Dokument voreingestellt:\newline
\hspace*{\fill}\verb+\seSetlistbaselineskip{1}{0.75}{0.75}{0.75}{0.75}+\hspace*{\fill}\newline\vspace*{-\baselineskip}

Mit dem Kommando \newline
\hspace*{\fill}\verb+\seResetlistbaselineskip{}+\hspace*{\fill}\newline
wird die letzte \"Anderung der Werte r\"uckg\"angig gemacht.

\newpage
\subsection{Die neue Listenumgebung \texttt{seToplist} f\"ur Listen mit einem Label und Aufz\"ahlungslisten}

Die neue Listenumgebung \verb+seToplist+ erlaubt es, jeden Stichpunkt mit einem Label zu versehen.
Die Liste\footnote{Die folgenden Werte sind frei erfunden.} 

\begin{seToplist}{Mercedes Benz:}
\item[Audi:] 400000 Gesamtverk\"aufe
\begin{seToplist}{3er Reihe:}
\item[A4:] 200000 Verk\"aufe
\item[A5:] 50000 Verk\"aufe
\item[A6:] 150000 Verk\"aufe
\end{seToplist}
\item[Mercedes Benz:] 500000 Gesamtverk\"aufe 
\item[BMW:] 650000 Gesamtverk\"aufe 
\begin{seToplist}{3er Reihe:}
\item[1er Reihe:] 100000 Verk\"aufe
\item[3er Reihe:] 300000 Verk\"aufe
\item[5er Reihe:] 250000 Verk\"aufe
\end{seToplist}
\end{seToplist}

wird durch die folgenden \LaTeX-Anweisungen erzeugt:

\vspace*{-\baselineskip}
\begin{verbatim}
\begin{seToplist}{Mercedes Benz:}
\item[Audi:] 400000 Gesamtverk\"aufe
\begin{seToplist}{3er Reihe:}
\item[A4:] 200000 Verk\"aufe
\item[A5:] 50000 Verk\"aufe
\item[A6:] 150000 Verk\"aufe
\end{seToplist}
\item[Mercedes Benz:] 500000 Gesamtverk\"aufe 
\item[BMW:] 650000 Gesamtverk\"aufe 
\begin{seToplist}{3er Reihe:}
\item[1er Reihe:] 100000 Verk\"aufe
\item[3er Reihe:] 300000 Verk\"aufe
\item[5er Reihe:] 250000 Verk\"aufe
\end{seToplist}
\end{seToplist}
\end{verbatim}

\vspace*{-\baselineskip}
Der Parameter \verb+par+ von \verb+\begin{seToplist}{par}+ definiert die Breite des Labels f\"ur die 
zugeh\"orige Liste.

F\"ur die \verb+seToplist+-Umgebung k\"onnen ebenfalls f\"unf Verschachtelungsebenen definiert werden. 
\"Uber die Kommandos \newline
\hspace*{\fill}\verb+\seSettoplistbaselineskip{b1}{b2}{b3}{b4}{b5}+\hspace*{\fill}\newline 
bzw. \newline
\hspace*{\fill}\verb+\seResettoplistbaselineskip{}+\hspace*{\fill}\newline
lassen sich analog zur \verb+seList+-Umgebung die Grundlinienabst\"ande der einzelnen Verschachtelungsebenen 
ver\"andern bzw. zur\"ucksetzen. Die folgenden Werte sind f\"ur ein Dokument voreingestellt:\newline
\hspace*{\fill}\verb+\seSettoplistbaselineskip{1}{0.75}{0.75}{0.75}{0.75}+\hspace*{\fill}\newline\vspace*{-\baselineskip}

Durch eine entsprechende Wahl der Labels k\"onnen Aufz\"ahlungslisten erzeugt werden:

\begin{seToplist}{a)}
\item[a)] Deutsche Automarken
\begin{seToplist}{1)}
\item[1)] Mercedes Benz
\item[2)] Audi 
\item[3)] VW
\item[4)] BMW 
\end{seToplist}
\item[b)] Japanische Automarken
\begin{seToplist}{1)}
\item[1)] Toyota
\item[2)] Honda
\item[3)] Mazda
\end{seToplist}
\end{seToplist}






\newpage
\section{\"Anderung der Schrifttypen im Dokument}

Standardm\"a{\ss}ig wird in dieser Vorlage f\"ur die \"Uberschriften, die Kopf- und Fu{\ss}zeilen 
sowie das Titelblatt eine serifenlose Schrift verwendet, w\"ahrend der Textteil in einer Serifenschrift 
gesetzt ist.

Soll das gesamte Dokument in einer \textsf{\textbf{serifenlosen Schrift}} gesetzt werden, dann ist in 
der Konfigurationsdatei \verb+wa-konfiguration.tex+ das Kommando \newline
\hspace*{\fill}\verb+\renewcommand{\familydefault}{\sfdefault}+\hspace*{\fill}\newline
zu verwenden.

Soll das gesamte Dokument in einer \textbf{Serifenschrift} gesetzt werden, dann ist in 
der Konfigurationsdatei \verb+wa-konfiguration.tex+ das Kommando \newline
\hspace*{\fill}\verb+\renewcommand{\sffamily}{\normalfont}+\hspace*{\fill}\newline
zu verwenden. Nach dieser \"Anderung ist es nicht mehr m\"oglich, \"uber das 
Kommando \verb+\textsf{}+ einen Textteil in einer serifenlosen Schrift zu setzen.




\section{Anpassungen des Gesamtlayouts}

\subsection{\"Anderung des vertikalen Zwischenraums beim Start eines neuen Kapitels}

Um den vertikalen Zwischenraum zu ver\"andern, den \LaTeX{} automatisch beim 
Start eines neuen Kapitels erzeugt, kann das Kommando \verb+\seNoChapterSkip+ 
verwendet werden. Dieses Kommando wird direkt vor \verb+\begin{document}+ eingef\"ugt.
Es besitzt einen optionalen Parameter, \"uber den ein Wert angegeben werden kann. Der 
Defaultwert ist \texttt{-14mm}. Damit wird erreicht, dass bei einem neuen Kapitel kein 
zus\"atzlicher vertikaler Zwischenraum eingef\"ugt wird. 

Beispiele:

\begin{seList}
\item
\verb+\seNoChapterSkip{}+\newline \verb+\begin{document}+ \newline
Es wird kein vertikaler Zwischenraum beim Beginn eines neuen Kapitels erzeugt.
\item
\verb+\seNoChapterSkip[11.5mm]+\newline \verb+\begin{document}+ \newline
Es wird der vertikale Zwischenraum erzeugt, der auch ohne Angabe dieses 
Kommandos Verwendung findet.
\item
\verb+\seNoChapterSkip[21.5mm]+\newline \verb+\begin{document}+ \newline
Im Vergleich zu dem standardm\"a{\ss}ig erzeugten vertikalen Zwischenraum 
wird ein 10\,mm gr\"o{\ss}erer Zwischenraum beim Start eines neuen Kapitels 
erzeugt.
\end{seList}

\subsection{M\"ogliche Layout-\"Anderungen f\"ur Seminararbeiten}

\subsubsection{Verwendung kleinerer Schriftgr\"o{\ss}en f\"ur \"Uberschriften}

Die Verwendung kleinerer Schriftgr\"o{\ss}en f\"ur \"Uberschriften wird durch 
die Angabe des Kommandos \verb+\KOMAoption{headings}{small}+ direkt 
vor \verb+\begin{document}+ erreicht.

Soll dieses Kommando bei Seminarbeiten mit \verb+\seNoChapterSkip{}+ kombiniert 
werden, ist die folgende Reihenfolge erforderlich:

\begin{seList}
\item[] \verb+\KOMAoption{headings}{small}+\newline
\verb+\seNoChapterSkip[-12.25mm]+\newline
\verb+\begin{document}+
\end{seList}

Da kleinere Schriftgr\"o{\ss}en f\"ur die \"Uberschriften verwendet werden, sollte das Kommando \verb+\seNoChapterSkip+ mit 
dem optionalen Parameter \texttt{-12.25mm} aufgerufen werden.

\subsubsection{Unterdr\"uckung des Seitenvorschubs f\"ur die folgenden Kapitel}

Das Kommando \verb+\seChaptersWithoutNewpage{}+ unterdr\"uckt den Seitenvorschub des \verb+\chapter+-Kom\-man\-dos f\"ur die folgenden Kapitel. 

Wenn dieses Kommando in Kombination mit \verb+\seNoChapterSkip{}+ benutzt wird, dann sollte nach jedem Kapitelende noch das Kommando 
\verb+\seChapterEndSkip{}+ ausgef\"uhrt werden, damit ein vern\"unftiger Abstand zur folgenden Kapitel\"uberschrift entsteht.

\verb+\seChapterNewpage{}+ erzeugt f\"ur die folgenden Kapitel wieder Seitenvorsch\"ube.






%
%  Erzeugung eines Glossars
%
% Achtung: Das Glossar wird nur ausgegeben, wenn mindestens ein Eintrag in der Arbeit 
%                definiert wurde
%
%
\newpage
\sePrintGlossary{}


%
% Literaturverzeichnisses
%
%\newpage
\sePrintBibliography{}


%
% Festlegung des grundlegenden Formatierungsstils des Literaturverzeichnis
%
\bibliographystyle{jurabib}

% Eigentliche Ausgabe der in der Arbeit verwendeten Quellen
%
%
% Angabe der bib-Dateien, in denen die Quellen beschrieben sind;
% die Angabe geht davon aus, dass eine wa.bib-Datei in demselben 
% Verzeichnis liegt, wie se-ba-vorlage.tex
%

% 2016-04-01
%
% Umbenennung von Quellen- in Literaturverzeichnis (nicht empfohlen, da sich 
% die 
% 
%\renewcommand*{\bibname}{Quellenverzeichnis}
\seBibliography{wa}

\end{document}











