\subsection{Harvard-Zitierweise}

Um die Harvard-Zitierweise anzuwenden, muss in der Konfigurationsdatei \newline
\hspace*{\fill}\verb+wa-konfiguration+\hspace*{\fill}\newline
die Zeile \newline
\hspace*{\fill}\verb+\usepackage{\seWaPathSty/se-jb-jurabib-theisen}+\hspace*{\fill}\newline
durch\newline
\hspace*{\fill}\verb+\usepackage{\seWaPathSty/se-jb-jurabib-harvard}+\hspace*{\fill}\newline
ausgetauscht werden. Das Literaturverzeichnis wird dann ohne die Angabe von 
Kurztiteln ausgegeben und die Autorennamen werden im Text nicht mehr kursiv 
dargestellt.

F\"ur das Zitieren im Text werden dann die Kommandos \verb+\citep+ bzw. \verb+\citealt+ 
verwendet.

\begin{seList}
\item \verb+\citep{Bri:WA}+ $\rightarrow$ \citep{Bri:WA}.
\item Optionale Angabe einer Seitenzahl: \verb+\citep[S.\,45]{Bri:WA}+ \newline$\rightarrow$ \citep[S.\,45]{Bri:WA}.
\item Optionale Angabe von vgl.: \verb+\citep[vgl.][]{Bri:WA}+ $\rightarrow$ \citep[vgl.][]{Bri:WA}.
\item Optionale Angabe von Seitenzahl und vgl.: \verb+\citep[vgl.][S.\,45]{Bri:WA}+\newline$\rightarrow$ \citep[vgl.][S.\,45]{Bri:WA}.
\item Angabe einer Liste von Referenzen: \verb+\citep[vgl.][]{Bri:WA,RP:WA}+\newline $\rightarrow$ \citep[vgl.][]{Bri:WA,RP:WA}.
\item Wenn man keine Klammern haben m\"ochte: \newline\verb+\citealt[vgl.][S.\ 44]{Bri:WA}+ $\rightarrow$ 
        \citealt[vgl.][S.\ 44]{Bri:WA}.
\item Damit kann man dann Verweise selbst zusammenbauen:
         \newline\verb+(\citealt[vgl.][S.\,48]{Bri:WA} und \citealt[S.\,96]{RP:WA})+      
         \newline$\rightarrow$ (\citealt[vgl.][S.\,48]{Bri:WA} und \citealt[S.\,96]{RP:WA}).
\end{seList}
