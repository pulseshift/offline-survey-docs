\chapter{Tipps und Tricks}

\section{Verwendung einer Listenumgebung in einer Tabelle}

Um z.\,B. die \verb+seList+-Umgebung in einer Tabelle verwenden zu k\"onnen, ist es notwendig, die Liste 
in eine \verb+\parbox+ zu integrieren, da andernfalls die \"Ubersetzung des Dokumentes mit einer 
Fehlermeldung abgebrochen wird. 

\vref*{quelltext-tab-ereignisse} enth\"alt den Quelltext f\"ur die \vref{tab-ereignisse}. Da das Kommando 
\newline\hspace*{\fill}\verb+\seSetlistbaselineskip+\hspace*{\fill}\newline in einer 
\verb+\parbox+ nicht funktioniert, muss das \textbf{Spacing} der Listeneintr\"age per Hand durch entsprechende
\verb+\vspace*+-Kommandos vorgenommen werden.
 
\vspace{\baselineskip}
\begin{table}[htbp]
\centering
\begin{tabular}{| c | c |}
\hline
Ereignis & Eventuelle Folgen/Kommentar \\
\hline
Konjunkturkrise &
\parbox[t]{6cm}{%
\begin{seList}
\vspace*{-0.55\baselineskip}
\item Produktionsverminderung
\vspace*{-0.5\baselineskip}
\item Entlassung von Mitarbeitern
\vspace*{-0.5\baselineskip}
\item Es ist aber in der Regel keine gute Idee, sich von
hochqualifizierten Mitarbeitern zu trennen
\vspace*{0.25\baselineskip}
\end{seList}
} % \parbox
\\
\hline
\end{tabular}
\caption{Ereignisse und ihre Folgen\label{tab-ereignisse}}
\end{table} 
 

\clearpage
\begin{programm}[htbp]
\begin{lstlisting}[keywordstyle=\color{black}]
\begin{table}[htbp]
\centering
\begin{tabular}{| c | c |}
\hline
Ereignis & Eventuelle Folgen/Kommentar \\
\hline
Konjunkturkrise &
\parbox[t]{6cm}{%
\begin{seList}
\vspace*{-0.55\baselineskip}
\item Produktionsverminderung
\vspace*{-0.5\baselineskip}
\item Entlassung von Mitarbeitern
\vspace*{-0.5\baselineskip}
\item Es ist aber in der Regel keine gute Idee, sich von
hochqualifizierten Mitarbeitern zu trennen
\vspace*{0.25\baselineskip}
\end{seList}
} % \parbox
\\
\hline
\end{tabular}
\caption{Ereignisse und ihre Folgen\label{tab-ereignisse}}
\end{table} 
\end{lstlisting}
\caption{Der Quelltext f\"ur die \vref{tab-ereignisse}\label{quelltext-tab-ereignisse}}
\end{programm}

\newpage
\section{Mehrseitige Listings}

Wenn die \verb+programm+-Umgebung verwendet wird, ist es nicht m\"oglich, mehrseitige 
Listings zu erzeugen.\footnote{Dies liegt in der Tatsache begr\"undet, dass sich Gleitobjekte (floats) nicht \"uber mehrere Seiten 
erstrecken k\"onnen.} 
Benutzt man die \verb+lstlisting+-Umgebung ohne die \verb+programm+-Umgebung, k\"onnen zwar einerseits mehrseitige Listings 
erzeugt werden, andererseits ist es aber auf direktem Weg nicht m\"oglich, die Listingunterschrift in das Listingverzeichnis aufzunehmen.%
\footnote{Von der Verwendung des \texttt{\textbackslash{}lstlistoflistings}-Kommandos wird \textbf{dringend abgeraten}, da es keine Kontrolle \"uber 
das Layout des Listings-Verzeichnisses erlaubt und nicht zum Layout der anderen Verzeichnisse passt.} 

Eine L\"osung f\"ur diese Problematik stellt das
neue Kommando \verb+\captionListing+ dar. Es besitzt einen Parameter, der den Wert f\"ur das ebenfalls neue Kommando \verb+\captionListingText+ 
definiert. Das \verb+\captionListing+-Kommando muss direkt vor einer Verwendung der \verb+lstlisting+-Umgebung angegeben werden. Es sorgt unter anderem 
daf\"ur, dass die \"uber \verb+\captionListingText+ definierte Listingunterschrift in das Listingverzeichnis aufgenommen wird.

Das \vref{lis-mehrseitig} wird durch den Quelltext aus \vref{quelltext-lis-mehrseitig} erzeugt.


\captionListing{Beispiel f\"ur ein mehrseitiges Listing}
\begin{lstlisting}[caption=\captionListingText,
       label=lis-mehrseitig,
       abovecaptionskip=3mm,aboveskip=\parskip]
zeile 01;
zeile 02;
zeile 03;
zeile 04;
zeile 05;
zeile 06;
zeile 07;
zeile 08;
zeile 09;
zeile 10;
zeile 11;
zeile 12;
zeile 13;
zeile 14;
zeile 15;
zeile 16;
zeile 17;
zeile 18;
zeile 19;
zeile 20;
zeile 21;
zeile 22;
zeile 23;
zeile 24;
\end{lstlisting}

\vspace*{\baselineskip}
\begin{programm}[htbp]
\begin{verbatim}
\captionListing{Beispiel f\"ur ein mehrseitiges Listing}
\begin{lstlisting}[caption=\captionListingText,
       label=lis-mehrseitig,
       abovecaptionskip=3mm,aboveskip=\parskip]
zeile 01;
zeile 02;
zeile 03;
zeile 04;
zeile 05;
zeile 06;
zeile 07;
zeile 08;
zeile 09;
zeile 10;
zeile 11;
zeile 12;
zeile 13;
zeile 14;
zeile 15;
zeile 16;
zeile 17;
zeile 18;
zeile 19;
zeile 20;
zeile 21;
zeile 22;
zeile 23;
zeile 24;
\end{lstlisting}
\end{verbatim}
\vspace*{-3.5mm}
\caption{Quelltext von \vref{lis-mehrseitig}\label{quelltext-lis-mehrseitig}}
\end{programm}

\clearpage
Die optionalen Parameter der \verb+lstlisting+-Umgebung haben hierbei die folgende Bedeutung:

\begin{seList}
\item \verb+caption=\captionListingText+

Festlegung der Listingunterschrift; hierf\"ur sollte grunds\"atzlich das Kommando \verb+\captionListingText+ verwendet werden.

\item \verb+label=lis-mehrseitig+

Definition eines Labels, mit dem sp\"ater, z.\,B. \"uber das \verb+\vref+-Kommando, auf die Listingnummer zugegriffen werden kann. \newline 
Dies entspricht dem \verb+\label{}+-Kommando, das beispielsweise bei einer \verb+programm+-Umgebung verwendet wird.

\item \verb+abovecaptionskip=3mm+

Definition eines zus\"atzlichen Abstands zwischen dem eigentlichen Listing und der Listingunterschrift.

\item \verb+aboveskip=\parskip+ 

Vor dem Listing wird ein zus\"atzlicher Abstand erzeugt, der dem Abstand zwischen zwei Abs\"atzen entspricht.
\end{seList}

Durch die Verwendung des \verb+\captionListing+-Kommandos wird sichergestellt, dass die \verb+lstlisting+-Umgebung und die 
\verb+programm+-Umgebung gemeinsam in einem Dokument verwendet werden k\"onnen.

