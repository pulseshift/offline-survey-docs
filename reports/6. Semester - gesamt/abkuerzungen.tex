% 2012-03-22 Verwendung des optionalen Parameters f\"ur die Pluralform einer Abk\"urzung
%
% 2012-02-06 Umstellung auf die neuen Kommandos
%
%
%
%  J\"org Baumgart
%  Definition einiger Abk\"urzungen
%  


% Definition von Abk\"urzungen
%
% 1. Parameter: Schluessel (key) der Abkuerzung
% 2. Parameter: Abkuerzung
% 3. Parameter: Vollform
% 4. Parameter: Vollform im Plural (optional; falls nicht definiert, wird der Wert des dritten Parameters verwendet)
%

\seNewAcronymEntry{epk}{EPK}{Ereignisgesteuerte Prozesskette}{Ereignisgesteuerte Prozesskette}

\seNewAcronymEntry{poc}{PoC}{Proof of Concept}{Proof of Concept}

\seNewAcronymEntry{psp}{PSP}{Projektstrukturplan}{Projektstrukturplan}

\seNewAcronymEntry{pwa}{PWA}{Progressive Web App}{Progressive Web App}


% Definition von Glossareintraegen
%
% 1. Parameter: Schluessel (key) des Glossareintrags
% 2. Parameter: Begriff, der im Glossar definiert wird
% 3. Parameter: Pluralform des Begriffes (optional; falls nicht definiert, wird der Wert des zweiten Parameters verwendet)
%                        Achtung: Pluralform gilt nur fuer das erste Auftreten des Begriffes im Text
% 4. Parameter: Beschreibung des Glossareintrags
%
%
%

\seNewGlossaryEntry{glos:offline_worker}{Offline-Mitarbeiter}{Offline-Mitarbeiter}{Mitarbeiter eines Unternehmens, dem keine Firmen-E-Mail-Adresse zugeordnet ist und der auch keinen anderen Zugang zu einem System hat, welches das Empfangen von mitarbeiterspezifischen digitalen Nachrichten erm\"oglicht.}

\seNewGlossaryEntry{glos:single_purpose_web_app}{Single-Purpose-Web-App}{Single-Purpose-Apps}{Eine Webapplikation, die nur zu einem bestimmten Zweck dient, z.B. zur Anzeige des Schichtplans oder des Lunchmenüs}

